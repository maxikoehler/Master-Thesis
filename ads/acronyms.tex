
%%%%%%%%%%%%%%%%%%%%%%%%%%%%%%%%%%%%%%%%%%%%%%%%%%%%%%%%%%%%%%%%
% Anmerkungen zur Verwendung:
%%%%%%%%%%%%%%%%%%%%%%%%%%%%%%%%%%%%%%%%%%%%%%%%%%%%%%%%%%%%%%%%
%
% nur verwendete Akronyme werden letztlich im Abkürzungsverzeichnis des Dokuments angezeigt
% Verwendung: 
%		\ac{Abk.}   --> fügt die Abkürzung ein, beim ersten Aufruf wird zusätzlich automatisch die ausgeschriebene Version davor eingefügt bzw. in einer Fußnote (hierfür muss in header.tex \usepackage[printonlyused,footnote]{acronym} stehen) dargestellt
%		\acs{Abk.}   -->  fügt die Abkürzung ein
%		\acf{Abk.}   --> fügt die Abkürzung UND die Erklärung ein
%		\acl{Abk.}   --> fügt nur die Erklärung ein
%		\acp{Abk.}  --> gibt Plural aus (angefügtes 's'); das zusätzliche 'p' funktioniert auch bei obigen Befehlen
%	siehe auch: http://golatex.de/wiki/%5Cacronym
%
%%%%%%%%%%%%%%%%%%%%%%%%%%%%%%%%%%%%%%%%%%%%%%%%%%%%%%%%%%%%%%%%
\cleardoublepage
\addcontentsline{toc}{chapter}{Acronyms}
\chapter*{Acronyms}

% \addchap{\langabkverz}
\begin{acronym}[mmmmmm] %hier längstes Acro
% \begin{doublespacing}
% \setlength{\itemsep}{-\parsep}

\acro{AC}{Alternating Current}
\acro{BESS}{Battery Energy Storage System}
\acro{CCT}{Critical Clearing Time}
\acro{CIGRE}{Conseil International des Grands Réseaux Électriques}
\acro{CSI}{Contingency Severity Index}
\acro{DC}{Direct Current}
\acro{EMT}{Electromagnetic Transient}
\acro{FRT}{Fault-Ride-Through}
\acro{FSM}{Fast Switching Module}
\acro{GOV}{Governor}
\acro{HV}{High Voltage}
\acro{IBB}{Infinite Bus Bar}
\acro{IEEE}{Institute of Electrical and Electronics Engineers}
\acro{IM}{Induction Machine}
\acro{LV}{Low Voltage}
\acro{ODE}{Ordinary Differential Equation}
\acro{OLTC}{On-Load Tap Changer}
\acro{PCC}{Point of Common Coupling}
% \acro{PSS}{Power System Simulation}
\acro{PSS}{Power System Stabilizer} % Das ist die eigentliche korrekte Bedeutung/Abkürzung!!!
\acro{RMS}{Root Mean Square}
\acro{SEXS}{Simple Exciter System}
\acro{SG}{Synchronous Generator}
\acro{SMIB}{Single Machine Infinite Bus}
\acro{TDS}{Time Domain Solution}
\acro{TVI}{Trajectory Violation Integral}
% \acro{TVI}{Tangent Vector Index}
\acro{VSC}{Variable Shunt Controller}
\acro{WAMPAC}{Wide-Area Monitoring Protection and Control}

\end{acronym}

%%%%%%%%%%%%%%%%%%%%%%%%%%%%%%%%%%%%%%%%%%%%%%%%%%%%%%%%%%%%%%%%
\addcontentsline{toc}{chapter}{Symbols}	
\chapter*{Symbols}
\label{chap:symbols}

% \addchap{Symbols}
\begin{tabbing}
    XXXXXXXXX \= XXXXXXXX \= XXXXXXXXXXXXXXXXXXXXXXXXXXXXXXXXXXXXXXXXXXXXXXXXX \kill
    $\delta$                \> $^\circ$ / deg                   \> Voltage phase angle \\
    $\phi$                  \> $^\circ$ / deg                   \> Power angle or power factor (as $\cos$, $\sin$, or $\tan$) \\
    $\omega$                \> $\mathrm{\frac{1}{s}}$           \> Machine rotor speed \\
    $\underline{\vartheta}$ \> -                                \> Transformer ratio; complex if phase shifting \\
    % $A$                     \> -                                \> acceleration or deceleration area \\
    $\underline{E}$         \> V                                \> Reference voltage \\
    $H_\mathrm{gen}$        \> s                                \> Inertia constant of a \acf{SG} \\
    $\underline{I}$         \> A                                \> Current \\
    $P$                     \> W                                \> Active power\\
    $Q$                     \> var                              \> Reactive power \\
    $R$                     \> $\mathrm{\Omega}$                \> Ohmic resistance \\
    $\underline{S}$         \> VA                               \> Apparent power \\
    $\underline{V}$         \> V                                \> Voltage \\
    $\underline{X}$         \> $\mathrm{\Omega}$                \> Reactance \\
    $\underline{Y}$         \> $\mathrm{\frac{1}{\Omega}}$ / S  \> Admittance \\
    $\underline{Z}$         \> $\mathrm{\Omega}$                \> Impedance \\
\end{tabbing}

% The different symbols are used with different indices, these are semantic and explained in the surrounding context.
Following notation is commonly used for mathematical and physical symbols:
\begin{itemize}[noitemsep]
    \item Phasors or complex quantities are underlined (e.g. $\underline{I}$)
    \item Arrows on top mark a spatial vector (e.g. $\overrightarrow{F}$)
    \item Boldface upright denotes matrices or vectors (e.g. $\mab{F}$)
    \item Roman typed symbols are units (e.g. $\mathrm{s}$)
    \item Lower case symbols denote instantaneous values (e.g. $\underline{i}$)
    \item Upper case symbols denote \acs{RMS} or peak values (e.g. $\underline{I}$)
    % \item References to objects are written capitalized Roman (e.g. $\underline{Z}_\mathrm{TRAFO}$)
    \item Subscripts relating to physical quantities or numerical variables are written italic (e.g. $\underline{I}_1$) 
    \item Boldface italic denotes sets (e.g. $\boldsymbol{R}$)
\end{itemize}

In the simulations and calculations the per unit system ($\mathrm{p.u.}$) is preferred, thus normalizing all values with a base value. 
% Where necessary, absolute units are added to indicate the explicit use of the normal unit system. 
For more information about this per-unit system please refer to \textcite{machowski_2020}, specifically Appendix A.1 provides a detailed description and explanation. Additionally, \textcite{glover_2017a}, chapter 3.3 can be considered with some transformer specific calculations.