
%%%%%%%%%%%%%%%%%%%%%%%%%%%%%%%%%%%%%%%%%%%%%%%%%%%%%%%%%%%%%%%%
% Anmerkungen zur Verwendung:
%%%%%%%%%%%%%%%%%%%%%%%%%%%%%%%%%%%%%%%%%%%%%%%%%%%%%%%%%%%%%%%%
%
% nur verwendete Akronyme werden letztlich im Abkürzungsverzeichnis des Dokuments angezeigt
% Verwendung: 
%		\ac{Abk.}   --> fügt die Abkürzung ein, beim ersten Aufruf wird zusätzlich automatisch die ausgeschriebene Version davor eingefügt bzw. in einer Fußnote (hierfür muss in header.tex \usepackage[printonlyused,footnote]{acronym} stehen) dargestellt
%		\acs{Abk.}   -->  fügt die Abkürzung ein
%		\acf{Abk.}   --> fügt die Abkürzung UND die Erklärung ein
%		\acl{Abk.}   --> fügt nur die Erklärung ein
%		\acp{Abk.}  --> gibt Plural aus (angefügtes 's'); das zusätzliche 'p' funktioniert auch bei obigen Befehlen
%	siehe auch: http://golatex.de/wiki/%5Cacronym
%
%%%%%%%%%%%%%%%%%%%%%%%%%%%%%%%%%%%%%%%%%%%%%%%%%%%%%%%%%%%%%%%%
\cleardoublepage
\addcontentsline{toc}{chapter}{Acronyms}
\chapter*{Acronyms}

% \addchap{\langabkverz}
\begin{acronym}[mmmmmm] %hier längstes Acro
% \begin{doublespacing}
% \setlength{\itemsep}{-\parsep}

\acro{CCT}{Critical Clearing Time}
\acro{EMT}{Electromagnetic Transient}
\acro{FSM}{Fast Switching Module}
\acro{IBB}{Infinite Bus Bar}
\acro{IEEE}{Institute of Electrical and Electronics Engineers}
\acro{IM}{Induction Machine}
\acro{ODE}{Ordinary Differential Equation}
\acro{OLTC}{On-Load Tap Changer}
\acro{PSS}{Power System Simulation}
% \acro{PSS}{Power System Stabilizer} % Das ist die eigentliche korrekte Bedeutung/Abkürzung!!!
\acro{RMS}{Root Mean Square}
\acro{SG}{Synchronous Generator}
\acro{SMIB}{Single Machine Infinite Bus}
\acro{TDS}{Time Domain Solution}
\acro{TVS}{Tangent Vector Index}

\end{acronym}

%%%%%%%%%%%%%%%%%%%%%%%%%%%%%%%%%%%%%%%%%%%%%%%%%%%%%%%%%%%%%%%%
\addcontentsline{toc}{chapter}{Symbols}	
\chapter*{Symbols}

% \addchap{Symbols}
\begin{tabbing}
    XXXXXXXXX \= XXXXXXXX \= XXXXXXXXXXXXXXXXXXXXXXXXXXXXXXXXXXXXXXXXXXXXXXXXX \kill
    $\delta$            \> $^\circ$ / deg                   \> power angle (or power angle difference) \\
    $\Delta\omega$      \> $\mathrm{\frac{1}{s}}$           \> change of rotor angular speed \\
    $\underline{\theta}$\> -                                \> transformer ratio; complex if phase shifting \\
    $A$                 \> -                                \> acceleration or deceleration area \\
    $\underline{E}$     \> V                                \> voltage of \acs{SG} or \acs{IBB} \\
    $H_\mathrm{gen}$    \> s                                \> inertia constant of a \acf{SG} \\
    $\underline{I}$     \> A                                \> current \\
    $P$                 \> W                                \> effective power; electrical or mechanical \\
    $Q$                 \> var                              \> reactive power \\
    $R$                 \> $\mathrm{\Omega}$                \> ohmic resistance \\
    $\underline{S}$     \> VA                               \> apparent power \\
    $\underline{V}$     \> V                                \> voltage \\
    $\underline{X}$     \> $\mathrm{\Omega}$                \> reactance \\
    $\underline{Y}$     \> $\mathrm{\frac{1}{\Omega}}$ / S  \> admittance \\
    $\underline{Z}$     \> $\mathrm{\Omega}$                \> impedance \\
\end{tabbing}

The different symbols are used with different indices, these are semantic and explained in the surrounding context. Following notation is commonly used for mathematical and physical symbols:
\begin{itemize}[noitemsep]
    \item Phasors or complex quantities are underlined (e.g. $\underline{I}$)
    \item Arrows on top mark a spatial vector (e.g. $\overrightarrow{F}$)
    \item Boldface denotes matrices or vectors (e.g. $\mab{F}$)
    \item Roman typed symbols are units (e.g. $\mathrm{s}$)
    \item Lower case symbols denote instantaneous values (e.g. $\underline{i}$)
    \item Upper case symbols denote \acs{RMS} or peak values (e.g. $\underline{I}$)
    % \item References to objects are written capitalized Roman (e.g. $\underline{Z}_\mathrm{TRAFO}$)
    \item Subscripts relating to physical quantities or numerical variables are written italic (e.g. $\underline{I}_1$) 
\end{itemize}

In the simulations and calculations the per unit system ($\mathrm{p.u.}$) is preferred, thus normalizing all values with a base value. Where necessary, absolute units are added to indicate the explicit use of the normal unit system. For more information about this per-unit system please refer to \textcite{machowskiPowerSystemDynamics2020}, specifically Appendix A.1 provides a detailed description and explanation.