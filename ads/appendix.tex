% !TeX root = ../main.tex

% \appendixtoc
% \appendix
% \ohead[]{\textsc{Appendix}}
\label{app:appendix}

% \renewcommand\thechapter{\roman{chapter}}
% \setcounter{chapter}{0}

% \pagebreak
% \includepdf[pages=-,scale=.9,pagecommand={}]{Aufgabenstellung.pdf} 
% PDF um 10% verkleinert einbinden --> Kopf- und Fußzeile  werden so korrekt dargestellt. Die Option `pages' ermöglicht es, eine bestimmte Sequenz von Seiten (z.B. 2-10 oder `-' für alle Seiten) auszuwählen.
% \pagebreak
%\includepdf[pages=-,scale=.8,pagecommand=\section*{A. eventGenerator.py}]{../appendix/eventGenerator.py.pdf}
%\includepdf[pages=-,scale=.8,pagecommand=\section*{B. sendEvents.py}]{../appendix/sendEvents.py.pdf}


% \RedeclareSectionCommand[beforeskip=\kapitelabstand         ]{chapter}


%%%%%%%%%%%%%%%%%%%%%%%%%%%%%%%%%%%%%%%%%%

%%%%%%%%%%%%%%%%%%%%%%%%%%%%%%%%%%%%%%%%%%
%%%%%%%%%%%%%%%%%%%%%%%%%%%%%%%%%%%%%%%%%%
\chapter{Fundamentals}

\section{Description of the Power System Simulation process}
\label{app:power-system-modeling}

In this appendix section, the general process of power system simulation is described. As this thesis is aiming to understand voltage stability and processes in longer periods of time, these explanations apply to pointer-based simulations, called RMS simulations. Meaning that the considered effects are slower electromechanical nature instead of faster electromagnetic ones. The in this thesis used Python framework \glqq \textit{diffpssi}\grqq~is based on this type of simulation, and due to its open-source based nature traceable.

\begin{figure}[htbp]
    \centering
    % \includegraphics[width=\textwidth]{fundamentals/power-system-simulation-process.pdf}
    \missingfigure{Power system simulation process}
    \caption{Power system simulation process; own illustration}
    \label{fig:power-system-simulation-process}
\end{figure}

\commenting{
    Really basic: (?)
    \begin{itemize}
        \item RMS vs EMT simulation (-> meaning one cannot simulate other faults than 3ph w/o ground)
        \item Phasor description
        \item Basic formulation: Static (algebraic) and dynamic (differential) equations
        \item Using of solvers (Integrators) for time domain simulation
        \item Using of different optimizatinon algorithms for steady state (load flow) simulation -> initial values
    \end{itemize}
    Less basic and more advanced:
    \begin{itemize}
        \item rountines in the framework
        \item two types: Algebraic and Differential equations have to be solved at each time step -> What is which? Which operational equipment is typically described with which type of equation?
        \item per unit system applying for easier simulation (different voltage levels)
        \item ...
    \end{itemize}
}

%%%%%%%%%%%%%%%%%%%%%%%%%%%%%%%%%%%%%%%%%%
\section{Jacobian based voltage stability criterions}
\label{app:jacobian-voltage-indices}

\textcite{danishVoltageStabilityElectric2015} is showing, describing, and referencing some voltage stability indices based on the Jacobian matrix. The following table is a collection of these indices.

\begin{sidewaystable}[h!]
    \centering
    \small
    \caption{Jacobian based voltage stability criterions; after \textcite{danishVoltageStabilityElectric2015}}
    \vspace*{12pt}
    \renewcommand{\arraystretch}{2}
    \begin{tabularx}{20cm}{llXXl}
        % \toprule
        \textbf{Index} & \textbf{Abbreviation} & \textbf{Calculation} & \textbf{Stability Threshold} & \textbf{Reference} \\
        \toprule
        Tangent Vector Index & \acs{TVI} & $\mathrm{TVI}_i=\left\lvert \frac{\dd{V_i}}{\dd{\lambda}}\right\rvert^{-1}$ & depending on load increase & \\ \midrule
        Test Function & & $t_{cc}=\left\lvert e^T_c \cdot \mab{J} \times \mab{J}_{cc}^{-1} \cdot e_c\right\rvert$ & details are given in reference & \\ \midrule
        & $i$ & $i=\frac{1}{i_0} \cdot \sigma_\mathrm{max} \cdot \big( \dv{\sigma_\mathrm{max}}{\lambda_\mathrm{total}} \big)^{-1}$ & $i > 0$ & \\ \midrule
        Minimum Eigenvalue & & $\Delta V=\sum_{i} \frac{\xi_i\eta_i}{\lambda_i} \Delta Q$ & all eigenvalues should be positive & \\ \midrule
        Minimum Singular Value & & $\begin{bmatrix} \Delta \vartheta \\ \Delta V \end{bmatrix}=\mab{V} \sum^-1 \mab{U}^T \begin{bmatrix} \Delta F \\ \Delta G \end{bmatrix}$ & details are given in reference & \\ \midrule
        Predicting Voltage Collapse & & $\frac{V}{V_0}$ & the smallest index value & \\ \midrule
        Impedance Ratio & & $\frac{Z_ii}{Z_i}$ & $\frac{Z_ii}{Z_i} \leq 1$ & \\
        \bottomrule
    \end{tabularx}
\end{sidewaystable}



%%%%%%%%%%%%%%%%%%%%%%%%%%%%%%%%%%%%%%%%%%
\section{Comparison of System based and Jacobian based indices}
\label{app:jacobian-vs-system-indices}

% %%%%%%%%%%%%%%%%%%%%%%%%%%%%%%%%%%%%%%%%%%
% %%%%%%%%%%%%%%%%%%%%%%%%%%%%%%%%%%%%%%%%%%
% \chapter{Python modeling}

% %%%%%%%%%%%%%%%%%%%%%%%%%%%%%%%%%%%%%%%%%%
% \section{Mathematical equations}
% \label{app:python-mathematical}

% %%%%%%%%%%%%%%%%%%%%%%%%%%%%%%%%%%%%%%%%%%
% \section{OLTC control}


% %%%%%%%%%%%%%%%%%%%%%%%%%%%%%%%%%%%%%%%%%%
% %%%%%%%%%%%%%%%%%%%%%%%%%%%%%%%%%%%%%%%%%%
% \chapter{Verification}

% %%%%%%%%%%%%%%%%%%%%%%%%%%%%%%%%%%%%%%%%%%
% \section{Single-machine infinite bus-bar model}
% \label{app:smib-model}


% %%%%%%%%%%%%%%%%%%%%%%%%%%%%%%%%%%%%%%%%%%
% %%%%%%%%%%%%%%%%%%%%%%%%%%%%%%%%%%%%%%%%%%
% \chapter{Case study}