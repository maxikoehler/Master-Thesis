%!TEX root = ../main.tex

\cleardoublepage

\renewcommand{\abstractname}{Abstract} % Text für Überschrift
% \phantomsection\addcontentsline{toc}{chapter}{\abstractname}
\begin{abstract}
    In this thesis, a new control concept for tap changer transformers is investigated, which takes into account an increase in dynamic capabilities by shortening the switching times. 
    This reduction is realized by a so-called \acf{FSM}, which is based on a conventional tap changer and is available in addition to it.
    A transformer model with variable transformation ratio and the associated control scheme for a tap changer are modeled in an existing Python package for electrical power grid simulation. 
    The implemented extensions are evaluated and compared with results of the commercial grid simulation software DIgSILENT PowerFactory.
    In addition, tools for the evaluation of simulation scenarios are implemented. 
    The toolset includes the calculation of so-called Nose Curves and a \acf{TVI}.
    The effects of the various controls on voltage stability are tested as part of an application study. 
    In addition, improvements to the controls are derived and outlined. 
    The implemented models are validated using commercial software. 
    The results show a stabilization of the bus voltages through the application of the \acs{FSM}. 
    The study can show a feedback of the fast tap-changing to the power and speed fluctuations. 
    In addition, a voltage band can be identified in which one of the controllers shows no reaction. 
    An implementation of the discussed improvements and the modeling of other equipment are not part of the work. 
    An application to phase-shifting transformers or other types of stability is neglected.

    % This thesis investigates an advanced control scheme for tap-changing transformers, considering increased dynamic capabilities through reduced switching times. 
    % This reduction is realized with a \acf{FSM} added to the conventional tap changer. 
    % A transformer with a variable ratio and the respective control scheme of the tap changers are modeled in an existing power system simulation framework. 
    % The implemented extensions are evaluated and compared to the commercial power system simulation software DIgSILENT PowerFactory.
    % Additional tools for the evaluation of simulation scenarios are implemented. 
    % The toolset contains the calculation of Nose Curves and a \acf{TVI}. 
    % Within an application study, the impacts of the different control schemes are tested, focussing on voltage stability. 
    % Additionally, improvements on the controls are derived and sketched. 
    % The implemented models are valid against the commercial software. 
    % The results show a possible stabilization of the bus voltages due to applying the \acs{FSM} control scheme. 
    % The study can show backpropagation of the fast tap changing on the power and speed oscillations. 
    % A zone with no reaction from one of the controllers is identified. 
    % Discussed improvements and the modeling of other operational units are not part of the thesis. 
    % The application of phase-shifting transformers or other stability aspects is also neglected.
\end{abstract}

%%%%%%%%%%%%%%%%%%%%%%%%%%%%%%%%%%%%%%%%%%%%%%%%%%%%%%%%%%%%%%%%%%%%%%%%%%%%%%%%%%%%%%%%%%%%
\cleardoublepage
%%%%%%%%%%%%%%%%%%%%%%%%%%%%%%%%%%%%%%%%%%%%%%%%%%%%%%%%%%%%%%%%%%%%%%%%%%%%%%%%%%%%%%%%%%%%

\begin{otherlanguage}{german}
\renewcommand{\abstractname}{Kurzfassung}
% \phantomsection\addcontentsline{toc}{chapter}{\abstractname}
\begin{abstract}
    In dieser Arbeit wird ein neuartiges Regelungskonzept für Stufenschaltertransformatoren untersucht, welches eine Erhöhung der dynamischen Fähigkeiten durch eine Verkürzung der Schaltzeiten berücksichtigt. 
    Diese Verkürzung wird durch ein sogenanntes \acf{FSM} realisiert, das auf einem konventionellen Stufenschalter basiert und zusätzlich zu diesem vorhanden ist. 
    Ein Transformatormodell mit variablem Übersetzungsverhältnis und das zugehörige Regelungsschema für einen Stufenschalter werden in einem bestehenden Python Paket zur elektrischen Energienetzsimulation modelliert. 
    Die implementierten Erweiterungen werden evaluiert und mit Ergebnissen der kommerziellen Netzsimulationssoftware DIgSILENT PowerFactory verglichen. 
    Zusätzlich werden Werkzeuge für die Auswertung von Simulationsszenarien implementiert. 
    Das Toolset beinhaltet die Berechnung von sogenannten Nasenkurven und eines \acf{TVI}. 
    Im Rahmen einer Anwendungsstudie werden die Auswirkungen der verschiedenen Regelungen auf die Spannungsstabilität getestet. 
    Zusätzlich werden Verbesserungen an den Regelungen abgeleitet und skizziert. 
    Die implementierten Modelle werden mit Hilfe der kommerziellen Software validiert. 
    Die Ergebnisse zeigen eine Stabilisierung der Busspannungen durch die Applikation des \acsp{FSM}. 
    Die Studie kann eine Rückkopplung der schnellen Stufenschaltung auf die Leistungs- und Drehzahlschwankungen zeigen. 
    Außerdem kann ein Spannungsband identifiziert werden, in der einer der Regler keine Reaktion zeigt. 
    Eine Umsetzung der diskutierten Verbesserungen und die Modellierung anderer Betriebsmittel sind nicht Teil der Arbeit. 
    Eine Anwendung auf Phasenschiebertransformatoren oder andere Arten von Stabilität werden vernachlässigt.
\end{abstract}
\end{otherlanguage}