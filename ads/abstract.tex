%!TEX root = ../main.tex

\cleardoublepage

\renewcommand{\abstractname}{Abstract} % Text für Überschrift
% \phantomsection\addcontentsline{toc}{chapter}{\abstractname}
\begin{abstract}
    In this thesis, a new control concept for transformer tap changers is investigated, which takes into account an increase in dynamic capabilities by shortening the switching times. 
    This reduction is realized by a \acf{FSM}, which is based on power electronics and extending the classical \acf{OLTC}, resulting in a hybrid solution.
    A transformer model with variable transformation ratio and the associated control scheme for a tap changer are modeled in an existing Python package for electrical power grid simulation. 
    The implemented extensions are evaluated and compared with results of the commercial grid simulation software DIgSILENT PowerFactory.
    In addition, tools for the evaluation of simulation scenarios are implemented. 
    The toolset includes the calculation of P-V Curves and a \acf{TVI}.
    The effects of the various controls on voltage stability are tested as part of an application study. 
    In addition, improvements to the controls are derived and outlined. 
    The implemented models are validated using commercial software. 
    The results show a stabilization of the bus voltages through the application of the \acs{FSM}. 
    The study illustrates a feedback of the fast tap-changing to the power and speed fluctuations of a synchronous machine. 
    In addition, a voltage band can be identified in which one of the controllers shows no reaction. 
    % An implementation of the discussed improvements and the modeling of other equipment are not part of this work. 
    % An application to phase-shifting transformers or other types of stability is neglected.
    As future outlook, the discussed control improvements can be realized. Futher, an investigation a \acs{FSM} applied to a phase shifting transformer seems useful.
\end{abstract}

%%%%%%%%%%%%%%%%%%%%%%%%%%%%%%%%%%%%%%%%%%%%%%%%%%%%%%%%%%%%%%%%%%%%%%%%%%%%%%%%%%%%%%%%%%%%
\cleardoublepage
%%%%%%%%%%%%%%%%%%%%%%%%%%%%%%%%%%%%%%%%%%%%%%%%%%%%%%%%%%%%%%%%%%%%%%%%%%%%%%%%%%%%%%%%%%%%

\begin{otherlanguage}{german}
\renewcommand{\abstractname}{Kurzfassung}
% \phantomsection\addcontentsline{toc}{chapter}{\abstractname}
\begin{abstract}
    In dieser Arbeit wird ein neuartiges Regelungskonzept für Stufenschalter für Leistungstransformatoren untersucht, welches eine Erhöhung der dynamischen Fähigkeiten durch eine Verkürzung der Schaltzeiten berücksichtigt. 
    Diese Verkürzung wird durch ein \acf{FSM} realisiert, das auf einem leistungselektronischen Komponenten basiert, den klassischen \acf{OLTC} erweitert und damit zu einer hybriden Lösung macht. 
    Ein Transformatormodell mit variablem Übersetzungsverhältnis und das zugehörige Regelungsschema für einen Stufenschalter werden in einem bestehenden Python Paket zur elektrischen Energienetzsimulation modelliert. 
    Die implementierten Erweiterungen werden evaluiert und mit Ergebnissen der kommerziellen Netzsimulationssoftware DIgSILENT PowerFactory verglichen. 
    Zusätzlich werden Werkzeuge für die Auswertung von Simulationsszenarien implementiert. 
    Das Toolset beinhaltet die Berechnung von P-V Kurven und eines \acf{TVI}. 
    Im Rahmen einer Anwendungsstudie werden die Auswirkungen der verschiedenen Regelungen auf die Spannungsstabilität getestet. 
    Zusätzlich werden Verbesserungen an den Regelungen abgeleitet und skizziert. 
    Die implementierten Modelle werden mit Hilfe der kommerziellen Software validiert. 
    Die Ergebnisse zeigen eine Stabilisierung der Busspannungen durch die Applikation des \acsp{FSM}. 
    Die Studie kann eine Rückkopplung der schnellen Stufenschaltung auf die Leistungs- und Drehzahlschwankungen einer Synchronmaschine zeigen. 
    Außerdem kann ein Spannungsband identifiziert werden, in der einer der Regler keine Reaktion zeigt. 
    % Eine Umsetzung der diskutierten Verbesserungen und die Modellierung anderer Betriebsmittel sind nicht Teil der Arbeit. 
    % Eine Anwendung auf Phasenschiebertransformatoren oder andere Arten von Stabilität werden vernachlässigt.
    Als Ausblick kann die Umsetzung der diskutierten Änderungen und Verbesserungen auf das Regelschema angemerkt werden.
    Des Weitern eröffnet eine Untersuchung der Applikation auf einen Phasenschiebertransformator weitere Möglichkeiten.
\end{abstract}
\end{otherlanguage}