%!TEX root = ../main.tex

\cleardoublepage

\renewcommand{\abstractname}{Abstract} % Text für Überschrift
\begin{abstract}
    The goal of this \arbeit~is the determination of the \acf{CCT} of a \acf{SG} in a simplified \acf{SMIB} model. For this, a simple three-phase fault scenario is applied, and the generator swing equation is solved in the time domain with a Python-integrated solver. A function is implemented to calculate the \acs{CCT} numerically, the result is compared to the analytical solution. Two additional fault scenarios are constructed, simulated and evaluated. These three cases illustrate the transient stability of a generator against an \acf{IBB}, especially in a visual context with selected plots. The numerical algorithm shows satisfying results compared to the analytical. Limitations rely on the complexity of the considered electrical network, also in the additional fault scenarios, and the missing possible machine interaction. Further, the damping in the system is neglected completely, as well as any control unit of a \acs{SG} (concrete the automatic voltage regulator (AVR), power system stabilizer (PSS) and governor (GOV)).
\end{abstract}

%%%%%%%%%%%%%%%%%%%%%%%%%%%%%%%%%%%%%%%%%%%%%%%%%%%%%%%%%%%%%%%%%%%%%%%%%%%%%%%%%%%%%%%%%%%%
\cleardoublepage
%%%%%%%%%%%%%%%%%%%%%%%%%%%%%%%%%%%%%%%%%%%%%%%%%%%%%%%%%%%%%%%%%%%%%%%%%%%%%%%%%%%%%%%%%%%%

\begin{otherlanguage}{german}
\renewcommand{\abstractname}{Kurzfassung}
\begin{abstract}
    Das Ziel dieser Seminararbeit ist die Bestimmung der kritischen Fehlerklärungszeit am Beispiel eines Synchrongenerators in einem vereinfachten Einmaschinenmodell. Zu diesem Zweck wird ein einfacher dreiphasiger Kurzschluss angewandt, und die dynamische Bewegungsgleichung des Generators wird im Zeitbereich mit einem in Python integrierten Algorithmus gelöst. Eine Funktion wird implementiert, um die kritische Fehlerklärungszeit numerisch zu berechnen. Das Ergebnis wird zusätzlich mit der analytischen Lösung verglichen. Zwei weitere Fehlerszenarien werden erstellt, simuliert und
    ausgewertet. Diese drei Fälle veranschaulichen die transiente Stabilität eines Generators gegenüber einer starren und idealen Spannungsquelle mit Innenimpedanz, insbesondere in einem visuellen Kontext mit ausgewählten Diagrammen. Der numerische Algorithmus zeigt im Vergleich zum analytischen zufriedenstellende Ergebnisse. Limitationen zeigen sich in der Komplexität des betrachteten elektrischen Netzes, vor allem auch im Unterbrechungsszenario, und die fehlende mögliche Interaktion von multiplen Maschinen im Netz. Des weiteren wird die Dämpfung im System vollständig vernachlässigt, wie auch sämtliche Regler der Maschine (konkret die automatische Spannungsregelung (AVR), die Pendeldämpfung (PSS) und der Governor (GOV)).
\end{abstract}
\end{otherlanguage}