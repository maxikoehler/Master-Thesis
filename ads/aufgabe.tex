%!TEX root = ../main.tex

\chapter*{Assignment of the paper}
%\vspace*{4cm}
\thispagestyle{plain.scrheadings}
{\large \textbf{Topic:} \parbox[t]{0.8\textwidth}{\titel}}
\newline

The \ac{CCT} is an essential parameter in power system stability
analysis. For example, in the case of \ac{SG}, the \acs{CCT} determines the
maximum fault-clearing time a generator can withstand without losing synchronism.
This seminar will introduce the concept of \acs{CCT} computing. We will discuss the factors
influencing \acs{CCT}, such as generator characteristics, system parameters, and fault type,
and explore the methods used to calculate \acs{CCT} in practical power system analysis.

% \begin{itemize}
%     \item Swing equation of synchronous generators
%     \item Lösen der Swing equation mithilfe von Python -> Lösung einer DGL zweiter Ordnung
%     \item Equal-area criterion -> Herleitung der Gleichungen
%     \item Simulation eines Fehlers -> Anwendung des equal-area criterions mithilfe von python.
%     \item Vergleich zwischen analytischer und simulativer Ergebnisse
% \end{itemize}

The seminar research paper should contain:
\begin{itemize}
    \item A literature research of governing equations describing the short-term dynamic behavior of \acs{SG}, relevant fault types and their influence;
    \item an investigation of the influences from machine characteristics and system parameters on the \acs{CCT};
    \item a computed simulation model for numerical determination of the \acs{CCT} with the \ac{EAC};
    \item simulations of system faults and comparisons to analytical solutions.
\end{itemize} 