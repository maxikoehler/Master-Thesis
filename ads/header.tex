%!TEX root = ../main.tex

%
% Nahezu alle Einstellungen koennen hier getaetigt werden
%

\RequirePackage[l2tabu, orthodox]{nag}	% weist in Commandozeile bzw. log auf veraltete LaTeX Syntax hin

\documentclass[%
    % final,
	pdftex,
	twoside,			% Einseitiger Druck (oneside) oder zweiseitig (twoside)
	headings=openright,	% Kapitelanfänge immer auf rechter Seite (bei zweiseitig)
	cleardoublepage=empty,	% Leere Vakatseiten
	11pt,				% Schriftgroesse
	parskip=half,		% Halbe Zeile Abstand zwischen Absätzen (half).
%	topmargin = 10pt,	% Abstand Seitenrand (Std:1in) zu Kopfzeile [laut log: unused]
	headheight = 20pt,	% Höhe der Kopfzeile
%	headsep = 30pt,	% Abstand zwischen Kopfzeile und Text Body  [laut log: unused]
	headsepline,		% Linie nach Kopfzeile.
	% footsepline,		% Linie vor Fusszeile.
	footheight = 10pt,	% Höhe der Fusszeile
	abstracton,		% Abstract Überschriften
	DIV=calc,		% Satzspiegel berechnen
	headinclude=false,	% Kopfzeile nicht in den Satzspiegel einbeziehen
	footinclude=false,	% Fußzeile nicht in den Satzspiegel einbeziehen
	listof=totoc,		% Abbildungs-/ Tabellenverzeichnis im Inhaltsverzeichnis darstellen
	toc=bibliography,	% Literaturverzeichnis im Inhaltsverzeichnis darstellen
	pointlessnumbers,
	fleqn,
	chapterprefix=false,
  appendixprefix=false,
	% bibliography=openstyle
]{scrreprt}	% Koma-Script report-Klasse (scrreprt), fuer laengere Bachelorarbeiten alternativ auch: scrbook

\raggedbottom

% Einstellungen laden
\usepackage{xstring}
\usepackage[utf8]{inputenc}
\usepackage[T1]{fontenc}
\usepackage{ragged2e}

\newcommand{\einstellung}[1]{%
  \expandafter\newcommand\csname #1\endcsname{}
  \expandafter\newcommand\csname setze#1\endcsname[1]{\expandafter\renewcommand\csname#1\endcsname{##1}}
}
\newcommand{\langstr}[1]{\einstellung{lang#1}}

\input{ads/einstellungen_liste.tex} % verfügbare Einstellungen
\input{einstellungen} % lese Einstellungen

\input{lang/strings} % verfügbare Strings
\input{lang/\sprache} % Übersetzung einlesen

% Einstellung der Sprache des Paketes Babel und der Verzeichnisüberschriften
\iflang{de}{\usepackage[english, ngerman]{babel}}
\iflang{en}{\usepackage[ngerman, english]{babel}}

%%%%%%% Package Includes %%%%%%%

\usepackage{lipsum}
\usepackage{bm}
% \usepackage[left=3cm,right=2cm,top=2.5cm,bottom=2.5cm,foot=.5cm]{geometry}	% Seitenränder und Abstände
\usepackage[left=2.5cm,right=4.5cm,top=2.5cm,bottom=2.5cm,foot=.5cm,marginparwidth=3.2cm,marginparsep=.6cm]{geometry}	% Seitenränder und Abstände
\usepackage[activate]{microtype} %Zeilenumbruch und mehr
\usepackage[onehalfspacing]{setspace}
\usepackage{makeidx}
\usepackage[autostyle=true,german=quotes]{csquotes}
\usepackage{longtable}
\usepackage{enumitem}	% mehr Optionen bei Aufzählungen
\usepackage{graphicx}
\usepackage{pdfpages}   % zum Einbinden von PDFs
% \usepackage[table]{xcolor} 	% für HTML-Notation
\usepackage{float}
\usepackage{array}
\usepackage{calc}		% zum Rechnen (Bildtabelle in Deckblatt)
\usepackage[right]{eurosym}
\DeclareUnicodeCharacter{20AC}{\euro}
\usepackage{wrapfig}
\usepackage{pgffor} % für automatische Kapiteldateieinbindung
\usepackage[hang, multiple, stable]{footmisc} % Fussnoten; perpage für jede Seite
\usepackage{chngcntr}
\counterwithout{footnote}{chapter}
\usepackage[printonlyused]{acronym} % falls gewünscht kann die Option footnote eingefügt werden, dann wird die Erklärung nicht inline sondern in einer Fußnote dargestellt
\usepackage{listings}
\usepackage{tabularx}
\usepackage{booktabs}
% \usepackage{pdflscape}
\usepackage{lscape}
\usepackage{rotating}
\usepackage[labelfont={bf},font={small},format={plain},indention=.5cm,singlelinecheck={false}]{caption} % Einstellungen für Bildunterschriften/Tabellenüberschriften
\setcapwidth[c]{.9\textwidth}
\usepackage{subcaption} 
\usepackage{ltxtable}
% \usepackage{filecontents}
\setlength{\skip\footins}{10pt plus 6pt minus 0pt} % Abstand zwischen Fußnoten und Fließtext erhöhen. Latex-Standard: 10pt plus 4pt minus 2pt
\usepackage{mathtools}
\usepackage{nicematrix}
\usepackage{physics}
\usepackage{tikz}
\usetikzlibrary{positioning}
% \usetikzlibrary{external}
\usepackage{pgf-pie} % https://www.namsu.de/Extra/pakete/Pie_Chart.html
\usepackage{pgfplots}
\pgfplotsset{compat=1.16}
\usepackage{multirow}
\usepackage[absolute]{textpos}
\usepackage{tcolorbox}
% \usepackage{minitoc}
\usepackage{afterpage}
\usepackage{marginnote}
\usepackage{snotez}
\setsidenotes{
  note-mark-format={\hspace*{-2pt}},
  text-mark-format={\hspace*{-4pt}},
  text-format+={\RaggedRight\color{ees_blue}\itshape},
  sidefloat-format={\RaggedRight\color{ees_blue}\itshape\footnotesize},
  perpage=true
}

\renewcommand*{\marginfont}{\color{ees_blue}\itshape\footnotesize}
% \renewcommand*{\sidecaption}{\color{ees_blue}\itshape\footnotesize\RaggedRight}
\renewcommand*{\raggedrightmarginnote}{}

% \usepackage[Glenn]{fncychap}
% Sonny, Lenny, Glenn, Conny, Rejne, Bjarne, Bjornstrup
\usepackage{circuitikzgit}
\usepackage{nolbreaks}
\usepackage{scrtime}
\usepackage{prelim2e}
\renewcommand{\PrelimText}{\textcolor{red}{\textbf{\footnotesize[\today\ at \thistime\ -- preliminary version \version]}}}

\newcommand\tocpageseparator{\mbox{\hspace*{12pt}}\,}
\newcommand\tocpagenumberbox[1]{\mbox{#1}}

\RedeclareSectionCommands[
  tocraggedpagenumber,
  toclinefill=\tocpageseparator,
  tocpagenumberbox=\tocpagenumberbox
]{part,chapter,section,subsection}
% \RedeclareSectionCommands[
%   toconstartentry=\rule,
% ]{chapter}

% Change blank space between Headings and Text
\RedeclareSectionCommands[
  afterskip=12pt
]{chapter}
\RedeclareSectionCommands[
  afterskip=6pt
]{section}
\RedeclareSectionCommands[
  beforeskip=6pt,
  afterskip=3pt,
]{subsection, subsubsection}

% Notizen. Einsatz mit \todo{Notiz} oder \todo[inline]{Notiz}. Documentation: https://tug.ctan.org/macros/latex/contrib/todonotes/todonotes.pdf
\usepackage[obeyFinal,backgroundcolor=yellow,linecolor=black, figwidth=.9\linewidth,figcolor=white,textwidth=3cm]{todonotes}
% \setlength{\marginparwidth}{2.5cm}
% \reversemarginpar
\newcounter{mycomment}
\newcommand{\mycomment}[2][]{%
% initials of the author (optional) + note in the margin
\refstepcounter{mycomment}%
{%
\setstretch{1}% spacing
\todo[color={red!100!green!33},size=\footnotesize]{%
\textbf{[\uppercase{#1}\themycomment]:}~#2}%
}}
% Alle Notizen ausblenden mit der Option "final" in \documentclass[...] oder durch das auskommentieren folgender Zeile
% \usepackage[disable]{todonotes}

% Kommentarumgebung. Einsatz mit \comment{}. Alle Kommentare ausblenden mit dem Auskommentieren der folgenden und dem aktivieren der nächsten Zeile.
\newcommand{\commenting}[1]{{\color{ees_red} #1}} % Kommentar anzeigen
\newcommand{\ai}[1]{{\color{ees_green} #1}} % Kommentar anzeigen
%\newcommand{\comment}[1]{} %Kommentar ausblenden


%%%%%% Configuration %%%%%

%% Anwenden der Einstellungen

\usepackage{\schriftart}
% Überschriften auch in gesetzter Schriftart
\setkomafont{disposition}{%
	\normalfont\bfseries
}
\setkomafont{dictum}{\normalfont}
% Verwendung der Schrift ohne Serifen
% \renewcommand*{\familydefault}{\sfdefault}
% \addtokomafont{disposition}{\sffamily}

\ladefarben{}

% Titel, Autor und Datum
\title{\titel}
\author{\autor}
\date{\datum}

% PDF Einstellungen
\usepackage[%
	pdftitle={\titel},
	pdfauthor={\autor},
	pdfsubject={\arbeit},
	pdfcreator={pdflatex, LaTeX with KOMA-Script},
	pdfpagemode=UseOutlines, 		% Beim Oeffnen Inhaltsverzeichnis anzeigen
	pdfdisplaydoctitle=true, 		% Dokumenttitel statt Dateiname anzeigen.
	pdflang={\sprache}, 			% Sprache des Dokuments.
	%hidelinks,						% entfernt Umrandung von verlinkten Stellen, ohne Verlinkung zu löschen
]{hyperref}

% (Farb-)einstellungen für die Links im PDF
\hypersetup{%
	colorlinks=true, 		% Aktivieren von farbigen Links im Dokument
	linkcolor=ees_blue, 	% Farbe festlegen
	citecolor=ees_blue,
	filecolor=ees_blue,
	menucolor=ees_blue,
	urlcolor=ees_blue,
	linktocpage=true, 		% Nicht der Text sondern die Seitenzahlen in Verzeichnissen klickbar
	bookmarksnumbered=true 	% Überschriftsnummerierung im PDF Inhalt anzeigen.
}
% Workaround um Fehler in Hyperref, muss hier stehen bleiben
\usepackage{bookmark} %nur ein latex-Durchlauf für die Aktualisierung von Verzeichnissen nötig

% Schriftart in Captions etwas kleiner
\addtokomafont{caption}{\small}

% Literaturverweise (sowohl deutsch als auch englisch)
\iflang{de}{%
\usepackage[
	backend=biber,		% empfohlen. Falls biber Probleme macht: bibtex
	bibwarn=true,
	bibencoding=utf8,	% wenn .bib in utf8, sonst ascii
	% sortlocale=de_DE,
	sorting=none,		% Altenativen: https://tex.stackexchange.com/questions/51434/biblatex-citation-order
	style=\zitierstil,
]{biblatex}
}
\iflang{en}{%
\usepackage[
	backend=biber,		% empfohlen. Falls biber Probleme macht: bibtex
	bibwarn=true,
	bibencoding=utf8,	% wenn .bib in utf8, sonst ascii
	% sortlocale=en_US,
	sorting=none,
	style=\zitierstil,
]{biblatex}
}

\setcounter{biburlnumpenalty}{100}
\setcounter{biburlucpenalty}{100}
\setcounter{biburllcpenalty}{100}

\ladeliteratur{}

% Glossar
% \usepackage[nonumberlist,toc,automake]{glossaries}
% \addtokomafont{descriptionlabel}{\normalfont\bfseries}

%Kopf- und Fußzeilen
\usepackage[plainfootsepline=yes]{scrlayer-scrpage}

%%%%%% Additional settings %%%%%%

% Hurenkinder und Schusterjungen verhindern
% http://projekte.dante.de/DanteFAQ/Silbentrennung
\clubpenalty = 10000 % schließt Schusterjungen aus (Seitenumbruch nach der ersten Zeile eines neuen Absatzes)
\widowpenalty = 10000 % schließt Hurenkinder aus (die letzte Zeile eines Absatzes steht auf einer neuen Seite)
\displaywidowpenalty=10000

% Bildpfad
\graphicspath{{images/}{../diffpssi-ma-kohler/}}

% Einige häufig verwendete Sprachen
\lstloadlanguages{PHP,Python,Java,C,C++,bash}
\listingsettings{}
% Umbennung des Listings
\renewcommand\lstlistingname{\langlistingname}
\renewcommand\lstlistlistingname{\langlistlistingname}
\def\lstlistingautorefname{\langlistingautorefname}

% Abstände in Tabellen
\setlength{\tabcolsep}{\spaltenabstand}
\renewcommand{\arraystretch}{\zeilenabstand}

%%%%%%%%%%%%%%%%%%%%%%%%%%%%%%%%%%%%%%%%%%%%
% Anhang mit separatem Inhaltsverzeichniss (alte Version)
%%%%%%%%%%%%%%%%%%%%%%%%%%%%%%%%%%%%%%%%%%%%
% \makeatletter% --> De-TeX-FAQ
% % Weitergabe des folgenden Codes oder Modifikationen davon nur unter Nennung
% % der Originalquelle: <http://www.komascript.de/comment/1073#comment-1073>,
% % gestattet.
% % Leistungsfähigere Lösung unter <https://komascript.de/comment/5578#comment-5578>.
% % 
% % Inhaltsverzeichnis für den Anhang erstellen 
% \newcommand*{\maintoc}{% Hauptinhaltsverzeichnis
%   \begingroup
%     \@fileswfalse% kein neues Verzeichnis öffnen
%     \renewcommand*{\appendixattoc}{% Trennanweisung im Inhaltsverzeichnis
%       \value{tocdepth}=-10000 % lokal tocdepth auf sehr kleinen Wert setzen
%     }%
%     \tableofcontents% Verzeichnis ausgeben
%   \endgroup
% }
% \newcommand*{\appendixtoc}{% Anhangsinhaltsverzeichnis
%   \begingroup
%     \edef\@alltocdepth{\the\value{tocdepth}}% tocdepth merken
%     \setcounter{tocdepth}{-10000}% Keine Verzeichniseinträge
%     \renewcommand*{\contentsname}{% Verzeichnisname ändern
%       \langanhang}%
%     \renewcommand*{\appendixattoc}{% Trennanweisung im Inhaltsverzeichnis
%       \setcounter{tocdepth}{\@alltocdepth}% tocdepth wiederherstellen
%     }%
%     \tableofcontents% Verzeichnis ausgeben
%     \setcounter{tocdepth}{\@alltocdepth}% tocdepth wiederherstellen
%   \endgroup
% }
% \newcommand*{\appendixattoc}{% Trennanweisung im Inhaltsverzeichnis
% }
% \g@addto@macro\appendix{% \appendix erweitern
%   \if@openright\cleardoublepage\else\clearpage\fi% Neue Seite
%   \phantomsection
%   \addcontentsline{toc}{chapter}{\appendixname}% Eintrag ins Hauptverzeichnis
%   \addtocontents{toc}{\protect\appendixattoc}% Trennanweisung in die toc-Datei
% }
% \makeatother


% Kreisdiagramme
\def\printonlylargeenough#1#2{\unless\ifdim#2pt<#1pt\relax
#2\printnumbertrue
\else
\printnumberfalse
\fi}
\newif\ifprintnumber

\clearpairofpagestyles
% \ohead[]{\headmark}       		% Kopfzeile außen immer mit Headmark versehen
\ohead[]{\pagemark}       		% Kopfzeile außen immer mit Headmark versehen
\ihead[]{\headmark}       		% Kopfzeile außen immer mit Headmark versehen
\automark[section]{chapter}		% Headmark bestehend aus Kolumnentitel
% \ofoot[\pagemark]{\pagemark}	% Fußzeile mit Seitenzahl außen
\renewcommand*\chapterpagestyle{plain.scrheadings}
% \renewcommand*\partpagestyle{plain.scrheadings}		%Bei Verwendung von Parts als Überschriftenebene: Setzen des Pagestyles global


%%%%%%%%%%%%%%%%%%%%%%%%%%%%%%%%%%%%%%%%%%%%
% Verzeichnisse gemeinsam auf einer Seite
%%%%%%%%%%%%%%%%%%%%%%%%%%%%%%%%%%%%%%%%%%%%
% \makeatletter
% \renewcommand\listoffigures{%
%         \@starttoc{lof}%
% }
% \makeatother
% \makeatletter
% \renewcommand\listoftables{%
%     \@starttoc{lot}%
% }
% \makeatother
% \makeatletter
% \renewcommand\lstlistoflistings{%
%         \@starttoc{lol}%
% }
% \makeatother

%%%%%%%%%%%%%%%%%%%%%%%%%%%%%%%%%%%%%%%%%%%%
% Anhang mit separaten Verzeichnissen (Inhalt, Figures, Tables, Listings)
%%%%%%%%%%%%%%%%%%%%%%%%%%%%%%%%%%%%%%%%%%%%
\DeclareNewTOC[%
  owner=\jobname, 
  listname={Appendix},
]{atoc}
% \DeclareNewTOC[%
%   listname={List of Appendix Figures},
% ]{alof}
% \DeclareNewTOC[%
%   listname={List of Appendix Tables},
% ]{alot}
 
% \def\listofalofentryname{\listoflofentryname}% gleicher Präfix für Abbildungen im Anhangs-LoF wie im LoF
% \def\listofalotentryname{\listoflotentryname}% gleicher Präfix für Tabellen im Anhangs-LoT wie im LoT

\makeatletter
\AfterTOCHead[atoc]{\let\if@dynlist\if@tocleft}% <- gleiches Verhalten (gratuated oder flat) wie toc 
\newcommand*{\useappendixtocs}{%
  \renewcommand*{\ext@toc}{atoc}%
  \scr@ifundefinedorrelax{hypersetup}{}{%
    \hypersetup{bookmarkstype=atoc}%
  }%
  \renewcommand*{\ext@figure}{alof}%
  \renewcommand*{\ext@table}{alot}%
}
\newcommand*{\usestandardtocs}{%
  \renewcommand*{\ext@toc}{toc}%
  \scr@ifundefinedorrelax{hypersetup}{}{%
    \hypersetup{bookmarkstype=toc}%
  }%
  \renewcommand*{\ext@figure}{lof}%
  \renewcommand*{\ext@table}{lot}%
}
\scr@ifundefinedorrelax{ext@toc}{%
  \newcommand*{\ext@toc}{toc}
  \renewcommand{\addtocentrydefault}[3]{%
    \expandafter\tocbasic@addxcontentsline\expandafter{\ext@toc}{#1}{#2}{#3}%
  }
}{}
\makeatother
 
\usepackage{xpatch}
\xapptocmd\appendix{%
%   \addpart{\appendixname}
  \useappendixtocs
  \listofatocs
%   \listofalofs
%   \listofalots
}{}{}

%%%%%%%%%%%%%%%%%%%%%%%%%%%%%%%%%%%%%%%%%%%%
% Quotes before chapter
%%%%%%%%%%%%%%%%%%%%%%%%%%%%%%%%%%%%%%%%%%%%
\definecolor{quotemark}{gray}{0.7}
\makeatletter
\newlength\origparskip

% \newcommand{\fquote}{}

\newcommand{\fquote}{%
  \@ifnextchar[{\fquote@i}{\fquote@i[]}%]
}

\def\fquote@i[#1]{%
  \@ifnextchar[{\fquote@ii{#1}}{\fquote@ii{#1}[]}%]
}%

\def\fquote@ii#1[#2]{%
  \def\pqm@tempa{#1}%
  \def\pqm@tempb{#2}%
  \noindent
  \list
    {}
    {\setlength{\leftmargin}{0.3\textwidth}%
     \setlength{\rightmargin}{0.1\textwidth}%
     \setlength{\origparskip}{\parskip}}%
    \item[]%
      \begin{picture}(0,0)%
        \put(-15,-8){\makebox(0,0){\scalebox{4}{%
          \textcolor{ees_blue}{\textquotedblright}}}}%
      \end{picture}%
      \begingroup
      \itshape
      \ignorespaces}%

\def\endfquote{%
  \endgroup
  \par
  \raggedleft
  \ifx\pqm@tempa\empty
  \else
    {\bfseries --- \pqm@tempa\par}%
    \setlength{\parskip}{\origparskip}%
    \ifx\pqm@tempb\empty
    \else
      (\pqm@tempb)%
    \fi
  \fi
  \par
  \endlist}
\makeatother

% Math - vectors/matrices: new command for easy typesetting
\newcommand{\mab}[1]{\mathrm{\textbf{#1}}}

% easy access for comments of missing sources
\newcommand{\quelle}[0]{\commenting{\textbf{[Quelle]}}~}