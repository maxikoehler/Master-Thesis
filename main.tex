%%**************************************************************
%% Vorlage fuer Bachelorarbeiten (o.ä.) der DHBW
%%
%% Autor: Tobias Dreher, Yves Fischer, Michael Gruben, Markus Barthel
%% Datum: 06.07.2011 - 22.08.2014
%%
%% Autor: Ferdinand König und Maximilian Köhler
%% Datum: 2020 - 2022
%%**************************************************************

%!TEX root = ../main.tex

%
% Nahezu alle Einstellungen koennen hier getaetigt werden
%

\RequirePackage[l2tabu, orthodox]{nag}	% weist in Commandozeile bzw. log auf veraltete LaTeX Syntax hin

\documentclass[%
    % final,
	pdftex,
	twoside,			% Einseitiger Druck (oneside) oder zweiseitig (twoside)
	headings=openright,	% Kapitelanfänge immer auf rechter Seite (bei zweiseitig)
	cleardoublepage=empty,	% Leere Vakatseiten
	11pt,				% Schriftgroesse
	parskip=half,		% Halbe Zeile Abstand zwischen Absätzen (half).
%	topmargin = 10pt,	% Abstand Seitenrand (Std:1in) zu Kopfzeile [laut log: unused]
	headheight = 20pt,	% Höhe der Kopfzeile
%	headsep = 30pt,	% Abstand zwischen Kopfzeile und Text Body  [laut log: unused]
	headsepline,		% Linie nach Kopfzeile.
	% footsepline,		% Linie vor Fusszeile.
	footheight = 10pt,	% Höhe der Fusszeile
	abstracton,		% Abstract Überschriften
	DIV=calc,		% Satzspiegel berechnen
	headinclude=false,	% Kopfzeile nicht in den Satzspiegel einbeziehen
	footinclude=false,	% Fußzeile nicht in den Satzspiegel einbeziehen
	listof=totoc,		% Abbildungs-/ Tabellenverzeichnis im Inhaltsverzeichnis darstellen
	toc=bibliography,	% Literaturverzeichnis im Inhaltsverzeichnis darstellen
	pointlessnumbers,
	fleqn,
	chapterprefix=false,
  appendixprefix=false,
	% bibliography=openstyle
]{scrreprt}	% Koma-Script report-Klasse (scrreprt), fuer laengere Bachelorarbeiten alternativ auch: scrbook

\raggedbottom

% Einstellungen laden
\usepackage{xstring}
\usepackage[utf8]{inputenc}
\usepackage[T1]{fontenc}
\usepackage{ragged2e}

\newcommand{\einstellung}[1]{%
  \expandafter\newcommand\csname #1\endcsname{}
  \expandafter\newcommand\csname setze#1\endcsname[1]{\expandafter\renewcommand\csname#1\endcsname{##1}}
}
\newcommand{\langstr}[1]{\einstellung{lang#1}}

\input{ads/einstellungen_liste.tex} % verfügbare Einstellungen
\input{einstellungen} % lese Einstellungen

\input{lang/strings} % verfügbare Strings
\input{lang/\sprache} % Übersetzung einlesen

% Einstellung der Sprache des Paketes Babel und der Verzeichnisüberschriften
\iflang{de}{\usepackage[english, ngerman]{babel}}
\iflang{en}{\usepackage[ngerman, english]{babel}}

%%%%%%% Package Includes %%%%%%%

\usepackage{lipsum}
\usepackage{bm}
% \usepackage[left=3cm,right=2cm,top=2.5cm,bottom=2.5cm,foot=.5cm]{geometry}	% Seitenränder und Abstände
\usepackage[left=2.5cm,right=4.5cm,top=2.5cm,bottom=2.5cm,foot=.5cm,marginparwidth=3.2cm,marginparsep=.6cm]{geometry}	% Seitenränder und Abstände
\usepackage[activate]{microtype} %Zeilenumbruch und mehr
\usepackage[onehalfspacing]{setspace}
\usepackage{makeidx}
\usepackage[autostyle=true,german=quotes]{csquotes}
\usepackage{longtable}
\usepackage{enumitem}	% mehr Optionen bei Aufzählungen
\usepackage{graphicx}
\usepackage{pdfpages}   % zum Einbinden von PDFs
% \usepackage[table]{xcolor} 	% für HTML-Notation
\usepackage{float}
\usepackage{array}
\usepackage{calc}		% zum Rechnen (Bildtabelle in Deckblatt)
\usepackage[right]{eurosym}
\DeclareUnicodeCharacter{20AC}{\euro}
\usepackage{wrapfig}
\usepackage{pgffor} % für automatische Kapiteldateieinbindung
\usepackage[hang, multiple, stable]{footmisc} % Fussnoten; perpage für jede Seite
\usepackage{chngcntr}
\counterwithout{footnote}{chapter}
\usepackage[printonlyused]{acronym} % falls gewünscht kann die Option footnote eingefügt werden, dann wird die Erklärung nicht inline sondern in einer Fußnote dargestellt
\usepackage{listings}
\usepackage{tabularx}
\usepackage{booktabs}
% \usepackage{pdflscape}
\usepackage{lscape}
\usepackage{rotating}
\usepackage[labelfont={bf},font={small},format={plain},indention=.5cm,singlelinecheck={false}]{caption} % Einstellungen für Bildunterschriften/Tabellenüberschriften
\setcapwidth[c]{.9\textwidth}
\usepackage{subcaption} 
\usepackage{ltxtable}
% \usepackage{filecontents}
\setlength{\skip\footins}{10pt plus 6pt minus 0pt} % Abstand zwischen Fußnoten und Fließtext erhöhen. Latex-Standard: 10pt plus 4pt minus 2pt
\usepackage{mathtools}
\usepackage{nicematrix}
\usepackage{physics}
\usepackage{tikz}
\usetikzlibrary{positioning}
% \usetikzlibrary{external}
\usepackage{pgf-pie} % https://www.namsu.de/Extra/pakete/Pie_Chart.html
\usepackage{pgfplots}
\pgfplotsset{compat=1.16}
\usepackage{multirow}
\usepackage[absolute]{textpos}
\usepackage{tcolorbox}
% \usepackage{minitoc}
\usepackage{afterpage}
\usepackage{marginnote}
\usepackage{snotez}
\setsidenotes{
  note-mark-format={\hspace*{-2pt}},
  text-mark-format={\hspace*{-4pt}},
  text-format+={\RaggedRight\color{ees_blue}\itshape},
  sidefloat-format={\RaggedRight\color{ees_blue}\itshape\footnotesize},
  perpage=true
}

\renewcommand*{\marginfont}{\color{ees_blue}\itshape\footnotesize}
% \renewcommand*{\sidecaption}{\color{ees_blue}\itshape\footnotesize\RaggedRight}
\renewcommand*{\raggedrightmarginnote}{}

% \usepackage[Glenn]{fncychap}
% Sonny, Lenny, Glenn, Conny, Rejne, Bjarne, Bjornstrup
\usepackage{circuitikzgit}
\usepackage{nolbreaks}
\usepackage{scrtime}
\usepackage{prelim2e}
\renewcommand{\PrelimText}{\textcolor{red}{\textbf{\footnotesize[\today\ at \thistime\ -- preliminary version \version]}}}

\newcommand\tocpageseparator{\mbox{\hspace*{12pt}}\,}
\newcommand\tocpagenumberbox[1]{\mbox{#1}}

\RedeclareSectionCommands[
  tocraggedpagenumber,
  toclinefill=\tocpageseparator,
  tocpagenumberbox=\tocpagenumberbox
]{part,chapter,section,subsection}
% \RedeclareSectionCommands[
%   toconstartentry=\rule,
% ]{chapter}

% Change blank space between Headings and Text
\RedeclareSectionCommands[
  afterskip=12pt
]{chapter}
\RedeclareSectionCommands[
  afterskip=6pt
]{section}
\RedeclareSectionCommands[
  beforeskip=6pt,
  afterskip=3pt,
]{subsection, subsubsection}

% Notizen. Einsatz mit \todo{Notiz} oder \todo[inline]{Notiz}. Documentation: https://tug.ctan.org/macros/latex/contrib/todonotes/todonotes.pdf
\usepackage[obeyFinal,backgroundcolor=yellow,linecolor=black, figwidth=.9\linewidth,figcolor=white,textwidth=3cm]{todonotes}
% \setlength{\marginparwidth}{2.5cm}
% \reversemarginpar
\newcounter{mycomment}
\newcommand{\mycomment}[2][]{%
% initials of the author (optional) + note in the margin
\refstepcounter{mycomment}%
{%
\setstretch{1}% spacing
\todo[color={red!100!green!33},size=\footnotesize]{%
\textbf{[\uppercase{#1}\themycomment]:}~#2}%
}}
% Alle Notizen ausblenden mit der Option "final" in \documentclass[...] oder durch das auskommentieren folgender Zeile
% \usepackage[disable]{todonotes}

% Kommentarumgebung. Einsatz mit \comment{}. Alle Kommentare ausblenden mit dem Auskommentieren der folgenden und dem aktivieren der nächsten Zeile.
\newcommand{\commenting}[1]{{\color{ees_red} #1}} % Kommentar anzeigen
\newcommand{\ai}[1]{{\color{ees_green} #1}} % Kommentar anzeigen
%\newcommand{\comment}[1]{} %Kommentar ausblenden


%%%%%% Configuration %%%%%

%% Anwenden der Einstellungen

\usepackage{\schriftart}
% Überschriften auch in gesetzter Schriftart
\setkomafont{disposition}{%
	\normalfont\bfseries
}
\setkomafont{dictum}{\normalfont}
% Verwendung der Schrift ohne Serifen
% \renewcommand*{\familydefault}{\sfdefault}
% \addtokomafont{disposition}{\sffamily}

\ladefarben{}

% Titel, Autor und Datum
\title{\titel}
\author{\autor}
\date{\datum}

% PDF Einstellungen
\usepackage[%
	pdftitle={\titel},
	pdfauthor={\autor},
	pdfsubject={\arbeit},
	pdfcreator={pdflatex, LaTeX with KOMA-Script},
	pdfpagemode=UseOutlines, 		% Beim Oeffnen Inhaltsverzeichnis anzeigen
	pdfdisplaydoctitle=true, 		% Dokumenttitel statt Dateiname anzeigen.
	pdflang={\sprache}, 			% Sprache des Dokuments.
	%hidelinks,						% entfernt Umrandung von verlinkten Stellen, ohne Verlinkung zu löschen
]{hyperref}

% (Farb-)einstellungen für die Links im PDF
\hypersetup{%
	colorlinks=true, 		% Aktivieren von farbigen Links im Dokument
	linkcolor=ees_blue, 	% Farbe festlegen
	citecolor=ees_blue,
	filecolor=ees_blue,
	menucolor=ees_blue,
	urlcolor=ees_blue,
	linktocpage=true, 		% Nicht der Text sondern die Seitenzahlen in Verzeichnissen klickbar
	bookmarksnumbered=true 	% Überschriftsnummerierung im PDF Inhalt anzeigen.
}
% Workaround um Fehler in Hyperref, muss hier stehen bleiben
\usepackage{bookmark} %nur ein latex-Durchlauf für die Aktualisierung von Verzeichnissen nötig

% Schriftart in Captions etwas kleiner
\addtokomafont{caption}{\small}

% Literaturverweise (sowohl deutsch als auch englisch)
\iflang{de}{%
\usepackage[
	backend=biber,		% empfohlen. Falls biber Probleme macht: bibtex
	bibwarn=true,
	bibencoding=utf8,	% wenn .bib in utf8, sonst ascii
	% sortlocale=de_DE,
	sorting=none,		% Altenativen: https://tex.stackexchange.com/questions/51434/biblatex-citation-order
	style=\zitierstil,
]{biblatex}
}
\iflang{en}{%
\usepackage[
	backend=biber,		% empfohlen. Falls biber Probleme macht: bibtex
	bibwarn=true,
	bibencoding=utf8,	% wenn .bib in utf8, sonst ascii
	% sortlocale=en_US,
	sorting=none,
	style=\zitierstil,
]{biblatex}
}

\setcounter{biburlnumpenalty}{100}
\setcounter{biburlucpenalty}{100}
\setcounter{biburllcpenalty}{100}

\ladeliteratur{}

% Glossar
% \usepackage[nonumberlist,toc,automake]{glossaries}
% \addtokomafont{descriptionlabel}{\normalfont\bfseries}

%Kopf- und Fußzeilen
\usepackage[plainfootsepline=yes]{scrlayer-scrpage}

%%%%%% Additional settings %%%%%%

% Hurenkinder und Schusterjungen verhindern
% http://projekte.dante.de/DanteFAQ/Silbentrennung
\clubpenalty = 10000 % schließt Schusterjungen aus (Seitenumbruch nach der ersten Zeile eines neuen Absatzes)
\widowpenalty = 10000 % schließt Hurenkinder aus (die letzte Zeile eines Absatzes steht auf einer neuen Seite)
\displaywidowpenalty=10000

% Bildpfad
\graphicspath{{images/}{../diffpssi-ma-kohler/}}

% Einige häufig verwendete Sprachen
\lstloadlanguages{PHP,Python,Java,C,C++,bash}
\listingsettings{}
% Umbennung des Listings
\renewcommand\lstlistingname{\langlistingname}
\renewcommand\lstlistlistingname{\langlistlistingname}
\def\lstlistingautorefname{\langlistingautorefname}

% Abstände in Tabellen
\setlength{\tabcolsep}{\spaltenabstand}
\renewcommand{\arraystretch}{\zeilenabstand}

%%%%%%%%%%%%%%%%%%%%%%%%%%%%%%%%%%%%%%%%%%%%
% Anhang mit separatem Inhaltsverzeichniss (alte Version)
%%%%%%%%%%%%%%%%%%%%%%%%%%%%%%%%%%%%%%%%%%%%
% \makeatletter% --> De-TeX-FAQ
% % Weitergabe des folgenden Codes oder Modifikationen davon nur unter Nennung
% % der Originalquelle: <http://www.komascript.de/comment/1073#comment-1073>,
% % gestattet.
% % Leistungsfähigere Lösung unter <https://komascript.de/comment/5578#comment-5578>.
% % 
% % Inhaltsverzeichnis für den Anhang erstellen 
% \newcommand*{\maintoc}{% Hauptinhaltsverzeichnis
%   \begingroup
%     \@fileswfalse% kein neues Verzeichnis öffnen
%     \renewcommand*{\appendixattoc}{% Trennanweisung im Inhaltsverzeichnis
%       \value{tocdepth}=-10000 % lokal tocdepth auf sehr kleinen Wert setzen
%     }%
%     \tableofcontents% Verzeichnis ausgeben
%   \endgroup
% }
% \newcommand*{\appendixtoc}{% Anhangsinhaltsverzeichnis
%   \begingroup
%     \edef\@alltocdepth{\the\value{tocdepth}}% tocdepth merken
%     \setcounter{tocdepth}{-10000}% Keine Verzeichniseinträge
%     \renewcommand*{\contentsname}{% Verzeichnisname ändern
%       \langanhang}%
%     \renewcommand*{\appendixattoc}{% Trennanweisung im Inhaltsverzeichnis
%       \setcounter{tocdepth}{\@alltocdepth}% tocdepth wiederherstellen
%     }%
%     \tableofcontents% Verzeichnis ausgeben
%     \setcounter{tocdepth}{\@alltocdepth}% tocdepth wiederherstellen
%   \endgroup
% }
% \newcommand*{\appendixattoc}{% Trennanweisung im Inhaltsverzeichnis
% }
% \g@addto@macro\appendix{% \appendix erweitern
%   \if@openright\cleardoublepage\else\clearpage\fi% Neue Seite
%   \phantomsection
%   \addcontentsline{toc}{chapter}{\appendixname}% Eintrag ins Hauptverzeichnis
%   \addtocontents{toc}{\protect\appendixattoc}% Trennanweisung in die toc-Datei
% }
% \makeatother


% Kreisdiagramme
\def\printonlylargeenough#1#2{\unless\ifdim#2pt<#1pt\relax
#2\printnumbertrue
\else
\printnumberfalse
\fi}
\newif\ifprintnumber

\clearpairofpagestyles
% \ohead[]{\headmark}       		% Kopfzeile außen immer mit Headmark versehen
\ohead[]{\pagemark}       		% Kopfzeile außen immer mit Headmark versehen
\ihead[]{\headmark}       		% Kopfzeile außen immer mit Headmark versehen
\automark[section]{chapter}		% Headmark bestehend aus Kolumnentitel
% \ofoot[\pagemark]{\pagemark}	% Fußzeile mit Seitenzahl außen
\renewcommand*\chapterpagestyle{plain.scrheadings}
% \renewcommand*\partpagestyle{plain.scrheadings}		%Bei Verwendung von Parts als Überschriftenebene: Setzen des Pagestyles global


%%%%%%%%%%%%%%%%%%%%%%%%%%%%%%%%%%%%%%%%%%%%
% Verzeichnisse gemeinsam auf einer Seite
%%%%%%%%%%%%%%%%%%%%%%%%%%%%%%%%%%%%%%%%%%%%
% \makeatletter
% \renewcommand\listoffigures{%
%         \@starttoc{lof}%
% }
% \makeatother
% \makeatletter
% \renewcommand\listoftables{%
%     \@starttoc{lot}%
% }
% \makeatother
% \makeatletter
% \renewcommand\lstlistoflistings{%
%         \@starttoc{lol}%
% }
% \makeatother

%%%%%%%%%%%%%%%%%%%%%%%%%%%%%%%%%%%%%%%%%%%%
% Anhang mit separaten Verzeichnissen (Inhalt, Figures, Tables, Listings)
%%%%%%%%%%%%%%%%%%%%%%%%%%%%%%%%%%%%%%%%%%%%
\DeclareNewTOC[%
  owner=\jobname, 
  listname={Appendix},
]{atoc}
% \DeclareNewTOC[%
%   listname={List of Appendix Figures},
% ]{alof}
% \DeclareNewTOC[%
%   listname={List of Appendix Tables},
% ]{alot}
 
% \def\listofalofentryname{\listoflofentryname}% gleicher Präfix für Abbildungen im Anhangs-LoF wie im LoF
% \def\listofalotentryname{\listoflotentryname}% gleicher Präfix für Tabellen im Anhangs-LoT wie im LoT

\makeatletter
\AfterTOCHead[atoc]{\let\if@dynlist\if@tocleft}% <- gleiches Verhalten (gratuated oder flat) wie toc 
\newcommand*{\useappendixtocs}{%
  \renewcommand*{\ext@toc}{atoc}%
  \scr@ifundefinedorrelax{hypersetup}{}{%
    \hypersetup{bookmarkstype=atoc}%
  }%
  \renewcommand*{\ext@figure}{alof}%
  \renewcommand*{\ext@table}{alot}%
}
\newcommand*{\usestandardtocs}{%
  \renewcommand*{\ext@toc}{toc}%
  \scr@ifundefinedorrelax{hypersetup}{}{%
    \hypersetup{bookmarkstype=toc}%
  }%
  \renewcommand*{\ext@figure}{lof}%
  \renewcommand*{\ext@table}{lot}%
}
\scr@ifundefinedorrelax{ext@toc}{%
  \newcommand*{\ext@toc}{toc}
  \renewcommand{\addtocentrydefault}[3]{%
    \expandafter\tocbasic@addxcontentsline\expandafter{\ext@toc}{#1}{#2}{#3}%
  }
}{}
\makeatother
 
\usepackage{xpatch}
\xapptocmd\appendix{%
%   \addpart{\appendixname}
  \useappendixtocs
  \listofatocs
%   \listofalofs
%   \listofalots
}{}{}

%%%%%%%%%%%%%%%%%%%%%%%%%%%%%%%%%%%%%%%%%%%%
% Quotes before chapter
%%%%%%%%%%%%%%%%%%%%%%%%%%%%%%%%%%%%%%%%%%%%
\definecolor{quotemark}{gray}{0.7}
\makeatletter
\newlength\origparskip

% \newcommand{\fquote}{}

\newcommand{\fquote}{%
  \@ifnextchar[{\fquote@i}{\fquote@i[]}%]
}

\def\fquote@i[#1]{%
  \@ifnextchar[{\fquote@ii{#1}}{\fquote@ii{#1}[]}%]
}%

\def\fquote@ii#1[#2]{%
  \def\pqm@tempa{#1}%
  \def\pqm@tempb{#2}%
  \noindent
  \list
    {}
    {\setlength{\leftmargin}{0.3\textwidth}%
     \setlength{\rightmargin}{0.1\textwidth}%
     \setlength{\origparskip}{\parskip}}%
    \item[]%
      \begin{picture}(0,0)%
        \put(-15,-8){\makebox(0,0){\scalebox{4}{%
          \textcolor{ees_blue}{\textquotedblright}}}}%
      \end{picture}%
      \begingroup
      \itshape
      \ignorespaces}%

\def\endfquote{%
  \endgroup
  \par
  \raggedleft
  \ifx\pqm@tempa\empty
  \else
    {\bfseries --- \pqm@tempa\par}%
    \setlength{\parskip}{\origparskip}%
    \ifx\pqm@tempb\empty
    \else
      (\pqm@tempb)%
    \fi
  \fi
  \par
  \endlist}
\makeatother

% Math - vectors/matrices: new command for easy typesetting
\newcommand{\mab}[1]{\mathrm{\textbf{#1}}}

% easy access for comments of missing sources
\newcommand{\quelle}[0]{\commenting{\textbf{[Quelle]}}~}
\input{images/timetable.tex}

% \makeglossaries

\begin{document}
	% Fancy Deckblatt
	% %!TEX root = ../main.tex
\begingroup
    \thispagestyle{empty}
    \newgeometry{left=4cm,right=2.5cm,top=2.5cm,bottom=2.5cm}
    \raggedleft
    % \begin{minipage}[r]{.49\textwidth}
        \includegraphics[width=4cm]{images/essential/schaeffler.png} \\
    % \end{minipage}
    \vspace*{6cm}

    \raggedright 
    \Large
    \textbf{Optimization of PTL structure} \\[6pt]
    \textbf{to enhance water and gas transport properties} \\[18pt]
    \colorbox{schaeffler!80}{\makebox(5cm, 5mm){}} \\
    \vspace*{1.5cm}

    \raggedright
    \Large{Master Thesis} \\
    \raggedleft
    \vspace*{1cm}
    \normalsize

    \vfill

    \begin{minipage}[r]{11cm}
        \raggedleft
        \includegraphics[width=8cm]{images/essential/fausiegel.pdf} \\
    \end{minipage}

    \vspace*{1cm}
    \newpage
    \section*{}
    \thispagestyle{empty}
\endgroup
	% \newpage

	% Deckblatt
	\begin{spacing}{1}
		%!TEX root = ../main.tex

\begin{titlepage}
\newgeometry{left=2.5cm,right=2.5cm,top=2cm,bottom=2.5cm}

\centering
\begin{figure}[t]
    \begin{minipage}[]{0.49\textwidth}
        \flushleft
        \includegraphics[width=4.5cm]{images/essential/schaeffler.png}\\
		% \vspace*{48pt}
		% \textsc{Schaeffler Technologies AG \& Co. KG}\\
		% R\&D Hydrogen Industrial
    \end{minipage}
    \begin{minipage}[]{0.49\textwidth}
        \flushright
        \includegraphics[width=4.5cm]{images/essential/HIERN.jpg}\\
		% \vspace*{6pt}
		% \textsc{Schaeffler Technologies AG \& Co. KG}\\
		% R\&D Hydrogen Industrial\\[12pt]
		% \par
		% \textsc{Helmholtz-Institut\\Erlangen-Nürnberg}
		% Vorstand: Prof. Dr. Peter Wasserscheid
    \end{minipage}
\end{figure}

\begin{textblock*}{\textwidth}(107mm,172mm)
	\includegraphics[width=120mm]{images/essential/fausiegel.pdf}
\end{textblock*}

\enlargethispage{20mm}
\vspace{10mm}

\begin{center}
	\doublespacing
	\vspace*{35mm}	
	\begin{minipage}{.7\textwidth}
		\centering
		{\Large{\titel}}
	\end{minipage}																				\\
	\vspace*{10mm}		{\textbf{\MakeUppercase{\arbeit}}}										\\
	\onehalfspacing
	\vfill
	% for obtaining the 																	% \\[5mm]
    % Master of Science																		\\
	% \vspace*{9mm}
	% \vspace*{3mm}		\langartikelstudiengang{} \langstudiengang{} \textbf{\studiengang}	\\
	% \vspace*{3mm}		\langanderdh{} 														\\
	\vspace*{15mm}	    \langvon															\\
	\vspace*{3mm}		{\large\textbf \autor}												\\
	\vspace*{12mm}	    \datumAbgabe														\\
\end{center}

\vspace{15mm}
%%%%%%
%%% Bottom tabular for additional, important information %%%
%%%%%%

\flushleft
\begin{spacing}{1.2}
\begin{tabbing}
		mmmmmmmmmmmmmmmmmmmmmmmm              \= \kill
		% \textbf{\langdbbearbeitungszeit} \> \zeitraum\\
		% \textbf{\langdbmatriknr} \> \matrikelnr\\
		% \textbf{\langdbfirma} \> \firma\\
		% 						\> \firmenort\\
		\textbf{\langdbbetreuer} %\> Dr. ir. Peter Bouwman\\ 
								\> \betreuer\\
		\textbf{\langdbgutachter}              \>  \gutachter\\
\end{tabbing}
\end{spacing}
% \vspace{1cm}

\vspace{1cm}
\restoregeometry
\end{titlepage}
	\end{spacing}
	\newpage

	% stellt Abstand vor Kapitelüberschriften ein
	\RedeclareSectionCommand[beforeskip=\kapitelabstand         ]{chapter}
	% \newgeometry{left=3cm,right=2cm,top=2.5cm,bottom=2.5cm}

	\pagenumbering{Roman}
	\clearpairofpagestyles
	\ohead[]{\textsc{\headmark}}				% Kopfzeile außen immer mit Headmark versehen
	\automark[section]{chapter}		% Headmark bestehend aus Kolumnentitel
	\ofoot[\pagemark]{\pagemark}	% Fußzeile mit Seitenzahl außen
	\renewcommand*\chapterpagestyle{plain.scrheadings}
	% \renewcommand*\partpagestyle{plain.scrheadings}		%Bei Verwendung von Parts als Überschriftenebene: Setzen des Pagestyles global
    
    \setcounter{page}{2}
	% Sperrvermerk
	% \input{ads/sperrvermerk}
    % \newpage
	
	% Erklärung
 	% %!TEX root = ../main.tex

\addchap*{\langerklaerung}
% \thispagestyle{empty}

\vspace*{1.5cm}

\begin{center}
    \begin{tabular}{| p{0.95\textwidth} |}
        \hline
        I confirm that I have written this \arbeit~unaided and without using sources other than those listed and that this thesis has never been submitted to another examination authority and accepted as part of an examination achievement, neither in this form nor in a similar form. All content that was taken from a third party either verbatim or in substance has been acknowledged as such.\\
        \vspace{.5cm}
        Erlangen, \today \\ % \datumAbgabe\\
        \vspace*{.5cm}
        \singlespacing
        \rule{7cm}{.5pt}\\
        \autor\\[12pt]
        \hline
    \end{tabular}
\end{center}

\vfill

\begin{flushright}
    \begin{minipage}[]{0.7\textwidth}
        \textbf{Note:}\\[6pt]
        For reasons of readability, the generic masculine is primarily used in this \arbeit. Female and other gender identities are explicitly included where this is necessary for the statement.
    \end{minipage}
\end{flushright}

\vspace{2cm}
 	% \newpage
	% Inhaltsverzeichnis
	\begin{spacing}{1.2}
		\begingroup
			% auskommentieren für Seitenzahlen unter Inhaltsverzeichnis
			\renewcommand*{\chapterpagestyle}{empty}
			\pagestyle{empty}

			%\setcounter{tocdepth}{1}
			%für die Anzeige von Unterkapiteln im Inhaltsverzeichnis
			\setcounter{tocdepth}{2}

			\tableofcontents
			% \maintoc
			\clearpage
		\endgroup
	\end{spacing}
	\newpage
    % \adjustmtc
    \pagestyle{scrheadings}
	
	% noch ausstehende ToDo's - Liste
	\listoftodos

	% Abkürzungsverzeichnis
 	\clearpage
 	
%%%%%%%%%%%%%%%%%%%%%%%%%%%%%%%%%%%%%%%%%%%%%%%%%%%%%%%%%%%%%%%%
% Anmerkungen zur Verwendung:
%%%%%%%%%%%%%%%%%%%%%%%%%%%%%%%%%%%%%%%%%%%%%%%%%%%%%%%%%%%%%%%%
%
% nur verwendete Akronyme werden letztlich im Abkürzungsverzeichnis des Dokuments angezeigt
% Verwendung: 
%		\ac{Abk.}   --> fügt die Abkürzung ein, beim ersten Aufruf wird zusätzlich automatisch die ausgeschriebene Version davor eingefügt bzw. in einer Fußnote (hierfür muss in header.tex \usepackage[printonlyused,footnote]{acronym} stehen) dargestellt
%		\acs{Abk.}   -->  fügt die Abkürzung ein
%		\acf{Abk.}   --> fügt die Abkürzung UND die Erklärung ein
%		\acl{Abk.}   --> fügt nur die Erklärung ein
%		\acp{Abk.}  --> gibt Plural aus (angefügtes 's'); das zusätzliche 'p' funktioniert auch bei obigen Befehlen
%	siehe auch: http://golatex.de/wiki/%5Cacronym
%
%%%%%%%%%%%%%%%%%%%%%%%%%%%%%%%%%%%%%%%%%%%%%%%%%%%%%%%%%%%%%%%%
\cleardoublepage
\addcontentsline{toc}{chapter}{Acronyms}
\chapter*{Acronyms}

% \addchap{\langabkverz}
\begin{acronym}[mmmmmm] %hier längstes Acro
% \begin{doublespacing}
% \setlength{\itemsep}{-\parsep}

\acro{CCT}{Critical Clearing Time}
\acro{EMT}{Electromagnetic Transient}
\acro{FSM}{Fast Switching Module}
\acro{IBB}{Infinite Bus Bar}
\acro{IEEE}{Institute of Electrical and Electronics Engineers}
\acro{IM}{Induction Machine}
\acro{ODE}{Ordinary Differential Equation}
\acro{OLTC}{On-Load Tap Changer}
\acro{PCC}{Point of Common Coupling}
\acro{PSS}{Power System Simulation}
% \acro{PSS}{Power System Stabilizer} % Das ist die eigentliche korrekte Bedeutung/Abkürzung!!!
\acro{RMS}{Root Mean Square}
\acro{SG}{Synchronous Generator}
\acro{SMIB}{Single Machine Infinite Bus}
\acro{TDS}{Time Domain Solution}
\acro{TVS}{Tangent Vector Index}

\end{acronym}

%%%%%%%%%%%%%%%%%%%%%%%%%%%%%%%%%%%%%%%%%%%%%%%%%%%%%%%%%%%%%%%%
\addcontentsline{toc}{chapter}{Symbols}	
\chapter*{Symbols}

% \addchap{Symbols}
\begin{tabbing}
    XXXXXXXXX \= XXXXXXXX \= XXXXXXXXXXXXXXXXXXXXXXXXXXXXXXXXXXXXXXXXXXXXXXXXX \kill
    $\delta$            \> $^\circ$ / deg                   \> power angle (or power angle difference) \\
    $\Delta\omega$      \> $\mathrm{\frac{1}{s}}$           \> change of rotor angular speed \\
    $\underline{\theta}$\> -                                \> transformer ratio; complex if phase shifting \\
    $A$                 \> -                                \> acceleration or deceleration area \\
    $\underline{E}$     \> V                                \> voltage of \acs{SG} or \acs{IBB} \\
    $H_\mathrm{gen}$    \> s                                \> inertia constant of a \acf{SG} \\
    $\underline{I}$     \> A                                \> current \\
    $P$                 \> W                                \> effective power; electrical or mechanical \\
    $Q$                 \> var                              \> reactive power \\
    $R$                 \> $\mathrm{\Omega}$                \> ohmic resistance \\
    $\underline{S}$     \> VA                               \> apparent power \\
    $\underline{V}$     \> V                                \> voltage \\
    $\underline{X}$     \> $\mathrm{\Omega}$                \> reactance \\
    $\underline{Y}$     \> $\mathrm{\frac{1}{\Omega}}$ / S  \> admittance \\
    $\underline{Z}$     \> $\mathrm{\Omega}$                \> impedance \\
\end{tabbing}

The different symbols are used with different indices, these are semantic and explained in the surrounding context. Following notation is commonly used for mathematical and physical symbols:
\begin{itemize}[noitemsep]
    \item Phasors or complex quantities are underlined (e.g. $\underline{I}$)
    \item Arrows on top mark a spatial vector (e.g. $\overrightarrow{F}$)
    \item Boldface denotes matrices or vectors (e.g. $\mab{F}$)
    \item Roman typed symbols are units (e.g. $\mathrm{s}$)
    \item Lower case symbols denote instantaneous values (e.g. $\underline{i}$)
    \item Upper case symbols denote \acs{RMS} or peak values (e.g. $\underline{I}$)
    % \item References to objects are written capitalized Roman (e.g. $\underline{Z}_\mathrm{TRAFO}$)
    \item Subscripts relating to physical quantities or numerical variables are written italic (e.g. $\underline{I}_1$) 
\end{itemize}

In the simulations and calculations the per unit system ($\mathrm{p.u.}$) is preferred, thus normalizing all values with a base value. Where necessary, absolute units are added to indicate the explicit use of the normal unit system. For more information about this per-unit system please refer to \textcite{machowskiPowerSystemDynamics2020}, specifically Appendix A.1 provides a detailed description and explanation.
	
	\cleardoublepage
	\pagenumbering{arabic}
    
    \pagestyle{scrheadings}		% Kopf- und Fußzeile wie zuvor eingestellt
	
	%\setcounter{footnote}{1}	% Dont start at 2

	% Inhalt
	\foreach \i in {01,02,03,04,05,06,07,08,09,...,99} {%
		\edef\FileName{content/\i kapitel}%
			\IfFileExists{\FileName}{%
				\input{\FileName}
			}
			{%
				%file does not exist
			}
	}

	\clearpage
	
    \pagenumbering{Roman}
    \setcounter{page}{21}
     
	% Literaturverzeichnis
	% \clearpage
	\printbibliography
	% \printbibliography[notkeyword=intern, nottype=patent]
	% \printbibliography[heading=subbibliography, keyword=intern, title={Interne Quellen}]
	% \printbibliography[heading=subbibliography, type=patent, title={Patentschriften}]
	% Unterverzeichnisse siehe: https://texwelt.de/fragen/7532/wie-unterteile-ich-meine-biblatex-bibliografie

	% Glossar
	% \cleardoublepage
	% \input{ads/glossary}
	% \printglossary[style=altlist,title=\langglossar]
    
	% sonstiger Anhang
	\cleardoublepage
	\pagenumbering{Alph}
	\appendix
	% !TeX root = ../main.tex

% \appendixtoc
% \appendix
% \ohead[]{\textsc{Appendix}}
\label{app:appendix}

% \renewcommand\thechapter{\roman{chapter}}
% \setcounter{chapter}{0}

% \pagebreak
% \includepdf[pages=-,scale=.9,pagecommand={}]{Aufgabenstellung.pdf} 
% PDF um 10% verkleinert einbinden --> Kopf- und Fußzeile  werden so korrekt dargestellt. Die Option `pages' ermöglicht es, eine bestimmte Sequenz von Seiten (z.B. 2-10 oder `-' für alle Seiten) auszuwählen.
% \pagebreak
%\includepdf[pages=-,scale=.8,pagecommand=\section*{A. eventGenerator.py}]{../appendix/eventGenerator.py.pdf}
%\includepdf[pages=-,scale=.8,pagecommand=\section*{B. sendEvents.py}]{../appendix/sendEvents.py.pdf}


% \RedeclareSectionCommand[beforeskip=\kapitelabstand         ]{chapter}


%%%%%%%%%%%%%%%%%%%%%%%%%%%%%%%%%%%%%%%%%%

%%%%%%%%%%%%%%%%%%%%%%%%%%%%%%%%%%%%%%%%%%
%%%%%%%%%%%%%%%%%%%%%%%%%%%%%%%%%%%%%%%%%%
\chapter{Fundamentals}

\section{Description of the Power System Simulation process}
\label{app:power-system-modeling}

In this appendix section, the general process of power system simulation is described. As this thesis is aiming to understand voltage stability and processes in longer periods of time, these explanations apply to pointer-based simulations, called RMS simulations. Meaning that the considered effects are slower electromechanical nature instead of faster electromagnetic ones. The in this thesis used Python framework \glqq \textit{diffpssi}\grqq~is based on this type of simulation, and due to its open-source based nature traceable.

\begin{figure}[htbp]
    \centering
    % \includegraphics[width=\textwidth]{fundamentals/power-system-simulation-process.pdf}
    \missingfigure{Power system simulation process}
    \caption{Power system simulation process; own illustration}
    \label{fig:power-system-simulation-process}
\end{figure}

\commenting{
    Really basic: (?)
    \begin{itemize}
        \item RMS vs EMT simulation (-> meaning one cannot simulate other faults than 3ph w/o ground)
        \item Phasor description
        \item Basic formulation: Static (algebraic) and dynamic (differential) equations
        \item Using of solvers (Integrators) for time domain simulation
        \item Using of different optimizatinon algorithms for steady state (load flow) simulation -> initial values
    \end{itemize}
    Less basic and more advanced:
    \begin{itemize}
        \item rountines in the framework
        \item two types: Algebraic and Differential equations have to be solved at each time step -> What is which? Which operational equipment is typically described with which type of equation?
        \item per unit system applying for easier simulation (different voltage levels)
        \item ...
    \end{itemize}
}

%%%%%%%%%%%%%%%%%%%%%%%%%%%%%%%%%%%%%%%%%%
\section{Jacobian based voltage stability criterions}
\label{app:jacobian-voltage-indices}

\textcite{danish_2015} is showing, describing, and referencing some voltage stability indices based on the Jacobian matrix. The following table is a collection of these indices.

\begin{sidewaystable}[htbp!]
    \centering
    \small
    \caption{Jacobian based voltage stability criterions; after \textcite{danish_2015}}
    \vspace*{12pt}
    \renewcommand{\arraystretch}{2}
    \begin{tabularx}{23cm}{llXXl}
        % \toprule
        \textbf{Index} & \textbf{Abbreviation} & \textbf{Calculation} & \textbf{Stability Threshold} & \textbf{Reference} \\
        \toprule
        Tangent Vector Index & \acs{TVI} & $\mathrm{TVI}_i=\left\lvert \frac{\dd{V_i}}{\dd{\lambda}}\right\rvert^{-1}$ & depending on load increase & \\ \midrule
        Test Function & & $t_{cc}=\left\lvert e^T_c \cdot \mab{J} \times \mab{J}_{cc}^{-1} \cdot e_c\right\rvert$ & details are given in reference & \\ \midrule
        Second Order Index & $i$ & $i=\frac{1}{i_0} \cdot \sigma_\mathrm{max} \cdot \big( \dv{\sigma_\mathrm{max}}{\lambda_\mathrm{total}} \big)^{-1}$ & $i > 0$ & \\ \midrule
        Minimum Eigenvalue & & $\Delta V=\sum_{i} \frac{\xi_i\eta_i}{\lambda_i} \Delta Q$ & all eigenvalues should be positive & \\ \midrule
        Minimum Singular Value & & $\begin{bmatrix} \Delta \vartheta \\ \Delta V \end{bmatrix}=\mab{V} \sum^-1 \mab{U}^T \begin{bmatrix} \Delta F \\ \Delta G \end{bmatrix}$ & details are given in reference & \\ \midrule
        Predicting Voltage Collapse & & $\frac{V}{V_0}$ & the smallest index value & \\ \midrule
        Impedance Ratio & & $\frac{Z_ii}{Z_i}$ & $\frac{Z_ii}{Z_i} \leq 1$ & \\
        \bottomrule
    \end{tabularx}
\end{sidewaystable}



%%%%%%%%%%%%%%%%%%%%%%%%%%%%%%%%%%%%%%%%%%
\section{Comparison of System based and Jacobian based indices}
\label{app:jacobian-vs-system-indices}

% %%%%%%%%%%%%%%%%%%%%%%%%%%%%%%%%%%%%%%%%%%
% %%%%%%%%%%%%%%%%%%%%%%%%%%%%%%%%%%%%%%%%%%
\chapter{Modeling}

%%%%%%%%%%%%%%%%%%%%%%%%%%%%%%%%%%%%%%%%%%
\section{Admittance Calculation of a Two-Port}
\label{app:admittance-deduction}

Follwing part shall just give a short, but complete and clear overview, how the admittance matrix of a two-port system is calculated.
Therefore the main focus of this thesis, a two-port with variable translation ratio, is kept.

%%%%%%%%%%%%%%%%%%%%%%%%%%%%%%%%%%%%%%%%%%
\section{Class Diagram of the Class Nose Curves}
\label{app:nose-curve}

\begin{figure}[htbp!]
    \centering
    \includegraphics[width=12cm]{tikz_graphics/images/class_diagram_nosecurve_complete.pdf}
    \caption{Complete class diagram of the class Nose Curves; including all attributes and methods with data types, returns, and inputs}
    \label{fig:class-diagram-nose-curves}
\end{figure}

%%%%%%%%%%%%%%%%%%%%%%%%%%%%%%%%%%%%%%%%%%
\section{Analytical Calculation of Simple Nose Curves}
\label{app:analytical-nose-curve}

Some blibla and equations about the analytical calculation of simple nose curves.


%%%%%%%%%%%%%%%%%%%%%%%%%%%%%%%%%%%%%%%%%%
\section{Alternative Current Injection Model}
\label{app:current-injection-model}

\textcite{machowski_2020} describes another way of modeling a \acs{OLTC} transformer with variable ratio.
This model is looking at the shunt brnaches as current injections, which are added to the individual busses.
Beneficial, the system admittance matrix is staying symmetrical, while the different transformer state(s) are represented by the different current injections.
This can be mathematically expressed by following set of equations:
\begin{align}
    \begin{bmatrix}
        \underline{I}_1 \\
        -\underline{I}_2
    \end{bmatrix} &=
    \begin{bmatrix}
        \underline{Y}_\mathrm{T} & -\underline{Y}_\mathrm{T} \\
        -\underline{Y}_\mathrm{T} & \underline{Y}_\mathrm{T}
    \end{bmatrix}
    \begin{bmatrix}
        \underline{U}_1 \\
        \underline{U}_2
    \end{bmatrix} -
    \begin{bmatrix}
        \Delta \underline{I}_1 \\
        \Delta \underline{I}_2
    \end{bmatrix}\text{, where } \notag \\[12pt]
    \begin{bmatrix}
        \Delta \underline{I}_1 \\
        \Delta \underline{I}_2
    \end{bmatrix} &=
    \begin{bmatrix}
        \underline{0} & (\underline{\vartheta}-1)\underline{Y}_\mathrm{T} \\
        -(\underline{\vartheta}^*+1)\underline{Y}_\mathrm{T} & (\underline{\vartheta}^*\underline{\vartheta}+1)\underline{Y}_\mathrm{T}
    \end{bmatrix}
    \begin{bmatrix}
        \underline{U}_1 \\
        \underline{U}_2
    \end{bmatrix} \text{ leading to } \notag \\[12pt]
    \underline{\mab{Y}}_\mathrm{\Pi,T,Current~Injection}&= 
    \begin{bmatrix}
        \underline{Y}_\mathrm{T} & -\underline{Y}_\mathrm{T} \\
        -\underline{Y}_\mathrm{T} & \underline{Y}_\mathrm{T}
    \end{bmatrix} -
    \begin{bmatrix}
        \underline{0} & (\underline{\vartheta}-1)\underline{Y}_\mathrm{T} \\
        -(\underline{\vartheta}^*+1)\underline{Y}_\mathrm{T} & (\underline{\vartheta}^*\underline{\vartheta}+1)\underline{Y}_\mathrm{T}
    \end{bmatrix} \notag % \label{eq:admittance-oltc-2}
\end{align}

% %%%%%%%%%%%%%%%%%%%%%%%%%%%%%%%%%%%%%%%%%%
% \section{OLTC control}


% %%%%%%%%%%%%%%%%%%%%%%%%%%%%%%%%%%%%%%%%%%
% %%%%%%%%%%%%%%%%%%%%%%%%%%%%%%%%%%%%%%%%%%
% \chapter{Verification}

% %%%%%%%%%%%%%%%%%%%%%%%%%%%%%%%%%%%%%%%%%%
% \section{Single-machine infinite bus-bar model}
% \label{app:smib-model}


% %%%%%%%%%%%%%%%%%%%%%%%%%%%%%%%%%%%%%%%%%%
% %%%%%%%%%%%%%%%%%%%%%%%%%%%%%%%%%%%%%%%%%%
% \chapter{Case study}

\end{document}
