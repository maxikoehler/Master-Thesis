\documentclass[
    % final,
    12pt,               										% Schriftgr��e
    a4paper,            										% Layout f�r DINA4
    german,             										% deutsche Sprache, global
    twoside,            										% Layout f�r beidseitigen Druck
    % headings=openright,
    headinclude=true,        									% Kopfzeile wird Seiten-Layouts mit ber�cksichtigt
    abstracton,         									% Abstract Überschriften
    headsepline,        									% horizontale Linie unter Kolumnentitel
    plainheadsepline,										% horizontale Linie unter Kolumnentitel auch bei Chapter
    BCOR=20mm,           									% Korrektur f�r die Bindung
    DIV=18,              									% DIV-Wert f�r die Erstellung des Satzspiegels, siehe scrguide
    parskip=half,       									% Absatzabstand statt Absatzeinzug
    % openany,            										% Kapitel k�nnen auf geraden und ungeraden Seiten beginnen
    bibliography=totoc,version=first, 						% Literaturverz. wird ins Inhaltsverzeichnis eingetragen
    numbers=noenddot,   								% Kapitelnummern immer ohne Punkt
    captions=tableheading,version=first, 					% korrekte Abst�nde bei Tabellen�berschriften
    fleqn,             										% fleqn: Glgen links (statt mittig)
    listof=totoc,version=first,								% Abbildungs- und Tabellenverzeichnis ins Inhaltsverzeichnis
    cleardoublepage=empty,								% Kopfzeile bei leeren Seiten entfernen
    footinclude=false,										% Fußzeile wird Seiten-Layouts nicht mit berücksichtigt
    footheight=0pt,
    autoenlargeheadfoot=false,                            % keine automatische Vergrößerung von Kopf- und Fußzeile
]{scrreprt}

\raggedbottom

\title{Modeling of Fast-Switching Transformers for Voltage Stability Studies in Python}
\author{Maximilian Markus Veit Köhler}
\date{02.05.2025}


%--------------- Packages ----------------
\usepackage[ngerman, english]{babel}
\usepackage[utf8]{inputenc}							% direkte Eingabe von Umlauten & Co.
\usepackage[T1]{fontenc}								% T1-Schriften
\usepackage{lmodern}								% moderne Schriften
\usepackage[format=hang,							% Captions ausrichten
justification=raggedright,
singlelinecheck=off,
labelfont=bf,
font=small,
skip=4pt
]{caption} 											
\usepackage[centertags]{amsmath} 					% AMS-Mathematik, centertags zentriert Nummer bei split
\usepackage{tabularx}								% erweiterte Tabellen
\usepackage{graphicx}            							% zum Einbinden von Grafiken
\usepackage{setspace}            							% Zeilenabstand einstellbar
\usepackage{scrlayer-scrpage}           					% Kopf- und Fu�zeilen-Layout 
\usepackage[pdfborderstyle={/S/U/W 1}]{hyperref}		% Die Links im PDF werden nur noch d�nn und rot unterstrichen, nicht mehr fett umrahmt. 
\usepackage[absolute]{textpos}					% benutzerdefinierte Textpositionierung 


%--------------- Sonstiges ----------------

\pagestyle{scrheadings}									% Kopf- und Fu�zeile...
\clearpairofpagestyles									% ...leeren
\ihead[]{\headmark}              								% Kopfzeile innen
\automark[section]{chapter}		% Headmark bestehend aus Kolumnentitel
\ohead[\pagemark]{\pagemark}     							% Kopfzeile au�en
% \ifoot[]{} 												% Fu�zeile innen
% \ofoot[]{}    												% Fu�zeile au�en	
\setlength{\headheight}{1.5\baselineskip}
\onehalfspacing											% 1,5 Zeilenabstand
%\typearea[current]{current}        							% Neuberechnung des Satzspiegels mit alten Werten nach �nderung von Zeilenabstand,etc
% \renewcommand{\bibname}{Literatur und Quellen} 			% Literaturverzeichnisbezeichnung
% \renewcommand{\figurename}{Abb.}   						% Abbildungsbezeichnung
% \renewcommand{\listfigurename}{Abbildungsverzeichnis} 	% Abbildungsverzeichnisbezeichnung
\renewcommand{\captionfont}{\small}						% Bildunterschriften klein kursiv
\graphicspath{{images/}{tikz_graphics/images/}}    							% Bildverzeichnis

\RedeclareSectionCommand[beforeskip=0pt]{chapter}		%kein Abstand zum oberen Seitenrand bei neuen Kapiteln

%%%%%%%%%%%%%%%%%%%%%%%%%%%%%%%%%%%%%%%%%%%%%%%%%%%%%
% Some more used packages
%%%%%%%%%%%%%%%%%%%%%%%%%%%%%%%%%%%%%%%%%%%%%%%%%%%%%
% \usepackage{charter}
% \setkomafont{disposition}{%
% 	\normalfont\bfseries
% }
% \setkomafont{dictum}{\normalfont}                   % Kapitelüberschriften nicht in KOMA Font, sondern in normaler Schriftart
% \renewcommand{\headfont}{\normalfont\sffamily}    		% Kolumnentitel serifenlos
% \renewcommand{\pnumfont}{\normalfont\sffamily}    		% Seitennummern serifenlos

\usepackage{tikz}
\usetikzlibrary{positioning}
\usepackage{tabularx}
\usepackage{amsmath}
\usepackage{amssymb}
\usepackage{float}
% \usepackage{marginnote}
% \usepackage{snotez}
% \setsidenotes{
%   note-mark-format={\hspace*{-2pt}},
%   text-mark-format={\hspace*{-4pt}},
%   text-format+={\RaggedRight\color{ees_blue}\itshape},
%   sidefloat-format={\RaggedRight\color{ees_blue}\itshape\footnotesize},
%   perpage=true
% }
\usepackage{booktabs}
\usepackage{mathtools}
\usepackage{nicematrix}
\usepackage{physics}
\usepackage{enumitem}
\usepackage{csquotes}
\usepackage{ragged2e}
\usepackage{circuitikz}
\usepackage[printonlyused]{acronym}
\usepackage[obeyFinal,backgroundcolor=ees_yellow,linecolor=black, figwidth=.9\linewidth,figcolor=white,textwidth=3cm]{todonotes}
% \usepackage{sidenotes}
\usepackage{lipsum}
\usepackage{tcolorbox}
\usepackage{wrapfig}
\usepackage{xcolor}
\usepackage{rotating}
\usepackage{longtable}
\usepackage{enumitem}	% mehr Optionen bei Aufzählungen
\setlist{itemsep=0pt, parsep=0pt}
\usepackage{graphicx}
\usepackage[right]{eurosym}
\DeclareUnicodeCharacter{20AC}{\euro}
\usepackage{lscape}
\usepackage{bookmark} %nur ein latex-Durchlauf für die Aktualisierung von Verzeichnissen nötig
\usepackage{listings}
\usepackage{comment}
\usepackage{subcaption}


%%%%%%%%%%%%%%%%%%%%%%%%%%%%%%%%%%%%%%%%%%%%%%%%%%%%%
% Citation and bibliobraphy
%%%%%%%%%%%%%%%%%%%%%%%%%%%%%%%%%%%%%%%%%%%%%%%%%%%%%
\usepackage[
	backend=biber,		% empfohlen. Falls biber Probleme macht: bibtex
	bibwarn=true,
	bibencoding=utf8,	% wenn .bib in utf8, sonst ascii
	% sortlocale=en_US,
	sorting=none,
	style=ieee,
]{biblatex}
\setcounter{biburlnumpenalty}{100}
\setcounter{biburlucpenalty}{100}
\setcounter{biburllcpenalty}{100}

\addbibresource{literatur.bib}

%%%%%%%%%%%%%%%%%%%%%%%%%%%%%%%%%%%%%%%%%%%%%%%%%%%%%
% Colors of the Chair
%%%%%%%%%%%%%%%%%%%%%%%%%%%%%%%%%%%%%%%%%%%%%%%%%%%%%
\definecolor{ees_blue}{RGB}{0, 112, 192}
\definecolor{ees_yellow}{RGB}{213, 223, 0}
\definecolor{ees_green}{RGB}{0, 166, 74}
\definecolor{ees_red}{RGB}{192, 0, 0}
\definecolor{ees_lightblue}{RGB}{0, 176, 240}
\definecolor{ees_black}{RGB}{0, 0, 0}

%% Farben (Angabe in HTML-Notation mit großen Buchstaben)
\newcommand{\ladefarben}{%
	\definecolor{LinkColor}{HTML}{00007A}
	% \definecolor{ListingBackground}{HTML}{FCF7DE}
}

%%%%%%%%%%%%%%%%%%%%%%%%%%%%%%%%%%%%%%%%%%%%%%%%%%%%%
% Own commands
%%%%%%%%%%%%%%%%%%%%%%%%%%%%%%%%%%%%%%%%%%%%%%%%%%%%%
\newcommand{\mycomment}{}
\newcommand{\commenting}[1]{{\color{ees_red} #1}} % Kommentar anzeigen
\newcommand{\ai}[1]{{\color{ees_green} #1}} % Kommentar anzeigen
% easy access for comments of missing sources
\newcommand{\quelle}[0]{\commenting{\textbf{[Quelle]}}~}
% Math - vectors/matrices: new command for easy typesetting
\newcommand{\mab}[1]{\mathrm{\textbf{#1}}}

% catching all beauties
\newcommand{\sidenote}[1]{}
\newcommand{\sidecaption}{}
\newcommand{\marginnote}[1]{}
\newcommand{\prefacelogo}{}
% \newenvironment{sidefigure}{}{}%{\begin{sidefigure}}{\end{sidefigure}}

%%%%%%%%%%%%%%%%%%%%%%%%%%%%%%%%%%%%%%%%%%%%%%%%%%%%%
% Preliminary comment on each page
%%%%%%%%%%%%%%%%%%%%%%%%%%%%%%%%%%%%%%%%%%%%%%%%%%%%%
\DeclareNewTOC[%
  owner=\jobname, 
  listname={Appendix},
]{atoc}

\makeatletter
\AfterTOCHead[atoc]{\let\if@dynlist\if@tocleft}% <- gleiches Verhalten (gratuated oder flat) wie toc 
\newcommand*{\useappendixtocs}{%
  \renewcommand*{\ext@toc}{atoc}%
  \scr@ifundefinedorrelax{hypersetup}{}{%
    \hypersetup{bookmarkstype=atoc}%
  }%
  \renewcommand*{\ext@figure}{alof}%
  \renewcommand*{\ext@table}{alot}%
}
\newcommand*{\usestandardtocs}{%
  \renewcommand*{\ext@toc}{toc}%
  \scr@ifundefinedorrelax{hypersetup}{}{%
    \hypersetup{bookmarkstype=toc}%
  }%
  \renewcommand*{\ext@figure}{lof}%
  \renewcommand*{\ext@table}{lot}%
}
\scr@ifundefinedorrelax{ext@toc}{%
  \newcommand*{\ext@toc}{toc}
  \renewcommand{\addtocentrydefault}[3]{%
    \expandafter\tocbasic@addxcontentsline\expandafter{\ext@toc}{#1}{#2}{#3}%
  }
}{}
\makeatother
 
\usepackage{xpatch}
\xapptocmd\appendix{%
%   \addpart{\appendixname}
  \useappendixtocs
  \listofatocs
%   \listofalofs
%   \listofalots
}{}{}


%%%%%%%%%%%%%%%%%%%%%%%%%%%%%%%%%%%%%%%%%%%%%%%%%%%%%
% Preliminary comment on each page
%%%%%%%%%%%%%%%%%%%%%%%%%%%%%%%%%%%%%%%%%%%%%%%%%%%%%
\usepackage{scrtime}
\usepackage{prelim2e}
\renewcommand{\PrelimText}{\textcolor{red}{\textbf{\footnotesize[\today\ at \thistime\ -- preliminary version 0.1]}}}

%%%%%%%%%%%%%%%%%%%%%%%%%%%%%%%%%%%%%%%%%%%%%%%%%%%%%
% Additional stuff
%%%%%%%%%%%%%%%%%%%%%%%%%%%%%%%%%%%%%%%%%%%%%%%%%%%%%
% Hurenkinder und Schusterjungen verhindern
% http://projekte.dante.de/DanteFAQ/Silbentrennung
\clubpenalty = 10000 % schließt Schusterjungen aus (Seitenumbruch nach der ersten Zeile eines neuen Absatzes)
\widowpenalty = 10000 % schließt Hurenkinder aus (die letzte Zeile eines Absatzes steht auf einer neuen Seite)
\displaywidowpenalty=10000

% Bildpfad
\graphicspath{{images/}{../diffpssi-ma-kohler/}}

% Abstände in Tabellen
\setlength{\tabcolsep}{10pt}
\renewcommand{\arraystretch}{1.3}

% Einige häufig verwendete Sprachen
\lstloadlanguages{PHP,Python,Java,C,C++,bash}
%% Programmiersprachen Highlighting (Listings)
\newcommand{\listingsettings}{%
	\lstset{%
		language=Java,			% Standardsprache des Quellcodes
		numbers=left,			% Zeilennummern links
		stepnumber=1,			% Jede Zeile nummerieren.
		numbersep=5pt,			% 5pt Abstand zum Quellcode
		numberstyle=\tiny,		% Zeichengrösse 'tiny' für die Nummern.
		breaklines=true,		% Zeilen umbrechen wenn notwendig.
		breakautoindent=true,	% Nach dem Zeilenumbruch Zeile einrücken.
		postbreak=\space,		% Bei Leerzeichen umbrechen.
		tabsize=2,				% Tabulatorgrösse 2
		basicstyle=\ttfamily\footnotesize, % Nichtproportionale Schrift, klein für den Quellcode
		showspaces=false,		% Leerzeichen nicht anzeigen.
		showstringspaces=false,	% Leerzeichen auch in Strings ('') nicht anzeigen.
		extendedchars=true,		% Alle Zeichen vom Latin1 Zeichensatz anzeigen.
		captionpos=b,			% sets the caption-position to bottom
		% backgroundcolor=\color{ListingBackground}, % Hintergrundfarbe des Quellcodes setzen.
    backgroundcolor=\color{white!50},
		xleftmargin=0pt,		% Rand links
		xrightmargin=0pt,		% Rand rechts
		frame=single,			% Rahmen an
		frameround=ffff,
		rulecolor=\color{black},	% Rahmenfarbe
		% fillcolor=\color{ListingBackground},
    fillcolor=\color{ees_blue!50},
		keywordstyle=\color[rgb]{0.133,0.133,0.6},
		commentstyle=\color[rgb]{0.133,0.545,0.133},
		%stringstyle=\color[rgb]{0.627,0.126,0.941}
		stringstyle=\color{red}
	}
}
\listingsettings{}
% Umbennung des Listings
\renewcommand\lstlistingname{Listing}
\renewcommand\lstlistlistingname{List of Listings}
\def\lstlistingautorefname{Listing}

\lstdefinestyle{style-python}
{
  language=python,	% Programmiersorache einstellen, Latex erkennt dann automatisch code, kommentare, funktionen,...
  basicstyle=\scriptsize\ttfamily,	% Schriftformatierung, ttfamily: text im Schreibmaschinen-Style
  backgroundcolor=\color{white},		% Hintergrundfarbe
  breaklines=true,	% automatischer Zeilenumbruch bei langen Zeilen, funktioniert nur bei bestimmen Zeichen, z.B. Umbruch nach Leerzeichen
  keywordstyle=\bfseries\ttfamily\color{blue},	% Schlüsselwörter einstellen
  stringstyle=\ttfamily\color{ees_yellow},			% Textsrtrings
  showstringspaces=false,	% Leerzeichen in Strings richtig darstellen
  commentstyle=\color{ees_green}\ttfamily,	% Kommentare
  flexiblecolumns=false,	% Spaltenbreite dynamisch/fest
  numbers=left,		% Position der Zeilennummern
  numberstyle=\tiny,	% Größe der Zeilennummern
  numberblanklines=false,		% leere Zeilen werde mit ‚false‘ nicht durchnummeriert
  stepnumber=1,		% Beginn der Nummerierung
  numbersep=10pt,		% Abstand zwischen Zeilennummern und Quellcode
  xleftmargin=20pt,	% Abstand zum linken Rand
  xrightmargin=10pt,	% Abstand zum rechten Rand
  extendedchars=true,	% Sonderzeichen korrekt darstellen
  frame=trbl,			% Rahmen um gesamten Code: Top, right, bottom, left (Großbuchstaben ergeben Doppellinien)
  frameround=ffff,	% Ecken des Rahmens anpassen, t: runde Ecken, f: default (eckig), es müssen 4 Buchstaben da stehen!
  literate=		% ersetzen von Zeichen 1 durch Zeichen 2, hier: korrekte Einbindung der Sonderzeichen
   {Ö}{{\"O}}1 
   {Ä}{{\"A}}1 
   {Ü}{{\"U}}1 
   {ß}{{\ss}}1 
   {ü}{{\"u}}1 
   {ä}{{\"a}}1 
   {ö}{{\"o}}1,
  % mit ‚emph‘ und ‚emphstyle‘ können eigene Styles für Wörter angelegt werden
  emph = [1]{clc, color},
  emphstyle = [1]{\color{blue}},
  emph = [2]{function, endfunction},
  emphstyle = [2]{\color{ees_red}},
  emph = [3]{gcf},
  emphstyle = [3]{\color{black}},
}

%%%%%%%%%%%%%%%%%%%%%%%%%%%%%%%%%%%%%%%%%%%%%%%%%%%%%
% Testing siome stuff
%%%%%%%%%%%%%%%%%%%%%%%%%%%%%%%%%%%%%%%%%%%%%%%%%%%%%
\newcommand{\einstellung}[1]{%
  \expandafter\newcommand\csname #1\endcsname{}
  \expandafter\newcommand\csname setze#1\endcsname[1]{\expandafter\renewcommand\csname#1\endcsname{##1}}
}
\newcommand{\langstr}[1]{\einstellung{lang#1}}
\einstellung{offset}
\setzeoffset{100mm}

\definecolor{quotemark}{gray}{0.7}
\makeatletter
\newlength\origparskip

% \newcommand{\fquote}{}

\newcommand{\fquote}{%
  \@ifnextchar[{\fquote@i}{\fquote@i[]}%]
}

\def\fquote@i[#1]{%
  \@ifnextchar[{\fquote@ii{#1}}{\fquote@ii{#1}[]}%]
}%

\def\fquote@ii#1[#2]{%
  \def\pqm@tempa{#1}%
  \def\pqm@tempb{#2}%
  \noindent
  \list
    {}
    {\setlength{\leftmargin}{0.3\textwidth}%
     \setlength{\rightmargin}{0.1\textwidth}%
     \setlength{\origparskip}{\parskip}}%
    \item[]%
      \begin{picture}(0,0)%
        \put(-15,-8){\makebox(0,0){\scalebox{4}{%
          \textcolor{ees_blue}{\textquotedblright}}}}%
      \end{picture}%
      \begingroup
      \itshape
      \ignorespaces}%

\def\endfquote{%
  \endgroup
  \par
  \raggedleft
  \ifx\pqm@tempa\empty
  \else
    {\bfseries --- \pqm@tempa\par}%
    \setlength{\parskip}{\origparskip}%
    \ifx\pqm@tempb\empty
    \else
      (\pqm@tempb)%
    \fi
  \fi
  \par
  \endlist}
\makeatother

% circles as enumeration
\newcommand*\circledblue[1]{\tikz[baseline=(char.base)]{%
            \node[shape=circle,fill=ees_blue!40,draw,inner sep=2pt] (char) {#1};}}
\newcommand*\circled[1]{\tikz[baseline=(char.base)]{%
            \node[shape=circle,fill=white,draw,inner sep=2pt] (char) {#1};}}
%!TEX root = ../main.tex

\usetikzlibrary{positioning,calc,shapes,arrows,shapes.multipart}

% circuitikz: creating a bus
\tikzset{
   bus/.style={fullgeneric, %
        bipoles/fullgeneric/width=0.02, bipoles/fullgeneric/height=#1
   },
   bus/.default=3
}
% Some predefined styles for the blocks
\tikzset{
    block/.style = {draw, fill=white, rectangle, minimum height=3em, minimum width=3em},
    tmp/.style  = {coordinate}, 
    sum/.style= {draw, fill=white, circle, node distance=1cm},
    input/.style = {coordinate},
    output/.style= {coordinate},
    pinstyle/.style = {pin edge={to-,thin,black}}
    limblock/.style = {
            draw, 
            fill=white, 
            rectangle, 
            minimum height=3em, 
            minimum width=3em
        },
}
\newcommand{\bushere}[3]{% length, text above, text below
% optional arguments do not work in paths
    %
    % starting point; draw an edge and then two nodes
    % save the position
    coordinate(tmp)
    % go up and do an edge down
    ++(0,#1) node[anchor=base]{#2} edge[ultra thick] ++(0,{-2*#1})
    % edges do not move the current point, go down to position the node
    ++(0,{-2*#1}) node[below]{#3}
    % go back to where we started
    (tmp)
}
\newcommand{\qbushere}[3]{% length, text above, text below
% optional arguments do not work in paths
    %
    % starting point; draw an edge and then two nodes
    % save the position
    coordinate(tmp)
    % go up and do an edge down
    ++(#1,0) node[anchor=base]{#2} edge[ultra thick] ++({-2*#1},0)
    % edges do not move the current point, go down to position the node
    ++({-2*#1},0) node[below]{#3}
    % go back to where we started
    (tmp)
}
\newcommand{\loadside}[3]{
   coordinate(tmp)
   -| ++(0.5,-1) node[fill=white,shape=regular polygon, rotate=180, regular polygon sides=3,minimum size=0.8,draw](){} ++(#2,#3) node[below]{#1}
   (tmp)
}
\newcommand{\loadhanging}[3]{
   coordinate(tmp)
   -- ++(0,-1) node[fill=white,shape=regular polygon, rotate=180, regular polygon sides=3,minimum size=0.8,draw](){} ++(#2,#3) node[below]{#1}
   (tmp)
}

% program plan
\tikzset{
   papDecision/.style = {
         diamond,
         draw, 
         text width = 20 mm, 
         align = center, 
         text badly centered,
         inner sep = 1 pt,
         font=\ttfamily\footnotesize,
         %line width = 1,
         minimum width = 30mm,
         minimum height = 7mm,
      },
   papStart/.style = {
         rectangle,
         draw, 
         align = center, 
         text width = 3cm, 
         text badly centered,
         inner sep = 4 pt,
         rounded corners=10pt,
         font=\ttfamily\footnotesize,
         %line width = 1,
         minimum width = 30mm,
         minimum height = 7mm,
      },
   papEnd/.style = {
         rectangle,
         draw, 
         align = center, 
         text width = 3cm, 
         text badly centered,
         inner sep = 4 pt,
         rounded corners=10pt,
         font=\ttfamily\footnotesize,
         %line width = 1,
         minimum width = 30mm,
         minimum height = 7mm,
      },
   papData/.style = {
         trapezium,
         draw, 
         align = center, 
         text width = 20 mm, 
         text badly centered,
         inner sep = 4 pt,
         trapezium left angle=70,
         trapezium right angle=110,
         font=\ttfamily\footnotesize,
         %line width = 1,
         minimum width = 30mm,
         minimum height = 7mm,
      },
   papPredProc/.style = {
         draw,
         rectangle split,
         rectangle split horizontal,
         rectangle split parts = 3,
         rectangle split empty part width=-8pt,
         align = center, 
 %       text width = 4.5 em, 
         text badly centered,
 %        inner sep = 4 pt,
         font=\ttfamily\footnotesize,
         %line width = 1,
         minimum width = 30mm,
         minimum height = 7mm,
      },
   papProcess/.style = {
         rectangle,
         draw,
         align = center, 
         text width = 3cm, 
         text badly centered,
         %inner sep = 2 pt,
         font=\ttfamily\footnotesize,
         %line width = 1,
         minimum width = 30mm,
         minimum height = 7mm,
      },
   papLine/.style = {
         draw,
         -stealth,
         font=\ttfamily\footnotesize,
         %line width = 1,
      },
}
\newcommand{\papYes}{ja}
\newcommand{\papNo}{nein}

\makeindex


%----------------------- Beginn des Dokuments -----------------------
\begin{document}

\excludecomment{sidefigure}

\pagenumbering{Roman}									% R�mische Seitennummerierung

\thispagestyle{empty}
\setcounter{page}{-1}

\begin{textblock*}{\textwidth}(30mm,33mm)
\includegraphics[width=47mm]{ees}
\end{textblock*}

\begin{textblock*}{\textwidth}(127mm,192mm)
\includegraphics[width=93mm]{fausiegel}
\end{textblock*}

{\raggedleft
\textsc{Chair for electrical enegy systems}\\
Chairman: Univ.-Prof. Dr.-Ing. Matthias Luther \\
Univ.-Prof. Dr.-Ing. Johann Jäger
\par}

\vspace{51mm}

{\centering

%##########################################################
% BEARBEITUNGSFELDER
\large{Master Thesis M347} \\	
\Large{Modeling of Fast-Switching Transformers\\for Voltage Stability Studies in Python}
%##########################################################

\par}

% \vspace{105mm}
\vfill

{\raggedright
\begin{tabbing}
XX \= XXXXXXXXXXXXXXXXX \= XXXXXXXXXXXXXXXXXXXXXX \kill
%##########################################################
% BEARBEITUNGSFELDER
		\> \textbf{Author:} 	\> Maximilian Markus Veit Köhler, B.Eng. \\
		\>						\> 23176975		\\
 		\>												\>									\\
		\> \textbf{Supervisor:}		\> Ilya Burlakin, M.Sc. \\
		\> 							\> Georg Kordowich, M.Sc. \\
		\>												\>									\\
		\> \textbf{Date of submission:}	\> 02nd May 2025															
%##########################################################
\end{tabbing}
\par}
\cleardoublepage									% Deckblatt
%!TEX root = ../main.tex

\thispagestyle{plain}
\begingroup
\begin{fquote}[Wim Crouwel]
    You can’t do better design with a computer, but you can speed up your work enormously.
\end{fquote}
\let\cleardoublepage\relax
\vspace{-1cm}

\chapter*{Acknowledgements}
\addcontentsline{toc}{chapter}{Acknowledgements}
\endgroup

Who to thank, which contributions, whatever\dots
Some text to fill a whole line with is some blibla with some explanation making no sense at all but just writing some characters.

\begin{flushright}
    Maximilian Köhler
\end{flushright}
%!TEX root = ../main.tex

\cleardoublepage

\renewcommand{\abstractname}{Abstract} % Text für Überschrift
% \phantomsection\addcontentsline{toc}{chapter}{\abstractname}
\begin{abstract}
    English version of this Master Thesis abstract.
\end{abstract}

%%%%%%%%%%%%%%%%%%%%%%%%%%%%%%%%%%%%%%%%%%%%%%%%%%%%%%%%%%%%%%%%%%%%%%%%%%%%%%%%%%%%%%%%%%%%
\cleardoublepage
%%%%%%%%%%%%%%%%%%%%%%%%%%%%%%%%%%%%%%%%%%%%%%%%%%%%%%%%%%%%%%%%%%%%%%%%%%%%%%%%%%%%%%%%%%%%

\begin{otherlanguage}{german}
\renewcommand{\abstractname}{Kurzfassung}
% \phantomsection\addcontentsline{toc}{chapter}{\abstractname}
\begin{abstract}
    Deutsche Version der Kurzzusammenfassung.
\end{abstract}
\end{otherlanguage}									% Abstract
I confirm that I have written this master thesis unaided and without using sources other than those listed and that this thesis has never been submitted to another examination authority and accepted as part of an examination achievement, neither in this form nor in a similar form. All content that was taken from a third party either verbatim or in substance has been acknowledged as such.
\vskip 0.5cm

\begin{table}[!htb]
\centering
\begin{tabularx}{\textwidth}{lXl}

       \hspace{6cm} &  & \hspace{6cm} \\
\cline{1-1}\cline{3-3}
Location, Date  &  & Sign
\end{tabularx}
\end{table}								% Erkl�rung der Eigenarbeit
% \chapter*{Assignment of the Thesis}
%\vspace*{4cm}

{\large \textbf{Topic:} \parbox[t]{0.8\textwidth}{Modeling of Fast-Switching Transformers\\for Voltage Stability Studies in Python}}
\newline

Hier wird die Aufgabenstellung beschrieben. Die Notwendigkeit dieser hängt vom Betreuer ab.
 									% Aufgabenstellung (optional)
\cleardoublepage
\begin{spacing}{1.15}									% evtl. kleinerer Zeilenabstand im IV, AV, TV
\pdfbookmark[1]{Contents}{toc}					% Inhaltsverzeichnis bei den Lesezeichen rein
\tableofcontents 											% Inhaltsverzeichnis erzeugen
\end{spacing}
\cleardoublepage

% \mainmatter												% Hauptteil beginnt
\pagenumbering{arabic}
\setcounter{page}{1}
%!TEX root = ../main.tex
\begingroup
\RedeclareSectionCommand[beforeskip=4.5cm]{chapter}
%%%%%%%%%%%%%%%%%%%%%%%%%%%%%%%
%%%%%%%%%%%%%%%%%%%%%%%%%%%%%%%
\chapter{Revitalization of the OLTC} % for Modern Grid Requirements}
\label{chap:intro}

% \begin{textblock*}{.7\textwidth}(70mm-\offset,25mm-\offset)%(70mm,25mm)
%     \begin{fquote}[Albert Einstein]
%         % All models are wrong, but some are useful.
%         Two things are infinite: the universe and human stupidity; and I'm not sure about the universe.
%     \end{fquote}
% \end{textblock*}
\begin{textblock*}{.7\textwidth}(70mm-\offset,25mm-\offset)%(70mm,25mm)
    \begin{fquote}[Albert Einstein]
        For knowledge is limited to all we know and understand, while imagination embraces the entire world, and all there will be ever to know and understand.
    \end{fquote}
\end{textblock*}

\endgroup

Some blibla as introduction. \autocite{machowski_2020}

% \begin{figure}
%     \centering
%     \includegraphics[width=\linewidth]{modeling/frt-type2_mv.png}
%     \caption{FRT curve; MV type 2 units}
% \end{figure}

\section{Research Interests}
\label{sec:research-interests}

Here are gaps and possible extension of knowledge.

Here are the research objectives and questions.

\commenting{
    \begin{itemize}[nosep]
        \item Influence of OLTC control on possible operational uses: Short-term voltage stability, long-term voltage stability; 
        \item Can a increased dynamic regulation help machine recovery?
        \item Does the increased tap ratio gradient harm transient stability of machines?
        Does it help or harm CCT of machines or machine groups?
        \item Transformers act as big low-pass filters: Can this behavior be beneficial as well for the interactions of inverters in the grid on AC side (in the sense of Harmonic Stability)? \quelle
    \end{itemize}
}

\begin{tcolorbox}[float, colback=ees_blue!5!white,colframe=ees_blue, toptitle=1mm, bottomtitle=1mm, left=2mm, right=2.5mm, top=2mm, bottom=2mm, title={\textbf{Research Question of this Thesis}}]
    How do different control types and characteristics of Tap Changing transformers influence the voltage stability of the given system?
\end{tcolorbox}

\sidenote{Influence of FSM on Voltage Stability}
Therefore following questions/steps can be imagined as supportive:
\begin{enumerate}
    \item How can Voltage stability of a system be classified and be looked at? Which indices, measurements, etc.
    \item Which transformer model has to be considered to show influences?
    \item Which systems are useful to consider in showing effects? Which circumstances lead to a stability support, which to a decrease? Where can limits be drawn?
\end{enumerate}

Additionally during the process of the thesis, the following question came up as an extension.
Is is the second interest of this thesis, and shall be more focused in the later part.
Therefore some assessments in the \autoref{chap:case-study} are conducted.

\begin{tcolorbox}[float, colback=ees_green!5!white,colframe=ees_green, toptitle=1mm, bottomtitle=1mm, left=2mm, right=2.5mm, top=2mm, bottom=2mm, title={\textbf{Additional Question of this Thesis}}]
    Can the already existing Tap Changer Control of the \acf{FSM} be improved towards a more operation oriented control?
\end{tcolorbox}

\sidenote{FSM Control Advancement? Operational Oriantation}
This question has following thoughs, concerning the different characteristics and dynamics of the \acs{FSM}:
\begin{enumerate}
    \item How do the different time constants of \acs{OLTC} and \acs{FSM} influence different stybility aspects in the system?
    \item Can the \acs{FSM} be used as a \glqq damping element\grqq~in the system?
    \item Does this possible different behavior of the \acs{FSM} lead to different operating strategy?
    \item What are thoughts on realizing such a strategy with different approaches on the \acs{FSM} controller?
\end{enumerate}

\section{Readers Guide}

The afore stated research interests in combination with the yet not sufficient framework to use make demands on the structure of this work. 
Therefore it seems not sufficient trying to apply a completely standard sequence of chapter like \glqq Introduction - Fundamentals - Methods - Results - Discussion\grqq.
Instead, the following structure is chosen to fulfill the research interests and to give a clear and understandable overview of the work.
This leads to the following structure for the thesis: 
\begin{itemize}
    \item \textbf{\hyperref[chap:fundamentals]{Chapter 2: Fundamentals},}\\
    is illustrating and recalling fundamentals for modeling, stability assessments, and discussions;
    \item \textbf{\hyperref[chap:methodical-modeling]{Chapter 3: Methodical Modeling},}\\
    describes the process of modeling in the tool \textit{diffpssi}, and the implementation of voltage stability indices;
    \item \textbf{\hyperref[chap:verification]{Chapter 4: Verification Setup and Result},}\\
    is showing the verification of the implemented models and tools with the help of common and simple test systems;
    \item \textbf{\hyperref[chap:case-study]{Chapter 5: Case Study},}\\
    is looking at the novel control methods from different perspectives and applications; 
    \item \textbf{\hyperref[chap:discussion]{Chapter 6: Discussion},}\\
    discusses the FSM control strategies, considering the fundamentals, verification, and case study results;
    \item \textbf{\hyperref[chap:summary]{Chapter 7: Summary},}\\
    is summarizing with regards to the research questions, and looking towards research potential and future developments. 
\end{itemize}

When reading this thesis, one might consider its motivation and its prior knowledge for allocating attention to the different chapters.
For someone interested in the strategic development of the \acs{FSM} or \acsp{OLTC} in general, the introduction with its research interests (\autoref{sec:research-interests}) are most important. 
Combined with the Summary and Outlook (\autoref{chap:summary}), the containts of the thesis, answers to the research questions and some perspectives are included.
For this level basic knowledge in power system stability and the electrical energy grid is sufficient.
When one is also interested in the explanations and thoughts why the answers to the research questions are as they are, the chapter \autoref{chap:discussion} is additionally recommended. Eventaully, the chapter \autoref{chap:fundamentals} is giving a few basics for an eased understanding of discussion itself.
The third level would be a demonstration of practical applications in the case studies (\autoref{chap:case-study}).
Lastly, if one wants to further improve or develop the tool {\itshape diffpssi} or the control strategies in particular, the chapters \autoref{chap:methodical-modeling} and \autoref{chap:verification} are recommended.
Additionally the referenced literature and the \autoref{app:power-system-modeling} are giving valuable insights and information.

% % \vspace*{6pt}

% \begin{tcolorbox}[colback=ees_blue!5!white,colframe=ees_blue, left=2mm, right=2.5mm, top=2mm, bottom=2mm]
%     \textbf{\hyperref[chap:fundamentals]{Chapter 2}}: Fundamentals
% \end{tcolorbox}

% \sidenote{Laying out the Edge of Development}
% The Fundamentals part of the thesis shall introduce the reader to the state-of-the-art connection points in standard literature and newer publications. 
% It is not starting at basic concepts in electrical or control engineering theory, but is trying to summarize the most important basics. 
% The reader shall be able to quickly adapt and think through the later applied methods, gathered results, and discussions. 
% It is further into stability topics, mathematical representation of transformers in simulations, and some informations about current applied control standards in the field of \acs{OLTC} transformers.
% \vspace*{6pt} 

% \begin{tcolorbox}[colback=ees_blue!5!white,colframe=ees_blue, left=2mm, right=2.5mm, top=2mm, bottom=2mm]
%     \textbf{\hyperref[chap:methodical-modeling]{Chapter 3}}: Methodical Modeling
% \end{tcolorbox}

% \sidenote{How are Things Done?}
% This chapter describes the advancements of the tool {\itshape diffpssi}, mainly the modeling of a ratio enabling $\Pi$-model for transformers and control circuits for their regulation. 
% Further it applies usable indices for a voltage stability assessment in the simulation environment. 
% Further, an assessment of the theoretical load-voltage boundaries is considered as well. 
% The chapter is mainly focussing on the implementation and interfaces towards the rest of the software.
% \vspace*{6pt} 

% \begin{tcolorbox}[colback=ees_blue!5!white,colframe=ees_blue, left=2mm, right=2.5mm, top=2mm, bottom=2mm]
%     \textbf{\hyperref[chap:verification]{Chapter 4}}: Verification Setup and Results
% \end{tcolorbox}

% \sidenote{Is the Tool Working?\\How Accurate is it?}
% The verification of the implemented models and tools is crucial for further studies and their validity. 
% Obtaining, discussing, and knowing the limits and boundaries of the tool is the main goal of this chapter. 
% Therefore the verification is done with the help of common and simple test systems, and the results are compared to a well known and widespread simulation tool {\itshape DIgSILENT PowerFactory} in the version 2023.
% \vspace*{6pt}

% \begin{tcolorbox}[colback=ees_blue!5!white,colframe=ees_blue, left=2mm, right=2.5mm, top=2mm, bottom=2mm]
%     \textbf{\hyperref[chap:case-study]{Chapter 5}}: Case study
% \end{tcolorbox}

% \sidenote{A Look from Different Perspectives}
% In the case study, some interesting found aspects are tried to pick up and to be further investigated. 
% The main goal is, with the background of both research questions, to show the influence of different control types and characteristics of Tap Changing transformers on the voltage stability of the given system. 
% The results are compared with the help of indices, introduced in the previous chapters, and theoretical boundaries.
% \vspace*{6pt}

% \begin{tcolorbox}[colback=ees_blue!5!white,colframe=ees_blue, left=2mm, right=2.5mm, top=2mm, bottom=2mm]
%     \textbf{\hyperref[chap:discussion]{Chapter 6}}: Discussion of the Results
% \end{tcolorbox}

% \sidenote{What Did we Learn? How can We make Use?}
% Discussing the results from the previous case studies combined with the verifacation and validation results, and the technological fundamentals rounds up the evaluation of the novel control strategies in a system perspective. 
% It shall demonstrate well suited applications, as well as current and possibly sustaining limits of the \acs{FSM}.
% \vspace*{6pt}    

% \begin{tcolorbox}[colback=ees_blue!5!white,colframe=ees_blue, left=2mm, right=2.5mm, top=2mm, bottom=2mm]
%     \textbf{\hyperref[chap:summary]{Chapter 7}}: Summary and Outlook
% \end{tcolorbox}

% The last chapter is summarizing the thesis with regards to the first initiated research questions. 
% In addition, an outlook is given in the sense of possible further research and development steps, or perspectives for application of the covered topics in this work.

%%%%%%%%%%%%%%%%%%%%%%%%%%%%%%%
%%%%%%%%%%%%%%%%%%%%%%%%%%%%%%%
\chapter{Fundamentals}
\label{chap:fundamentals}

\begin{textblock*}{.7\textwidth}(70mm-\offset,25mm-\offset)
    \begin{fquote}[Frank Zappa]
        So many books, so little time.
    \end{fquote}
\end{textblock*}

Following chapter shall introduce the basics for implementing an \acs{OLTC} equipped transformer into a existing \acs{PSS} framework.
This is considering the already existing surrounding, more detailed the electric behavior of the transformer itself and some control engineering theory for the corrosponding \acs{OLTC}. 
Thus its main goal is increasing voltage stability \autocite{machowski_2020}, main indices and assessment methods are considered as well.

%%%%%%%%%%%%%%%%%%%%%%%%%%%%%%%
\section{Voltage Stability Basics}
\label{sec:voltage-stability}

% \subsection{Voltage Stability Definitions, Classifications, and Conditions}
A Practical introduction to voltage stability assessment, methods and indices is given in the standard and extending literature of \textcite{danish_2015,cutsem_1998}. Further, some useful definitions about voltage sagging and practices are mentioned by \textcite{shoup_2004}, some other best practices, current standards, and development potential is presented by \textcite{rueda-torres_2024}.

\commenting{
    Interesting to note / implement here: Basic classification, definitions, and the nature or conditions of voltage stability. Such as
    \begin{itemize}[nosep]
        \item Short term vs. long term
        \item Static vs. dynamic
        \item Transmission driven vs. load driven vs. generation driven; stability/instability, and/or contributions 
        \item Load vs. transmission aspects
        \item \textbf{Influence OLTC:} Restoring voltage level, but not adding reactive capacities; hence adding risk of voltage collapses
        \item Example mechanism: \textbf{Collapse effect of the nordic test system} \autocite{vancutsem_2020,cutsem_1998}
    \end{itemize}
}

\commenting{
    Illustrate possible tools for Voltage Stability Analysis:
    \begin{itemize}[nosep]
        \item Nose Curves as static assessment
        \item Voltage - Power Gradients -> Some sort of Load stiffness
        \item FRT curves
        \item And the overall possibilities / goals: Increase time until collapse or problematic behavior: Other units, FACTS, generators, inverters, controllable shunts, load shedding, can begin to act, has more time etc. to impact the system and its voltage stability
        \item The lomgitudinal OLTC can't be really "supporting" in voltage support -> no change in $\tan \phi$ possible; phase shifting transformers could adress this problem
    \end{itemize} 
}
        
\begin{table}
    \centering
    \caption[Voltage instability types and different time frames]{Voltage instability types and different time frames with examples; after \quelle}
    \small
    \renewcommand\tabularxcolumn[1]{m{#1}}
    \vspace*{12pt}
    \begin{tabularx}{\textwidth}{llXX}
        % \toprule
        \textbf{No} & \textbf{Type} & \textbf{Cause of incident} & \textbf{Time frames} \\
        \toprule
        1 & Long-term & Slowly use up of reactive reserves and no outage & Several minutes to several hours \\
        2 & Classical & Key outage leads to reactive power shortage & One to five minutes \\
        3 & Short-term & Induction motor stalling leads to reactive power shortage & Five to fifteen seconds \\
        \bottomrule
    \end{tabularx}
\end{table}

An important comment has to be added towards the dynamic behaviors and the connected analysis strategy towards this.
This thesis aims to partly enlighten the dynamic influence of \acs{OLTC} control strategies on the dynamic beahavior of the, or resp. one power system.
According to many standard literautes, is is quite complicated to simplified state or predict stable or unstable operation of a power system in terms of voltage behavior \autocite{machowski_2020}.
Simply the complex interaction between different units, their control schemes and characteristics, and at least the characteristics of protectional devices, are too many influences.
As this thesis is only considering a few aspects of the previous mentioned, only a relative comparison can be targeted.
The use of one stability index for this analysis or predictions is highly unlikely.
Therefore mixtures of possible illustrations, analysis techniques or similar are combined, enabling a discussion about the thesis scope, transformers and their tap changer control schemes.  
        
\subsection{Analytical Stability Calculation of Simple Static Power Systems}
\label{sec:analytical-voltage-stability}

When looking at a simple power system, consistent of a load and a source, analytically deriving the behavior of voltage over power is in its simplest form.
The resulting curves show how the system behaves in (quasi-) stationary scenarios.
Following a system like \autoref{fig:v-stability-system} is used. 
All follwing equations and analysis methods can be re read in standard literatur like \textcite{machowski_2020}, \textcite{kundur_2022}, or \autocite{cutsem_1998}. 

\begin{figure}[htbp!]
    \centering
    % \missingfigure{Simple system from standard literature}
    \includegraphics[width=\linewidth]{fundamentals/Simple_Load.png}
    \caption[Simplest Load - Source system for deriving voltage - power behaviors]{Simplest Load - Source system for deriving voltage - power behaviors; own illustration after \autocite{machowski_2020,kundur_2022,milano_2010}}
    \label{fig:v-stability-system}
\end{figure}

When looking at the load flow equations, the transferable power over the system from bus one to bus two can be represented by
\begin{align}
    S&=P + jQ = \underline{V} \cdot \underline{I}^* \notag \\[12pt]
    &=\underline{V} \cdot \frac{\underline{E}^* - \underline{V}^*}{-jX} \notag \\[6pt]
    &=\frac{j}{X} (EV \cos \phi + jEV \sin \phi -V^2). \notag
\end{align}
These equations can be split up into the transferable real power in \autoref{eq:func-p-transfer} and the transferable reactive power in \autoref{eq:func-q-transfer}, which might be more common in knowledge and use.
\begin{align}
    P&=-\frac{EV}{X} \cdot \sin \phi \label{eq:func-p-transfer} \\[6pt]
    Q&=-\frac{V^2}{X} + \frac{EV}{X} \cdot \cos \phi \label{eq:func-q-transfer}
\end{align}
After elimination of $\phi$, a second order equation dependent on $V^2$ can be obtained.
Simplifying the terms and rearanging is giving the follwing \autoref{eq:transfer-func}.
\begin{align}
    &-P^2 - \frac{E^2}{X}Q + \bigg(\frac{E^2}{2X}\bigg)^2 \geq 0 \label{eq:transfer-func} \\[12pt]
    &P \leq \frac{E^2}{2X} \quad\text{for}\quad Q=0 \label{eq:simplified-transfer-p} \\[6pt]
    &Q \leq \frac{E^2}{4X} \quad\text{for}\quad P=0 \label{eq:simplified-transfer-q}
\end{align}
The easy intuitively accessed functions, which can be kept in mind are \autoref{eq:simplified-transfer-p} and \autoref{eq:simplified-transfer-q}.
These represent the functions of the voltage dependent on the real power, when setting the reacitve power to zero, and vice versa. 
This accounts for power factors of $\cos \phi = 0$ or respectively $\sin \phi = 0$, or translated to the angle itself $\phi = \{0^\circ; 90^\circ\}$.
\footnote{for all angles in the interval of $\phi = [0^\circ, 180^\circ]$}
One note to take here, is that the power factor is not consistently used.
For load flow calculations mostly the sine and cosine representation of the angle between current and voltage is used, for stability analysis, often the tangent function is preferred.
A table and plot comparison of relations between the functions $\cos$, $\sin$, and $\tan$ are included in the appendix \autoref{app:trogonometric-func-comp}. 
This also leads to the dependency of $\tan \phi$ in the plotting of the so called Nose Curves, the result of when one is plotting the solutions of \autoref{eq:transfer-func}.

\begin{figure}[htbp!]
    \centering
    % \missingfigure{Nose Curves P and Q dependent on V}
    \begin{subfigure}[b]{.49\linewidth}
        \includegraphics[width=\linewidth]{fundamentals/p_v_curve.png}
        \subcaption{P-V \glqq Nose Curve\grqq}
    \end{subfigure}
    \begin{subfigure}[b]{.49\linewidth}
        \includegraphics[width=\linewidth]{fundamentals/v_q_curve.png}
        \subcaption{V-Q Curve}
    \end{subfigure}
    \caption[Nose Curves resulting from maximum power transfer equations]{Nose Curves resulting from maximum power transfer equations; own illustration after \autocite{machowski_2020,kundur_2022,cutsem_1998}}
    \label{fig:v-stability-system}
\end{figure}

\commenting{Describe the Nose Curves here briefly: What can one see and extract from that?}
The typical nose shape is visible for a simple network, with only considering reactances $X$ as attributes of the line or transportation grid.
Directly visible, that no static solution can be found, when surpassing a certain level of real power. 
This maximum often is referred to as the power transfer limit of the grid or network.


\commenting{
    Comment on $X$: If completely reactive then Nose curve shape, if real part in there, then form is drastically changing.
    There is a more complex illustration in 3D with axis P-Q-V.
}

% \subsection{Stability Limits of Complexer Systems: Continuation Power Flow}
% \label{sec:continuation-power-flow}

\subsection{Stability Indices for Time Series Calculation}
\label{sec:stability-indices}

\sidenote{Basic idea\\and references}
The idea behind stability indices is monitoring the current voltage stability state of the power system in relation to the critical point.
This applies not only to static load flow cases, but for time series calculations of short circuits, load shedding or other disturbances.
Therefore the dynamic contribution can be included as well. \quelle

Reviewing possible indices, either for online resp. real time monitoring or for subsequent analysis after a simulation, is out of the scope of this thesis.
\textcite{danish_2015} has already included an extensive review in his work, with reference to \textcite{doigcardet_2010}, which is focussing on indices in interest for this thesis in particular. 
\commenting{Jacobian Matrix based interesting, because of the possible observability in \textit{diffpssi}.}

\sidenote{The basics of the Jacobian Matrix}
One easy idea for obtaining a stable operation is looking at the Jacobian Matrix. 
Although it is mainly used in load flow calculations, it is possible to derive statements about the voltage stability as well.
This can be realized through the relation of real power $P$, reactive power $Q$, voltage angle $\delta$, and the voltage resp. voltage gradients $\Delta V$. 
If the Jacobian Matrix is getting singular, the System will not remain in a stable operation.
Singularity of matrices is checked by following two hypothesis tests:
\begin{align}
    &\det(\mab{J})=0 \label{eq:jacobian-det} \\[12pt]
    &J \times J^{-1} \uparrow \label{eq:jacobian-rank}
\end{align}
The Jacobian Matrix is defined as:
% \sidenote(5.2cm){Jacobian Matrix}
\begin{align}
    \mab{J}=
    \begin{bmatrix}
        \Delta \mab{P} \\
        \Delta \mab{Q}
    \end{bmatrix}&=
    \begin{bNiceArray}{c|c}
        \mab{H} & \mab{M'} \\ \hline
        \mab{N} & \mab{K'}
    \end{bNiceArray} \cdot
    \begin{bmatrix}
        \Delta \delta \\
        \Delta V/V
    \end{bmatrix} \notag \\[12pt]
    \begin{bmatrix}
        \Delta P_1 \\
        \vdots \\
        \Delta P_n \\ \hline
        \Delta Q_1 \\
        \vdots \\
        \Delta Q_n
    \end{bmatrix}&=
    \begin{bNiceArray}{ccc|ccc}
        \frac{\partial P_1}{\partial \delta_1} & \hdots & \frac{\partial P_1}{\partial \delta_n} & V_1\frac{\partial P_1}{\partial V_1} & \hdots & V_n\frac{\partial P_1}{\partial V_n} \\
        \vdots & \ddots & \vdots & \vdots & \ddots & \vdots \\
        \frac{\partial P_n}{\partial \delta_1} & \hdots & \frac{\partial P_n}{\partial \delta_n} & V_1\frac{\partial P_n}{\partial V_1} & \hdots & V_n\frac{\partial P_n}{\partial V_n} \\ \hline
        \frac{\partial Q_1}{\partial \delta_1} & \hdots & \frac{\partial Q_1}{\partial \delta_n} & V_1\frac{\partial Q_1}{\partial V_1} & \hdots & V_n\frac{\partial Q_1}{\partial V_n} \\
        \vdots & \ddots & \vdots & \vdots & \ddots & \vdots \\
        \frac{\partial Q_n}{\partial \delta_1} & \hdots & \frac{\partial Q_n}{\partial \delta_n} & V_1\frac{\partial Q_n}{\partial V_1} & \hdots & V_n\frac{\partial Q_n}{\partial V_n} \\
    \end{bNiceArray} \cdot
    \begin{bmatrix}
        \Delta \delta_1 \\
        \vdots \\
        \Delta \delta_n \\ \hline
        \Delta V_1/V_1 \\
        \vdots \\
        \Delta V_n/V_n
    \end{bmatrix}
\end{align}

Although this method seems easy to implement, there are some numerical problems realted to that.
Checking if a Matrix is singular with numerical methods, can only be realised as a probablilty expression. 
A result could be, that the determinant of the matrix is below a certain threshold. 
The algorithm would propose, that the matrix is probabilistic singular. \quelle

\sidenote{Indices adressing numerical problems}
This problem leads to the necessity of applying other methods or indices for stability assessment. 
Further, the relation of the system state to the critical voltage collapse point is highly nonlinear in the Jacobian Matrix. 
This problem is adressed by the indices in various ways, leading to a more or less linearized relation. 
\quelle \textcite{danish_2015} is proposing a few other indices, that are based on the Jacobian Matrix, and shows comparitive characteristics between Jacobian Matrix and system variable based voltage stability indices. 
These Jacobian Matrix based indices are listed and further described in \autoref{app:jacobian-voltage-indices}, while the comparative characteristics are described in \autoref{app:jacobian-vs-system-indices}. 
The aforeside mentioned work of \textcite{doigcardet_2010} is focussing on these indices in particular.

For this thesis, the indices in \autoref{tab:considered-indices} are chosen on the basis of the work of \textcite{danish_2015} and \textcite{doigcardet_2010}. 
They will be implemented in the Python framework in \autoref{sec:application-voltage-stability}, and used for with the theoretical calculated limits of \autoref{sec:analytical-voltage-stability} and \autoref{sec:continuation-power-flow}.

\begin{table}[htb!]
    \centering
    \caption[Voltage stability indices for the assessment in this thesis]{Voltage stability indices for the assessment of the power system stability in this thesis; after \quelle}
    \label{tab:considered-indices}
    \vspace{12pt}
    \begin{tabularx}{\textwidth}{llX}
        % \toprule
        \textbf{Index} & \textbf{Basis} & \textbf{Description} \\
        \toprule
        VCPI & Jacobian Matrix & The index is based on the Jacobian Matrix, and is showing the proximity of the current system state to the critical voltage collapse point. \\
        VSI & System Variables & The index is based on the system variables, and is showing the proximity of the current system state to the critical voltage collapse point. \\
        \bottomrule
    \end{tabularx}
\end{table}

\section{Voltage Angle Stability of Machines}

\commenting{
    Just short and briefly Describe
    \begin{itemize}[nosep]
        \item What is Voltage Angle Stability?
        \item Why relevant to this thesis?
        \item How to evaluate: EAC and CCT of a generator
        \item Maybe something like a max. utilization of max power angle?
    \end{itemize}
}

%%%%%%%%%%%%%%%%%%%%%%%%%%%%%%%
\section{Power System Modeling}

The simulation of power systems is a crucial tool, not only for stability studies, but for evaluating extensions or modifications in the planning process, the development of Assistance systems for operational management, and many other applications. \quelle 
Due to the complexity of the systems, simulations are often simplified. 
Not only with model constraints, bus as well in the way of calculations. 
Mainly seperating between \acf{EMT} and \acf{RMS} simulations, the latter is used in this thesis. 
The \autoref{app:power-system-modeling} is giving more details, literature, and summarizing the processes behind the used tool \textit{diffpssi}, if the reader is further interested in the topic.

\subsection{Transformer Electric Model and Behavior}
\label{sec:trafo-model}

% \begin{figure}[htb!]
%     \centering
%     \includegraphics[width=.8\textwidth]{fundamentals/trafo_model_deviation.png}
%     \caption[Two-Winding Transformer Circuit in the Positive Sequence]{Two-Winding Transformer Circuit in the Positive Sequence; a) ideal representation with impedances on each \acs{HV} and \acs{LV} side and b) related impedances on the XX side; own figure after \autocite{machowski_2020}}
%     \label{fig:trafo-model-deviation}
% \end{figure}

% \sidenote{Basics}
Typical for \acs{RMS}-modeling is the usage of sequence components, especially the positive sequence for symmetrical grid operation and test cases. \quelle 
An equivalent circuit for the positive sequence is shown in \autoref{fig:trafo-model} part a), respectively reduced to the transformer ratio, the series impedances of the windings on the \acs{LV} and \acs{HV} side, and the shunt branch affected by iron and magnetization losses. \autocite{machowski_2020,kundur_2022,milano_2010}

The transformer ratio is typically noted as $\underline{\vartheta}$. 
Generally speaking it is the ratio between the number of windings of the secondary side to the primary side, as noted in \autoref{eq:trafo-ratio-easy}. 
With the typically used calculation unit \glqq per unit\grqq\footnote{means standardization to a reference value; further information on page \pageref{chap:symbols} and \textcite{machowski_2020}, Appendix A}, the ratio becomes one in the standard case. 
A transformer ratio, which is only shifting current angles with the shifting angle $\phi$, is represented through a complex number using the Euler Identity, as shown in \autoref{eq:trafo-ratio}.
\begin{align}
    \vartheta&=\frac{N_2}{N_1} \label{eq:trafo-ratio-easy} \\[6pt]
    \underline{\vartheta}&=\frac{N_2}{N_1} \cdot \exp(j \cdot \phi \cdot \frac{\pi}{180})\label{eq:trafo-ratio}
\end{align}

\begin{figure}% [htb!]
    \centering
    % \captionsetup[subfigure]{justification=centering} 
    \begin{subfigure}[c]{.53\textwidth}
        \centering
        \includegraphics[width=\linewidth]{tikz_graphics/images/transformer_complete.pdf}
        \subcaption{Ideal Representation}
    \end{subfigure}
    \begin{subfigure}[c]{.46\textwidth}
        \centering
        \includegraphics[width=\linewidth]{tikz_graphics/images/transformer_reduced.pdf}
        \subcaption{Simplifyied Representation}
    \end{subfigure}
    % \includegraphics[width=.8\textwidth]{fundamentals/trafo_model_deviation.png}
    \caption[Two-Winding Transformer Circuit in the Positive Sequence]{Two-Winding Transformer Circuit in the Positive Sequence; a) ideal representation with impedances on the \acs{HV} side and b) simplifyied circit with only the series impedance related on the \acs{HV} side; own figure after \autocite{machowski_2020,kundur_2022,milano_2010}}
    \label{fig:trafo-model}
\end{figure}

\sidenote{Simplifications}
The first simplification is step is considering two assumptions. 
First, the iron and magnetization losses are neglectable. 
This can be illustrated with a short-circuit test of the transformer on the secondary side. 
During this test, one can obtain with the concept of a voltage devider, that
\begin{align}
    \underline{U}_\mathrm{Fe, \mu} \ll \underline{U}_\mathrm{T,rated}, \notag
\end{align}
meaning that the shunt branch impedance is much greater that the series impedance of the transformer. 
Secondly, it is assumed, that the on the primary side related impedance of the secondary side, is equal to the impedance on the primary side. 
This leads to a symmetrical circuit of the transformer and the positive sequence equivalent circuit simplifies to \autoref{fig:trafo-model} part b). 
Mathematically this is shortly expressable as \autoref{eq:related-impedances}, \autoref{eq:trafo-symmetrical}, and \autoref{eq:series-impedance}. \autocite{machowski_2020,kundur_2022,milano_2010}
\begin{align}
    \underline{Z}_1 &= R_1 + jX_1\text{;}\quad\underline{Z}_2 = R_2 \vartheta^2 + jX_2 \vartheta^2 \label{eq:related-impedances} \\
    \underline{Z}_1 &= \underline{Z}_2 \label{eq:trafo-symmetrical} \\
    \underline{Z}_\mathrm{T} &= \underline{Z}_1 + \underline{Z}_2 \label{eq:series-impedance}
\end{align}
The afore described simplification leads to only the necessity of considering the series impedance. 
Considering the afore mentioned normal ratio of $\vartheta=1$ in the per unit system, the Python framework \textit{diffpssi} has been using this model with only the series impedance and no variable ratio, meaning no shunt branches, before.

\sidenote{Introducing variable transformer ratios}
When one wants to look at variable transformer ratios, either with representing vector groups, or implementing \acfp{OLTC}, this model of only considering the series impedance has to be extended. 
Using shunt branches, the variable ratio behavior can be represented in a $\Pi$-model, as shown in \autoref{fig:pi-transformer}. \autocite{machowski_2020,kundur_2022,milano_2010}

\begin{figure}%[htb!]
    \centering
    \includegraphics[width=.7\textwidth]{tikz_graphics/images/transformer_pi.pdf}
    \caption[$\Pi$-representative circuit of an idealized transformer with a tap changer]{$\Pi$-representative circuit of an idealized transformer with a tap changer; own figure after \autocite{milano_2010,burlakin_2024}}
    \label{fig:pi-transformer}
\end{figure}

Looking at the transformer as a black box two-port, with the index one being the \acs{LV} side, the index two being the \acs{HV} side, the admittance matrix for the variable ratio behavior can be expressed as in \autoref{eq:admittance-behavior}. 
The voltages and current are defined as in \autoref{fig:trafo-model} part b). 
With rearranging the equation, one can obtain the admittance matrix of the $\Pi$-model with to the \acs{HV} side related values as in \autoref{eq:admittance-matrix-pi}. \autocite{milano_2010,burlakin_2024}

\begin{align}
    \begin{bmatrix}
        {\color{ees_green}\underline{\vartheta}^*} \underline{I}_1 \\ \underline{I}_2
    \end{bmatrix}&= 
    \begin{bmatrix}
        \underline{Y}_\mathrm{T} & -\underline{Y}_\mathrm{T} \\
        -\underline{Y}_\mathrm{T} & \underline{Y}_\mathrm{T}
    \end{bmatrix} \cdot
    \begin{bmatrix}
        {\color{ees_green}\frac{1}{\underline{\vartheta}}}~\underline{U}_1 \\ \underline{U}_2
    \end{bmatrix} \label{eq:admittance-behavior} \\[12pt]
    \underline{\mab{Y}}_\mathrm{\Pi,T}&=\underline{Y}_\mathrm{T} \cdot
    \begin{bmatrix}
        \frac{1}{\underline{\vartheta}\underline{\vartheta}^*} & -\frac{1}{\underline{\vartheta}^*} \\
        -\frac{1}{\underline{\vartheta}} & 1
    \end{bmatrix} \label{eq:admittance-matrix-pi}
\end{align}

For calculation of the individual shunt branches, one can apply the standard representation of two-ports consistent of a linear $\Pi$-circuit:
\begin{align}
    \begin{bmatrix}
        \underline{I}_1 \\ \underline{I}_2
    \end{bmatrix}=
    \begin{bmatrix}
        \underline{Y}_{10} + \underline{Y}_{12}& -\underline{Y}_{12} \\
        -\underline{Y}_{21} & \underline{Y}_{20} + \underline{Y}_{21}
    \end{bmatrix} \cdot
    \begin{bmatrix}
        \underline{U}_1 \\ \underline{U}_2
    \end{bmatrix} \notag % \label{eq:shunt-calc}
\end{align}
When equating this with \autoref{eq:admittance-matrix-pi}, the shunt branches can be calculated respectivly giving the admittances written down as \autoref{eq:y-12}, \autoref{eq:y-10}, and \autoref{eq:y-20}, as they are noted in \autoref{fig:pi-transformer} as well. \autocite{milano_2010,burlakin_2024}
\begin{align}
    \underline{Y}_{12}&=\frac{1}{\underline{\vartheta} \cdot \underline{a}_\mathrm{T}^*} \cdot \underline{Y}_\mathrm{T}\text{, and} \notag \\[6pt]
    \underline{Y}_{21}&=\frac{1}{\underline{\vartheta} \cdot \underline{a}_\mathrm{T}} \cdot \underline{Y}_\mathrm{T} \label{eq:y-12} \\[12pt]
    \underline{Y}_{10}&=\frac{1}{\underline{\vartheta}} \cdot \bigg(\frac{1}{\underline{\vartheta}} - \frac{1}{\underline{a}_\mathrm{T}}\bigg) \cdot \underline{Y}_\mathrm{T} \label{eq:y-10} \\[12pt]
    \underline{Y}_{20}&=\bigg(1 - \frac{1}{\underline{\vartheta} \cdot \underline{a}_\mathrm{T}}\bigg) \cdot \underline{Y}_\mathrm{T} \label{eq:y-20}
\end{align}

% After MACHOWSKI:
% \begin{align}
%     \underline{\mab{Y}}_\mathrm{\Pi,T}&= 
%     \begin{bmatrix}
%         \underline{Y}_\mathrm{T} & -\underline{\vartheta}\underline{Y}_\mathrm{T} \\
%         \underline{\vartheta}^*\underline{Y}_\mathrm{T} & -\underline{\vartheta}^*\underline{\vartheta}\underline{Y}_\mathrm{T}
%     \end{bmatrix} \label{eq:admittance-oltc}
% \end{align}
% Another way of writing down the admittance matrix is shown in \autoref{eq:admittance-oltc-2}. It is considering, that the matrix can be split up in a symmetric, constant part, and a variable current injection part. The latter is not symmetrical and depends on the tap position of the transformer. Therefore in some simulation algorithms the static part is used in the admittance matrix, and the variable part is considered in the current injection vector. \autocite{machowski_2020}



\sidenote{Per unit system\\specialities}
Reactances and resistances are referred to the base voltage and apparent power of the operational unit, such as the transformer. 
The power system simulation uses its own base voltage and base apparent power, enabling the use of one single calculation domain. 
This is done to simplify the calculation and to make the results easily comparable to each other. 
Hence, the reffered values have to be transformed from the equipment based values to the simulation based values. 
The relation for the transformer admittance is defined as follows. 
Generally speaking, this thesis is using and reffering to the per unit based values, although it is not denoted in the index of the values.
\begin{align}
    \underline{Y}_\mathrm{T,~based}&=\underline{Y}_\mathrm{T} \cdot \frac{S_\mathrm{n}}{S_\mathrm{n,~sim}} \label{eq:y-t-based} \\[6pt]
    \underline{Z}_\mathrm{line,~based}&=\underline{Z}_\mathrm{line} \cdot \frac{S_\mathrm{n,~sim}}{V_\mathrm{n,~sim}^2} \label{eq:y-line-based}
\end{align}
Displayed like in \autoref{eq:y-t-based}, the characteristic of the operational unit is referred to the simulation base value. 
Here, the admittance of the transformer is multiplied with its own rated apparent power, then devided by the apparent power of the simulation system. 
Similar, the impedance of the lines are calculated via \autoref{eq:y-line-based}. 
This specialities are considered in the tap changer modeling, thus further information is given in \autocite{machowski_2020}, Appendix A.

\subsection{Further Considerations of a Transformer Model}
\label{sec:further-considerations}

\commenting{Describe here the asymetric and non-idealized transformer; phase shifting transformers; transformers, which can control logitudinal and transversal ratios; ...}

\sidenote{Ideal vs. asymmetric ratio}
\commenting{Just briefly desribe the influence of an asymetric vs. an ideal transformer. Why the difference and how the representation could work.}
\begin{figure}[htb!]
        \centering
        \includegraphics[width=.7\linewidth]{images/modeling/asymetric_ratio_vector.png}
        \caption[Illustration of the tap ratio vector for an ideal and an asymmetric transformer]{Illustration of the tap ratio vector for an ideal and an asymmetric transformer; from the \textcolor{ees_red}{\textbf{DIgSILENT Technical Reference Manual}} \dots \quelle}
        \label{fig:asymetric-ratio-vector}
\end{figure}

\subsection{Open-Source Power System Simulation tools}
\label{sec:simulation-tools}

For this thesis relevant are mainly two Power System Simulation tools.
The first and main objective, \textit{diffpssi} is an open source project from Georg Kordowich, member of the chair hosting this thesis.
This open source tool provides a \acs{RMS} domain simulation of Power Systems, considering passive grid elements like lines, shunts, loads etc., but dynamic and controlled models like Synchronous Machines, Inverters or similar as well.
The project and the integration of a full Synchronous Machine model is presented in the published paper of \textcite{kordowich_2023}.
The project is based on little dependencies like numpy, but can as well use the Python module \textit{PyTorch} as backend.
\textit{PyTorch} allows for differentiation of simulation variables.
With this, optimzation of grid and machine parameters is possible.
Future extension and usage in the topics like voltage stability and resp. or control strategies is likely as well.

The second relevant tool is a commercial software called \textit{PowerFactory} by the company \textit{DIgSILENT}.
Modeling of the power system elements is described in its technical references, but not open source like \textit{diffpssi}.
\commenting{Some blibla about PF...}

To give an overview of other open source power system simulation projects, this section shall be used as well.

    
\commenting{
    Some information about other open source python power system simulation tools, such as:
    \begin{itemize}[nosep]
        \item Pandapower,
        \item TOPS,
        \item ... .
    \end{itemize}
    Maybe relate to \href{https://github.com/ps-wiki/best-of-ps}{this GitHub Repo}, comparing different Open-Source Power System Simulation tools. See at source \autocite{jinningwang_2025}.
    Build up like a scan (see Georg's thesis).
}
        
%%%%%%%%%%%%%%%%%%%%%%%%%%%%%%%
\section{On-Load Tap Changer Controls}

\subsection{Commonly Used On-Load Tap Changer Control}

A few basics are in the interest, understanding differences between real world beahavior, or possible ways of building up a \acs{OLTC} transformer control. 
This control theory difference can be limiting as well for the results and objectives compared to the actual possible control in the field.

% \subsubsection{Typical presets are manually set}
\sidenote{Typical presets are manually set}
The target voltage is typically set from the control room of the grid operator, coming from pre-calculated load flow analysis. 
This can be set hours before, or even day-ahead with the estimated loads of the grid. 
This value is set locally for each operating unit subsequently. 
The control is then operating locally and without further involvement of the grid operator. \quelle

% \subsubsection{Discrete controllers are used in the field}
\sidenote{Discrete controllers are used in the field}
Typically the used controller in the field is a discrete controller, which can change tap positions under load within a time frame of around few seconds. 
Practical tap steps are around $2~\mathrm{\%}$ of the overall transforming ratio. 
The control is set up with a dead band, to avoid unnecessary tap changes. 
It is necessecary to note here, that this control and its mathematical caracteristics contains logical elements, blocks, and delays, which cannot be translated in a typical control theory transmission function. 
This leads to the missing possibility to easily obtain mathematical stability for the control of the overall considered power system. \quelle

\subsection{Advancement: Fast Switching Module and its Control}

\commenting{
    Describe the findings of Ilya:
    \begin{enumerate}[nosep]
        \item Approach and technological (short and not too detailed),
        \item Control proposal of Ilya,
        \item (Hardware) Development potential.
\end{enumerate}
}


%%%%%%%%%%%%%%%%%%%%%%%%%%%%%%%
\section{Summary in Short and Simple Terms}

\commenting{
    Shortly summarize the bullet points / key takeaways from the topics
    \begin{itemize}
        \item Voltage Stability - Static and dynamic
        \item Power System Modeling - especially Transformer Modeling
        \item New Advancement from Ilya
    \end{itemize}
}
%!TEX root = ../main.tex

%%%%%%%%%%%%%%%%%%%%%%%%%%%%%%%
%%%%%%%%%%%%%%%%%%%%%%%%%%%%%%%
\chapter{Methodical Modeling}
\label{chap:methodical-modeling}

\begin{textblock*}{.7\textwidth}(70mm-\offset,25mm-\offset)
        \begin{fquote}[Albert Einstein]
            All models are wrong, but some are useful.
        \end{fquote}
\end{textblock*}

This chapter mehodical modeling focusses on the description of thoughs and structures of the implementation in Python.
It is not evolving more than necessary details about the package {\itshape diffpssi}, but trying to comprehensible illustrate the structure of the algorithms theirselves and the necessaary bordering interfaces.

%%%%%%%%%%%%%%%%%%%%%%%%%%%%%%%
%%%%%%%%%%%%%%%%%%%%%%%%%%%%%%%
\section{Transformer Equipment Modeling}
\label{sec:transformer-modeling}

This section respectively focusses on the dynamics and model behavior of the transformer itself.
It is split according to the structure of the implementation itself, into the modeling of the $\Pi$-model and the tap changer control.
For the last mentioned, there are different control schemes implemented and thus described in the subsequent section.

%%%%%%%%%%%%%%%%%%%%%%%%%%%%%%%
\subsection{Implementing a $\Pi$-Representative Circuit with Variable Ratio}

Before detailing in the software side of the implemetation, some mathematical differences are explained.
This results on the one hand from the major differences in the standard literature, especially between \textcite{machowski_2020} and \textcite{kundur_2022}, resp. \textcite{milano_2010}.
On the other hand, these differences occur as well in the comparative and validation simulation software \textit{DIgSILENT PowerFactory}.
The use of these different models is described in its technical reference manual \quelle. 

\newpage
\subsubsection{Mathematical Description and Definitions}

\sidenote{Important definitions and literature differences}
Firstly it is important to comment on the use of indices in this thesis, and especially following for this chapter.
The index 1 is always referring to the \acs{LV} side, the index 2 to the \acs{HV} side. 
The impedances can be concentrated and related to either the \acs{LV}, or as usual to the \acs{HV} side of the transformer. 
The in \autoref{sec:trafo-model} used derivation is using a relation on the \acs{HV} side.
The same accounts for the definition of the \acs{OLTC} ratio $\underline{\vartheta}$.     
The \acs{OLTC} ratio $\underline{\vartheta}$ in this thesis is always placed on the HV side.
% If one wants to place this ratio on the \acs{LV} side, the in this thesis defined ratio has to be used reciprocal.
% For the simulation tool, this is crucial to understand and define correctly in order to acquire correct results.

\sidenote{Definition OLTC ratio}
This thesis focusses on an ideal tap changer model at first, other possible considerations from \autoref{sec:further-considerations} are neglected.
As vector groups are as well not considered, the tap ratio stays solely a rational number.
Like previously mentioned, and consequently described, the ratio $\vartheta$ is then pülaced on the \acs{HV} side of the transformer.
The \acs{OLTC} ratio $\vartheta$ is then defined as:
\begin{align}
        \vartheta_\mathrm{HV} &= 1 + k \cdot \Delta v \label{eq:tap-ratio-hv} \\[6pt]
        \text{with } k &\in [k_\mathrm{min};k_\mathrm{max}]; k_\mathrm{min} \equiv  -k_\mathrm{max} \label{eq:tap-pos}
\end{align}
Within this definition, $k_\mathrm{min}$ defines the minimum tap position, $k_\mathrm{max}$ the maximum \acs{OLTC} position. 
The variable $\Delta v$ defines the change of the ratio in percent for alterning one position.

% \sidenote{Representation of\\Vector Groups}
% Voltage angle shifting through the influence of vector groups, meaning a different wiring and thus magnetic coupling of the transformer can be expressed within the transformer model. 
% By mathematically applying a turning vector with the length of one to the overall tap ratio, this can be included in the model. 
% Mathematical, this is expressed by the following equation. 
% The characteristic number $n_\mathrm{T}$ is relating to to angle, with one step being equal to $30^\circ$ angle ratio.
% \begin{align}
%         \underline{a}_\mathrm{T} &= \exp(j \cdot n_\mathrm{T} \cdot \frac{\pi}{6}) \label{eq:vector-group}
% \end{align}

\subsubsection{Mathematical Different Representations}
\sidenote{Relation to the LV side}
When one either wants to relate the transformer admittance, or the tap ratio to the \acs{LV} side, a different admittance matrix definition has to be used.
The admittance matrix is then defined as:
\begin{align}
        \underline{\mab{Y}}_\mathrm{\Pi,T}&= 
        \begin{bmatrix}
            \underline{Y}_\mathrm{T} & -\underline{\vartheta}\underline{Y}_\mathrm{T} \\
            \underline{\vartheta}^*\underline{Y}_\mathrm{T} & -\underline{\vartheta}^*\underline{\vartheta}\underline{Y}_\mathrm{T}
        \end{bmatrix} \label{eq:admittance-oltc}
    \end{align}
The following mathematical result leads to a necessary change in the software implementation.
Either
\begin{itemize}
        \item the admittance matrix bus indices have to be changed,
        \item the tap ratio has to be reciprocal according to \autoref{eq:tap-ratio-lv}, or
        \item using the \acs{HV} side admittance matrix, but changing the tap ratio definition and the bus indices.
\end{itemize}
These different ways of variable and placing definitions also characterize the ways, the admittance matrix of the \acs{OLTC} transformer is derived from either \textcite{machowski_2020}, versus \textcite{kundur_2022}, \textcite{milano_2010}, or \textcite{burlakin_2024}.
Another thought or way of representing a transformer with off-nominal ratio is described in the appended \autoref{app:current-injection-model}.
\begin{align}
        \vartheta_\mathrm{LV} &= \frac{1}{1 + k \cdot \Delta v} \label{eq:tap-ratio-lv}
\end{align}

\subsubsection{Thoughts, Design, and Implemetation of Algorithmics}

\commenting{
        Base idea here: 
        \begin{itemize}[nosep]
                \item Show the thought process, design sketches and the implemetation algorithmics in the Python framework.
                \item Add a class diagram of the transformer model, with all needed interface / logable data, interface methods, and the abstract design.
                \item Describe addtional methods.
                \item Show the algorithmic implemetation logic of the Pi model, but not the Tap Changer, as this is seperated. 
        \end{itemize}
        Additionally interesting extensions:
        \begin{itemize}[nosep]
                \item How to automatically determine switching direction?\\
                -> switchin direction dependent on what? (load-flow direction?)
                \item Controller set points: also dependent on load flow?
                \item How can I change the transformer control setpoints to be load flow dependent?
                \item How can I ensure, utilization of the transformer is not $>S_\mathrm{n}$?
        \end{itemize}
}

%%%%%%%%%%%%%%%%%%%%%%%%%%%%%%%
\subsection{Tap Changer Control Modeling}

\commenting{This is the description of the ideas, development, and implementation of a OLTC control scheme.}

\subsubsection{Software Architecture Design}

The first scope of the \textit{diffpssi} extension is to form a modular and easy to maintain class structure. 
The background is to enable support of adding other types of transformers and resp. or connected control circuits.
A conceptual chart of this architecture is shown in \autoref{fig:transformer-architecture}.
It is representing only necessary classes or packages for the transformer and its control.

\begin{figure}[htbp!]
        \centering
        \includegraphics[angle=90, width=\linewidth]{modeling/diffpssi_trafo_architecture.png}
        \caption[Architecture of the implemented models in \textit{diffpssi}]{Architecture of the implemented models in \textit{diffpssi}; Using abstract classes for correct interfaces and improved reusability; only necessary packages, modules and classes are depicted}
        \label{fig:transformer-architecture}
\end{figure}

\subsubsection{Discrete Control Loop}
\commenting{
        \begin{itemize}[nosep]
                \item Describe implementation
                \item Describe benefits / drawbacks
                \item Control scheme
                \item Switching logic and behavior (voltage tracking)
        \end{itemize}
}

\sidenote{General aspects\\and references}
This control method represents the currently most used and thus representative control scheme for \acsp{OLTC}. 
With the mechanic nature of the switching mechanism, the control look can only access discrete ratios within time frames of around a few seconds. 
Such a discrete control loop is described by \textcite{milano_2011,milano_2010}. 
A scheme of this control loop is shown in \autoref{fig:discrete-control-loop}.

\begin{figure}[htb!]
        \centering
        \missingfigure{Discrete control loop}
        \caption{Discrete control loop of an \acs{OLTC}; scheme based on \textcite{milano_2011}}
        \label{fig:discrete-control-loop}
\end{figure}

This control loop type is beneficial due to its accurate representability of current \acs{OLTC} abilities. 
It gains access to assess stability within simulation environments, as analytical methods are not suited.

A negative aspect of a discrete control loop is the missing opportunity of generating a transfer function. 
This blocks the stability assessment with standard control engineering methods. 
Further, popular analysis methods like eigenvalue analysis is not possible, due to the lack of possibility to form derivatives.

% \lstinputlisting[caption={Output Function of the discrete OLTC controller},captionpos=b,style=style-python,label=lst:oltc-discrete]{images/code/oltc-discrete.py}

\sidenote{Implementation\\Structure}
\commenting{The structure of the implementation is illustrated in the block diagram of \autoref{fig:discrete-oltc-implementation}. 
The controller is actively chainging the algebraic funtions of the simulation environment, therefore it is quasi dynamic. 
The controller output logic is called, when updating the admittance matrix of the transformer. 
Additionally, the differential functions of the connected simple controllers, like integrators, PT1-blocks, etc., are called by the solver and are thus part of the differential equations. 
The logic determines the physical interpretation of the \acs{OLTC}, and therefore
\begin{enumerate}
        \item If the OLTC has to switch,
        \item When the switching operation is finished, and
        \item What the current, or in case after a switching the new, tap ratio is.
\end{enumerate}
It is important to note, that this structure relies on the calculation of the dynamic admittance matrix on each time step.}

\begin{figure}[htb!]
        \centering
        \missingfigure{Implementation structure of the discrete OLTC controller}
        \caption[Implementation structure of the discrete OLTC controller]{Implementation structure of the discrete OLTC controller}
        \label{fig:discrete-oltc-implementation}
\end{figure}

\sidenote{Output logic}
The output of the controller is based on the following logic.

\sidenote{Characterization plot and validation}

% \lstinputlisting[caption={Output Function of the discrete OLTC controller 2},captionpos=b,style=style-python,label=lst:oltc-discrete2]{images/code/oltc-discrete.py}

\subsubsection{Continous Control Loop}

\begin{figure}[htb!]
        \centering
        \includegraphics[width=.7\linewidth]{development_files/validation/data/oltc_control_characterization.pdf}
        \caption[Characterization of the OLTC control loops]{Characterization of the OLTC control loop; the input function simulates the to be regulated voltage, the output functions are characterized by $o(t)=i(t) \cdot \underline{\vartheta}_\mathrm{trafo}$}
        \label{fig:oltc-control-characterization}
\end{figure}

\subsubsection{Control Schemes for the Fast Switching module}
\commenting{
        \begin{itemize}[nosep]
                \item Describe implementation
                \item Describe benefits / drawbacks
                \item Control scheme
                \item Switching logic and behavior (voltage tracking)
        \end{itemize}
}
\sidenote{Functional basics\\of a \acs{FSM}}
Describe the operational logic and structure of the \acf{FSM} first.

\sidenote{Control logic}
A control logic for a so called \acs{FSM} has been presented from \textcite{burlakin_2024}, and illustrated in \autoref{fig:fsm-control-loop}.

\begin{figure}[htb!]
        \centering
        \includegraphics[width=\textwidth]{modeling/fsm_control_scheme.png}
        \caption[Control loop of a \acf{FSM}]{Control loop of a \acs{FSM}; scheme based on \textcite{burlakin_2024}}
        \label{fig:fsm-control-loop}
\end{figure}

\sidenote{Implementation\\differences}
However, the implementation logic in Python is slightly differing from the presented scheme in \autocite{burlakin_2024}, simply for not overcomplication of the code and therefeore debugging. 
The implementation is similar to the afore discussed one of a standard \acs{OLTC} controller. 

\sidenote{Characterization\\and validation}

%%%%%%%%%%%%%%%%%%%%%%%%%%%%%%%
\subsection{Experimental: Extended Ideas and Improvements}

\subsubsection{Operational Oriented FSM Control}

\ai{
        The time constants are used in the control model to model the influnce of switching operation duration.
        This is coming from the mechanical movement of the OLTC, therefore it is the \glqq maximum possible dynamic behavior\grqq.
        The FSM doesnt't have this limitation, as it can switch after every $\sin$ period (0.02 s).
        
        Currently we can access two different operational modes: Prefer the FSM, or switching on how far is the voltage deviated in both time constants.
        This is a first, more targeted approach towards concerning dynamics in the voltage behavior, but still based on the time constants, not only limited through them.
        
        Why don't we approach a control strategy, which is only considering dynamics, and let the time constants just restrain and block the switching.
        Meaning, that the faster the voltage deviates, the more the FSM gets preferred, the slower the dynamics, the more the standard OLTC gets preferred.
        This could also lead to neglecting a dead-band, and still preventing so called tap-hunting.

        Combined with an operational oriented thought, keeping the possible switching movements of the FSM at its maximum / optimal position.
        This would mean, that in more static times, the FSM switches in its defined \glqq as neutral defined\grqq position and the OLTC is balancing the devaitions.
        One might call this a corrective supervision or monitoring.
}

\subsubsection{Alternative Tap Skipping Logic}

\ai{
        Curretnly the tap skipping logic is formulated as
        \begin{quote} \itshape
                How many times does the deadband fit into the voltage deviation?
        \end{quote}
        to determine the floored number of skips (with contraints). 
        Meaning in mathematical terms: 
        \begin{align}
                \eta(t)=\text{floor}\bigg(\frac{\vert \Delta v(t)\vert}{db_\mathrm{v} \cdot \Delta m}\bigg)
        \end{align}

        An alternative approach would be: 
        \begin{quote} \itshape
                How many switches of the FSM would the current offset voltage bring back to the reference value?
                How many times does one FSM switch fit into the voltage deviation?
        \end{quote}
        Meaning in mathematical terms:
        \begin{align}
                \eta(t)=\text{floor}\bigg(\frac{\vert \Delta v(t)\vert}{\Delta k \cdot \Delta m}\bigg)
        \end{align}
        
        Last approach should be more accurate for different pairs of preset values (deadband, added voltage per tap, etc.).
        BUT: both approaches do not consider the true effect on the dynamic loads and the grid.
        Different grid strengths could react differently on the applied transformer ratio.
}

\subsubsection{Varying the Voltage Setpoint and Target Calculation}

\commenting{
        Here, another idea of control target creation shall be mentioned. 
        Instead of a fixed bus voltage reference, the difference of both bus voltages is considered. 
        Further, the sign of that difference is used to determine the direction of the tap change.
}

\ai{
        Different things to consider here:
        \begin{itemize}
                \item \textbf{Load-flow Direction} with ranking the bus voltages in p.u. against each other,
                \item \textbf{Dynamic Setpoints} through automated calculation of target voltage (nose curves),
                \item \textbf{Different Control Input} as not with a fixed target value, but the difference between both bus voltages; thus tentiative, becuase it is not considering supporting the load, but falsly trying to prevent a wrong switching direction.
        \end{itemize}
}


%%%%%%%%%%%%%%%%%%%%%%%%%%%%%%%
%%%%%%%%%%%%%%%%%%%%%%%%%%%%%%%
\section{Application of Voltage Stability}
\label{sec:application-voltage-stability}

% Combined with the nose curves, critical points shall be identified, highlighted, and characterized.
% Additionally to help this highlighting and characterization, one index based of the nose curves for linearization of remaining distance to the critical point is added. 

As previously discussed in the fundamentals of voltage stability, \autoref{sec:voltage-stability}, ensuring power quality is a secondary goal.
Concerning that voltage stability, regardless of short- or long-time evaluation, is a topic of power quality, it is hard to determine stability or instability.
In terms of static possible solutions, there are a lot of tools determing the critical points, as well as the current distance to it.
Looking into the short-term, more dynamic assessment, there are less elegant solutions. 
This thesis is trying to keep the perspective on both, short- and long term voltage stability.
The following is the approach to synthesize a toolset for voltage stability analysis, that is at least dynamically comparable.
As nose curves are a valid and popular tool, they shall be implemented first. 
Afterwards the time series calculation is tried to integrate into this static evaluation, including tap changer dependent behavior.
Lastly, a more dynamic rating of a sceario shall be computed, enabling also the confirmity with grid codes for example.

%%%%%%%%%%%%%%%%%%%%%%%%%%%%%%%
\subsection{Generation of Nose Curves}
\label{sec:nose-curves}

\commenting{NOTE: Please reconsider and remake this section after completation of the Rating Tool / Assessment Method.}

This section describes the implementation of a prevoiusly discussed static voltage analysis tool.
The generation of nose curves helps in finding the critical loading of the system at the bus of interest, although it is static nature. 

\subsubsection{Basic Simplification Idea}

\textcite{ajjarapu_1992, ajjarapu_2007} are presenting a method for numerical calculation of nose curves in their work. 
It is called {\itshape Continuation Power Flow} and is based on a modified Newton-Raphson method.
The differences rely in a slightly different definition of the power flow equations, considering a load factor $\lambda$.
Combined with a predictor-corrector iterative solver method, this algorithm is capable of nose curve calulation, and finding the critical loading of the system.
While in the first work \autocite{ajjarapu_1992}, only the upper part of the curve including the critical point is calculated, the second work \autocite{ajjarapu_2007} is capable of calculating the complete curve with both solutions. 
As the trade off between implementation effort and the benefits, this method is not exchanging the reduced and simplyfied one.

While this method would be appealing to implement, an additional load flow algorithm, solver, and wrapper seem not profitable for this thesis.
An idea was occuring, just iteratively using the available implemented standard Newton-Raphson algorithm, and implementing a wrapper around it.
The proposed result should be the upper and stable nose curve branch, with the critical point of active power loading.
This shall seem sufficient, as the lower branch solutions are not stable load flow solutions.

The often used parameterization of a function of voltage dependent on the active power and the power angle $\phi$ should be implemented.
In mathematical term, this is expressed as \autoref{eq:pv-mathematical}.
\begin{align}
        \vert \underline{V} \vert : P &\mapsto f(P, \phi) \label{eq:pv-mathematical} \\[6pt]
        Q : \underline{V} &\mapsto f(\underline{V}, \phi) \label{eq:vq-mathematical}
\end{align}
Under consideration of a complex representation of voltage and powers, this algorithm can calculate $V-Q$ curves as well. 
Mathematically this is expressable as \autoref{eq:vq-mathematical}.

\subsubsection{Implementation Details}

\begin{wrapfigure}[12]{r}{0.4\textwidth}
        % \vspace{-20pt}
        \centering
        \includegraphics[width=.9\linewidth]{tikz_graphics/images/class_diagram_nosecurve_red.pdf}
        \caption{Class diagram of the NoseCurve class in the package diffpssi}
        \label{fig:nose-curve-characterization}
\end{wrapfigure}
The implementation of the nose curve generation is realized as a class in the package {\itshape diffpssi.stability$\_$lib.voltage}.
Its class diagram with all attributes and methods is shown in \autoref{fig:nose-curve-characterization}, an extended version is included in \autoref{app:nose-curve}.
For an easy and generic use of the {\itshape diffpssi} package, {\itshape PowerSystemSimulation} objects are used, as well as the function {\itshape do$\_$load$\_$flow()} from the package.

As the afore mentioned idea, the method for running the calculation is a iterative wrapper of the load flow calculation. 
This can be as well applied for mutiple busses as a list input.
At first, the grid and therefore models of the {\itshape PowerSystemSimulation} object has to be cleared with the method {\itshape reset$\_$sim$\_$parameters()}.
Then the active power vector is iterated as load input, together with the power angle $\phi$ for the reactive power in the model.
Important to note here, is the usage of an {\itshape **kwargs} argument.
The callable for the model is called with load parameters for each load bus as the Bus name, and a list with active and reactive power.
The initials of this grid callable are used as the standard values, so only one bus can be varied at a time.
The result is saved as a {\itshape pandas DataFrame} in a dict, with the keys being the bus names.

The method {\itshape plot$\_$nose$\_$curve()} is used to plot the results, and is using the {\itshape matplotlib} package.
Further, the method {\itshape get$\_$max$\_$loadings()} can provide details about the critical point.
Giving back a dict with keys as bus names, the values itself are dicts with key of the power angle parameter $\tan \phi$ and the values as {\itshape pandas DataFrame}.
The contained details are maximum active power $P_\mathrm{max}$, the reactive power $Q$ at this point, and the voltage magnitude $\vert \underline{V} \vert$ at the bus.

\subsubsection{Results of the Nose Curve Generation}

The following figure \autoref{fig:nose-curve-simple-grid} shows the generated nose curve for a simple grid as illustrated in \autoref{fig:single-line-voltage-stability}.
The grid is characterized at Bus 1, with a varying power angle as parameter $\tan \phi$.
The power angle $\tan \phi$ is used to vary the power factor of the load, thus representing different load characteristics, as
\begin{align}
        \tan \phi &= \frac{Q}{P}. \notag %\label{eq:tan-phi} \\[6pt]
\end{align}
Displayed are a few combinations with different load characteristics, leading to a different possible maximum acitve power transfer.
\autoref{fig:nose-curve-simple-comp} shows the comparison between the analytical calculation and the implemented solution.
The analytical calculation is carryied out with the method described in \autoref{sec:analytical-voltage-stability}.
For this specific example, the complete calculation, including the set of used parameters, is shown in \autoref{app:analytical-nose-curve}.
What seems conspicious is the missing lower part of the curve, meaning the second possible solution when solving the power flow equations.
Although this seems like a major drawback, the resulting curve contains all the necessary parts, where a stable solution can occur. \quelle
The solution is reaching exactly until the critical point of power transfer.

\begin{figure}[htbp!]
        \centering
        \includegraphics[width=\linewidth]{development_files/theoretical/plots/simple_load_B1_nose_curve.pdf}
        \caption[Examplary generated nose curve for a simple generator - load grid]{Examplary generated nose curve for a simple generator - load grid for various power angle level parameters $\tan \phi$; Applied on the grid of \autoref{fig:single-line-voltage-stability} with a characterization at Bus 1}
        \label{fig:nose-curve-simple-grid}
\end{figure}

\begin{figure}[htbp!]
        \centering
        \includegraphics[width=\linewidth]{development_files/theoretical/plots/simple_load_B1_nose_curve_w-theoretical.pdf}
        \caption[Comparison between the analytical calculation and the implemented solution]{Comparison between the analytical calculation and the implemented solution}
        \label{fig:nose-curve-simple-comp}
\end{figure}

%%%%%%%%%%%%%%%%%%%%%%%%%%%%%%%
\subsection{Linearization of Power Reserves: Voltage Indices}
\label{sec:voltage-indices-implementation}

\commenting{
        Basic tangent vector index or second order index implementation possible?
}

%%%%%%%%%%%%%%%%%%%%%%%%%%%%%%%
\subsection{Using Voltage Envelopes for Criticality Evaluation}
\label{sec:comb-rating-tool}

\commenting{
        Using the methods of \autocite{scheiner_2022, wildenhues_2015}: Trajectory Violation Index.
        Not only for this classifyied envelope, bus as well for FRT envelopes (MV, HV; Type 2 generation units)
}


%%%%%%%%%%%%%%%%%%%%%%%%%%%%%%%
\subsection{Utilization of Time Series Calculations}
\label{sec:voltage-stability-time-series}

\commenting{
        As section before: 
        \begin{itemize}[nosep]
                \item Finding critical points of the given system and load parameters
                \item At every load change: Find static voltage stability point
                \item Map distance between those points an on the Nose Curves. As well for different Tap changer positions.
                \item Find the time between an event / begin of non-static behavior, until the critical point is reached and overstepped
                \item Nice visualization possible?
        \end{itemize}
        With similar structure as before:
        \begin{enumerate}[nosep]
                \item Idea and background 
                \item Implementation
                \item Results
        \end{enumerate}
}


%%%%%%%%%%%%%%%%%%%%%%%%%%%%%%%
\subsection{Combination of Static Methods with Time Domain Solutions}
\label{sec:comb-rating-tool}

\commenting{
        Idea here: Show the dynamic RMS simulation results in the quasi-staionary assessment techniques.
        These are static solutions to the network, the electromechanic equalization processes should on a long-term watch result in these states.
        With controls of the machines etc. one can obtain more or less a following of the static solutions until a certain point.
        If the grid, or the machine, or its control units are stronger, a certain (heavy) level of load increase can be better and faster compensated.
        
        Combining here:
        \begin{itemize}[nosep]
                \item Nose Curve plots
                \item Curves for different positions of the OLTC control
                \item Time Domain projection in the Nose Curves
                \item Static load solutions of a Transient increase
                \item Time Indices: Transfer Maximum, Voltage Band, FRT cut
                \item Integration of Voltage difference over time: Difference between voltage and voltage envelope 
        \end{itemize}
}

%%%%%%%%%%%%%%%%%%%%%%%%%%%%%%%
%%%%%%%%%%%%%%%%%%%%%%%%%%%%%%%
\section{Summary in Short and Simple Terms}
%!TEX root = ../main.tex

% \part{Enhancing the System Stability Assessment}

%%%%%%%%%%%%%%%%%%%%%%%%%%%%%%%
%%%%%%%%%%%%%%%%%%%%%%%%%%%%%%%
\chapter{Verification Setup and Results}
\label{chap:verification}

\begin{textblock*}{.7\textwidth}(70mm-\offset,25mm-\offset)
    \begin{fquote}[Mark Twain]
        If you tell the truth, you don't have to remember anything.
    \end{fquote}
\end{textblock*}

%%%%%%%%%%%%%%%%%%%%%%%%%%%%%%%
\section{Representative Electrical Networks}

The following section shall introduce the used power systems in the simulation with the Python framework, considering verification, and also extension meaning the performed case studies in \autoref{chap:case-study}. 
The models are chosen to represent different network sizes and complexities, thus allowing the objective of graded interaction levels of the developed (transformer) model. 
The models are based on the work of \textcite{machowski_2020}, \textcite{kundur_2022}, \textcite{IEEELoadModeling_2022}, and \textcite{vancutsem_2020}.

\subsubsection{Single Machine Infinite Bus (SMIB) Model}

One very popular and thus powerful electrical network for the verification of power system stability is the \acs{SMIB} model. 
It is a compact and simplified model of a power system, allowing easy analytical calculation, verification and development. 
Mutual influences are comparably simple to understand and calculate, as the infinite bus bus is acting as a fixed grid connection point with a large adjoining grid. 
The generator is connected to the bus bar via a transmission line and a transformer. 
The model was largely discussed by \textcite{kundur_2022}, and is shown in Figure \ref{fig:smib-model}. 
The generator and the \acs{IBB} are represented by synchronous machines, developed and discussed by \textcite{kordowich_2023}. 
The specific model details are included in \autoref{app:smib-model}, additionally the simulation setup for verification is described in \autoref{tab:smib-model}.

\begin{figure}[htb]
    \centering
    \vspace{12pt}
    \includegraphics{tikz_graphics/images/smib_model.pdf}
    \vspace{12pt}
    \caption[]{\acf{SMIB} model for verification and validation of the Python framework; own figure after \autocite{machowski_2020,kundur_2022}}
    \label{fig:smib-model}
\end{figure}

\begin{table}[htb]
    \caption[Simulation Setup for validation of the $\Pi$-modeled transformer]{Simulation Setup for validation of the $\Pi$-modeled transformer; considering a transforming ratio $\underline{\vartheta} \neq 1$ and $\underline{\vartheta} \in \mathbb{C}$}
    \label{tab:smib-model}
    \vspace*{12pt}
    \centering
    \small
    \begin{tabularx}{\textwidth}{Xr}
        % \toprule
        \textbf{Parameter} & \textbf{Value} \\ \hline
        \toprule
        Generator inertia $H$ & 3.5 s \\
        Generator damping $D$ & 0.1 p.u. \\
        Generator resistance $R$ & 0.01 p.u. \\
        Generator reactance $X$ & 0.1 p.u. \\
        Transformer resistance $R$ & 0.01 p.u. \\
        Transformer reactance $X$ & 0.1 p.u. \\
        Transmission line resistance $R$ & 0.01 p.u. \\
        Transmission line reactance $X$ & 0.1 p.u. \\
        \bottomrule
    \end{tabularx}
\end{table}

Further, this model shall be slightly modified according to \autoref{fig:smib-model-mod}. 
A load is added at the secondary bus of the transformer, the rest of the system is kept. \autoref{tab:smib-model} already contains this modification.

\begin{figure}[htb!]
    \centering
    \vspace{12pt}
    \includegraphics{tikz_graphics/images/smib_model_with_load.pdf}
    \vspace{12pt}
    \caption[]{Modified \acf{SMIB} model with additional load}
    \label{fig:smib-model-mod}
\end{figure}

\subsubsection{Simple Single Machine Load Model}

Following model is often recommended \quelle for easy voltage control studies, in explicit for \acsp{OLTC}. 
Similar to the \acs{SMIB} model, it consists from one synchronous generator, busses, and lines in a single branch. 
The \acs{IBB} is thus removed and changed to a load. 
This two element type o configuration allows for an easy analytical calculation of voltage stability and control. 
Although this thesis is focussing on \acs{OLTC} transformers, the model is extended with one in between. 
A single line representation is depicted in \autoref{fig:single-line-voltage-stability}.

\begin{figure}[htb!]
    \centering
    \vspace{12pt}
    \includegraphics{tikz_graphics/images/sm_load_model.pdf}
    \vspace{12pt}
    \caption[Single line representation of a simple single machine load model]{Single line representation of a simple single machine load model; own illustration with characterstics from \quelle}
    \label{fig:single-line-voltage-stability}
\end{figure}

Further details about its configuration and simulation setup are included in \autoref{app:single-line-model}. 
It should be noted, that simple load models are not useful for simulation of this example network. 
Usually constant Z models are used as loads, therefore simulation results can be misleading and not showing desired effects or voltage instability mechanisms \quelle. 
\commenting{The simulation framework is extended with XX types of load models, to satisfy the requirements of the single machine load model, and a connected stability assessment.}

\subsubsection{IEEE Nine-Bus System}

\subsubsection{Nordic Test System}

%%%%%%%%%%%%%%%%%%%%%%%%%%%%%%%
%%%%%%%%%%%%%%%%%%%%%%%%%%%%%%%
\section{Validation Steps}

%%%%%%%%%%%%%%%%%%%%%%%%%%%%%%%
\subsection{Validation of the Modeled Transformer with Variable Tap Position}

\begin{figure}[htbp!]
    \centering
    \includegraphics[width=\textwidth]{validation/comp_simple_pi.pdf}
    \caption{Comparison of the $\Pi$-modeled transformer in the \acs{SMIB} model between PowerFactory and the Python framework}
    \label{fig:comp-simple-pi}
\end{figure}

\begin{figure}[htbp!]
    \centering
    \missingfigure{Comparison S, theta, x1}
    \caption{Comparison of different varied parameters between \textit{diffpssi} and \textit{DIgSILENT PowerFactory}}
    \label{fig:validation-params-pi-trafo}
\end{figure}

%%%%%%%%%%%%%%%%%%%%%%%%%%%%%%%
\subsection{Validation of the OLTC Control Schemes}

\begin{figure}[htbp!]
    \centering
    \missingfigure{Scenario -> Comparison of bus voltage development between PF and diffpssi}
    \caption{Comparison of bus voltages over the simulation time for both Python framework \textit{diffpssi} and \textit{DIgSILENT PowerFactory} with the applied \acs{FSM} control scheme}
    \label{fig:comp-fsm-control}
\end{figure}

\commenting{
    Due to some factors (which?), the control becomes more off the more states it has -> FSM seems not good, OLTC okay, Loads nearly identical.
    This means there must be differences within the static / dynamic time steps, the load model parameterization, The load model behavior, the model of the transformer itself, the discrecity of the transformer tap control etc.

    This means: Showing, that the control algorithmics is switching correctly, no matter what the result does look like at the bus voltages e.g.
}

\begin{figure}[htbp!]
    \centering
    \missingfigure{Scenario -> K and M integgrator, switching signals, voltahe differences (inputs etc.)}
    \caption{Internal states in the \acs{FSM} control during the applied scenario}
    \label{fig:internal-states-fsm-control}
\end{figure}

\subsubsection{Standard Discrete OLTC Control}

\subsubsection{Fast Switching OLTC Control}

%%%%%%%%%%%%%%%%%%%%%%%%%%%%%%%
\subsection{Voltage Stability Rating Plausibility}

\commenting{
    Place results here, looking at: off nominal tap ratio, and with off nominal phase shifting (e.g. $110^\circ$)

    Things to illustrate:
    \begin{enumerate}
        \item Validity of Top Branch Nose Curves: Analytical and PowerFactory; Simple Network and e.g. IEEE 9-bus 
        \item Analytical Validity OLTC ratio dependent Nose Curves
        \item Time Domain Projection: Dependence and Stability in transient scenarios vs. continous load increase; Dependence on Machine Controls; Dependence on Apparent Power Capacity of machine 
    \end{enumerate}
}

%%%%%%%%%%%%%%%%%%%%%%%%%%%%%%%
\section{Model Limitations and Improvements}

\section{Summary in Short and Simple Terms}
%!TEX root = ../main.tex

% \begingroup
% \newgeometry{left=2.5cm,right=2.5cm,top=2.5cm,bottom=2.5cm}
% \part{Practical Application: Simulation}

% \endgroup

%%%%%%%%%%%%%%%%%%%%%%%%%%%%%%%
%%%%%%%%%%%%%%%%%%%%%%%%%%%%%%%
\chapter{Case study}
\label{chap:case-study}

% \begin{textblock*}{.7\textwidth}(70mm,25mm)
%     \begin{fquote}[Narcotics Anonymous]
%         Insanity is doing the same thing, over and over again, but expecting different results.
%     \end{fquote}
% \end{textblock*}

\commenting{
    In the interest of investigation / the Case Study are:
    \begin{itemize}
        \item Influence of switching times on stability margin/begin of destabilization,
        \item Influence of max. ratio change per switching event, and
        \item Influence on different test systems (destabilization mechanisms).
    \end{itemize}
}

\section{Scenario setting}

\commenting{
    Does it make sense to structure like that? (Scenarios - Simulation - Results)

    Or is it a better idea thinking in terms of specific \glqq use cases\grqq~as sections:
    \begin{itemize}
        \item What happens under strong grid conditions? -> Section: Strong grid condition behavior
        \item What happens under weak grid conditions? -> Section: Weak grid condition behavior
        \item Strongly interconnected grids
        \item Widely extended linear string grids
        \item Section: Use case of Wind farm integration
        \item Influence on transient stability: SMIB model with and without OLTC
    \end{itemize}
}

\subsubsection{Influence of FSM on Machines and their stability criterions}

\commenting{Thinking of Rotor Angle Stability, maybe considered by an EAC implementation?
What does the fast Switching, esp. at up to 8\% of the nominal voltage, do to the machines?}

\subsubsection{Novel Control Strategy FSM}

\commenting{Thinking of fast and slow voltage gradients: fast gradients are compensated by the FSM, slow gradients are compensated by the OLTC. Therefore optimal utilisation of injected damping moment of the FSM. 

Also thinking of different presets of the OLTC and FSM, which are tried to keep constant. Different grid operators can utilize for typical grid conditions of over- or undervoltage at \acs{PCC}.

Following contains:
\begin{itemize}
    \item Implemetation of different logic
    \item Testing of presets and switchin logic
    \item Damping moment beneficial?
\end{itemize}
}

\subsubsection{Possible Extension for Power Flow congruent Control}

\commenting{Extension in the Control Algorithm to decide which Bus has to be regulated, to avoid contrairy actions of the OLTC and FSM against the power flow. Therefore not decrease of stability, but increase. Possible Application: Grid coupling Transformers, Battery Storage assisted Virtual powerplants, etc.

For this thinking maybe another control strategy is relevant:
\begin{itemize}
    \item No setpoint from a load flow day-ahead or similar time frame; but rather current load and bus voltages are considered
    \item Not the deviation of one transformer bus voltage from a setpoint, but the deviation between the two bus voltages is relevant
    \item Maybe the absolut deviation to the optimal bus voltages at the current load situation is relevant  
\end{itemize}

Big general problem: In which direction does the OLTC / the FSM have to swith? In some cases, the direction is not correct, in some it is correct.}

\section{Simulation}

\section{Results}
%!TEX root = ../main.tex

% \addtocontents{toc}{\protect\addvspace{2.25em}}
% \bookmarksetup{startatroot}

%%%%%%%%%%%%%%%%%%%%%%%%%%%%%%%
%%%%%%%%%%%%%%%%%%%%%%%%%%%%%%%
\chapter{Discussion of the Results}
\label{chap:discussion}

\begin{textblock*}{.7\textwidth}(70mm-\offset,25mm-\offset)
    \begin{fquote}[Joseph Joubert]
        The aim of argument, or discussion, should not be victory, but progress.
    \end{fquote}
\end{textblock*}

This chapter discusses all chapters combined, with found aspects, expected or unexpected behaviors, or comparisons.
Singular discussions in each chapter are avoided, to maintain a high level view on the \acs{FSM}.
Nevertheless, details do matter here and are included in the summarized evaluation.
The structure of this chapter does not hold up with the before used structure.
This shall make use of the combined discussion, more orientating on the found aspects, than the done work. 

%%%%%%%%%%%%%%%%%%%%%%%%%%%%%%%
\section{Integration in diffpssi}

First, the done implemetation in the tool of choice, \textit{diffpssi} is discussed.
Conspicuities, current existing restrictions of the assessments, as well as missing ones are elaborated.
At last, a further idea of utilizing the benefits of \textit{diffpssi} is illustrated.

\subsubsection{Implementation of the Models}

\sidenote{Transformer equipment}
As the results for the model validation are already described in \autoref{chap:verification}, there is a relatively short discussion on the accounted errors.
The errors show overall low values in comparison to the commercial software \textit{DIgSILENT PowerFactory}.
The variable ratio transformer itself has errors of maximum in the one digit percent range.
As this can also account for solver or time step issues, the Python framework is giving competitive results.
Considering the injected errors of the already implemented load model, the results of the control schemes are even less error prone.
It is assumed, that because of the longer time period of holding the voltages closer to the reference value, the error can even be dropped. 
Peaks only occur during the switching of tap positions of the \acs{OLTC} or \acs{FSM}, where not only solver issues, but the time filtering or else can accumulate for a very short time.
The logic of the \acs{FSM} controls only shows correct results, as also confirmed by \autoref{chap:case-study}.

\sidenote{Voltage stability tools}
As the only comparison for the voltage stability tools can be drawn with respect to the Nose Curves, these deliver very good results as well.
The curves are congruent, making it hard to visually spread them apart, both compared analytically and with \textit{DIgSILENT PowerFactory}.
The other tools show a logic beahavior, as the visual inspections allow for the same claims as the algorithmics.
Only one results seems somehow suspect, as the \acs{TVI} calculation shows even a violation area for the stable cases of one bus in \autoref{chap:case-study}, specifically \autoref{tab:case1-tvi}.
It was not possible to investigate on the cause of this unexpected behavior so far.

\subsubsection{Currently Existing Restrictions}

\sidenote{General}
However there are some restrictions to the Python framework, considering the applicability on voltage stability studies.
As only static load models or synchronous machines can be represented as loads, the characteristics of a realistic grid is very limited.
Further, \textit{diffpssi} only contains the constant impedance load model.
Often pointed out by \textcite{cutsem_1998}, \textcite{kundur_2022}, or \textcite{danish_2015}, the bottleneck for remaining stable voltage levels are induction machines.
These are currently not available in \textit{diffpssi}, so a lot of studies will not show the same characteristics as one would be able to conduct with comparative (commercial) software.
If this model would be available in the future, even combinations with inverters as supporting reactive power source, or connected controllers could be tested very easily.
Nevertheless this package is suited well for basic developments of single components, such as a tap changer controller.
The logic and algorithmics are implemented fast and allow for a transparent debugging.
The easy and close connection with Python does make the package especially great for direct evaluation of results.
Either in plotting and visualization, for statistic assessments, or even batch calculations, but as well for futher processing and calculations with the results. 

\sidenote{Transformer related}
Additionally, the variety of available transformer types is also very limited.
There are a lot of different technologies possible to consider, from line drop compensations over phase shifting transformers or \ac{WAMPAC} approaches.
Even simpler things, like parallel transformers, and the related complications with circular currents cannot be adressed with \textit{diffpssi} or this thesis.
Regarding transformers, \textcite{sarimuthu_2016} is presenting a review paper considering a few of these topics as an overview.
When looking at the possible damping factor from the \acs{FSM} control, one could also think of something like a \glqq machine drop compensation\grqq~for futher development.

\subsubsection{Missing Assessments in this Thesis}

Looking at the integrity of the implemented models, there are a few edge cases or considerations, which are excluded.
These however could be beneficial, depending on the application of the technologies.
A parameter variation for the \acs{FSM} characteristics has not been carried out.
The standard values presented by \textcite{burlakin_2024} have continuously been used, as they are comparable and show conclusive results so far.
However for different use cases, voltage and power levels or dynamic loads, also other parameters have to be considered.
On top of that and regarding the different basic use cases for transformers, only a transformer connecting a machine was used in this thesis.
As transformers for complete power plants, virtual power plants, connecting \acp{BESS}, grid coupling, etc. \autocite{schwab_2022} is usually also covered, these application could also be of interest.
The later on discussed control improvements could be not implemented and evaluated as well.

\subsubsection{Increasing the Benefits of diffpssi}

One further interesting application of the Python package \textit{diffpssi} is the possibility for parameter differentiation based on \textit{PyTorch}.
\textcite{kordowich_2023} uses this for an optimization approach, allowing to optimize certain defined simulation parameters to acquire a defined system behavior.
For example, the dampening of the synchronous machine swinging after a short-circuit event can be increased.
The idea is, to apply this method also on the \acs{FSM} control scheme, thus optimizing its behavior, e.g. number of switch operations, power oscillation damping or other things.
As the tap changer model and its control is already implemented, one could make use of this functionality.

%%%%%%%%%%%%%%%%%%%%%%%%%%%%%%%
\section{Evaluation Current FSM Control}

% After the toolchain, the content of interest is more central.
The \acs{FSM} control schemes are analyzed in the following, which characteristics, benefits, and current drawbacks exist.
Lastly, further investigations on the control schemes are named, as they are not included in this thesis any more.

\subsubsection{Characteristics of the FSM controls}

The results as differences are mainly illustrated in the \autoref{chap:case-study}, but in some terms visible in \autoref{chap:verification} as well.
The increased stabilization through the \acs{FSM} control can be confirmed for both switching only with the \acs{FSM} part and with both dependent on the voltage deviation.
% The increased stabilization through the \acs{FSM} control can be confirmed, for both only switching with the \acs{FSM} part, but as well with both dependent on the voltage deviation.
The envelopes are not cut after the fault at all, while the \acs{OLTC} scheme can only postbone the rapid fall a few seconds.
Even if one would want to argue, that with the accounted \acs{FSM} schemes, the control has a wider range of operation.
One can clearly see in the \acs{TDS}, that the rapid intervention of the \acs{FSM} is holding the voltage closer to the reference value a lot earlier.
On top of that, considering the \acs{FSM} preferring control, no single switch of the \acs{OLTC} has occured, and yet the system is stable for a longer period of time.

One main finding in this thesis is the back propagation of the \acs{FSM} control on the oscillation behavior of synchronous machines after the fault.
Clearly, an influence through the ratio is visible, if not even the crucial part to stabilizing the whole system after the fault.
Thus the \acs{FSM} as well help stabilize a system with more and faster dynamics compared to the \acs{OLTC}.
Another finding is the accounted deaf band in \autoref{sec:validation-fsm-schemes}, which has especially be considered for different combinations of deadband size and switching magnitude of the \acs{FSM}.
As one can also see in \autoref{fig:case1-trans-ratio}, the \acs{FSM} ratios are not vastly different from the \acs{OLTC}, but just are coarser per tap change.
As this is leading to a lot less switches of the \acs{OLTC} and therefore less use of the mechanical switching contacts, both \acs{FSM} controls seem to have an optimal behavior somewhere in between them.
Utilizing the \acs{FSM} where possible and only short- or mid-term fluctuations occur.
And using the \acs{OLTC} where the load flow and voltage has to be influenced long-term, enabling the most dyanmic capabilities.   

\subsubsection{Further Investigations}

Regarding the further investigations of the \acs{FSM} control schemes, one aspect has to mentioned.
The next step would be an assessment with machine controllers, as gaining insights on how these controllers would interact.
If one would want to use a \acs{FSM} equipped controller as a machine or power plant connector to the grid, this is a crucial part.
As the short control times already have an influence on the machine dynamics, they could also interfer with the short term considering machine controllers, like exciters \autocite{machowski_2020}.
This expectation holds true for inverter controllers as well.

Regarding the topic increasing voltage stability through a \acs{FSM}, an even bigger potential is the application on to a phase shifting transformer in particular.
As these transformers are also able to change the voltage angle difference between two nodes, this could have an even bigger impact as the standard longitudinal tap changer.
When the time until destabilization of the system can be further increased by that, the system cannot only be designed with more risk, but as well withstand larger disturbances.
    
%%%%%%%%%%%%%%%%%%%%%%%%%%%%%%%
\section{Development Potential of the FSM and its Control}
\label{sec:experimental-modeling}

This section introduces a few ideas for improvements of the \acs{FSM} voltage controller.
Based on the conducted application studies from \autoref{chap:case-study} and the before deducted discussions, these ideas are not implemented and tested.

\subsubsection{Alternative Tap Skipping Logic}
\label{sec:modeling-alt-tap-skip}

The first idea is concerning the function \textit{tap$\_$skip()} in the voltage controller of the \acsp{FSM}.
Especially as the in \autoref{sec:validation-fsm-schemes} illustrated results show a deaf band for the \acs{FSM} preferred control loop, this logic is to be questioned for the voltage deviation dependent switching.
In the latter logic, this function \textit{tap$\_$skip()} is making a big influence on the dynamic behavior.
If one would try to formulate the current tap skipping function from \autoref{eq:tap-skip} in words, something in the following form could describe it: 
\begin{quote} \itshape
        How many times does the deadband fit into the voltage deviation? The tap skips are then considered under the amplifying factor of the \acs{FSM} applied on the tap addition of the \acs{OLTC} $\Delta m$.
\end{quote}
As the deadband has less to do with the impact of a \acs{FSM} switch, and it is already respected within the controller activation of both \acs{OLTC} and \acs{FSM} contribution, this relation to the dynamics seems obsolete.
An alternative and seemingly more targeted approach would be: 
\begin{quote} \itshape
        How many switches of the \acs{FSM} would the current offset voltage bring back to the reference value?
        In translated terms meaning: How many times does one \acs{FSM} switch fit into the voltage deviation?
\end{quote}
This being translated in mathematical terms, according to \autoref{eq:tap-skip} and the respect of $\eta(t) \in \mathbb{Z}$ and the \acs{OLTC} voltage addition per tap, the new function for calculating the ideal tap skips by the \acs{FSM} is formulated in \autoref{eq:new-tap-skips}.
\begin{align}
        \eta(t)=\text{floor}\bigg(\frac{\vert \Delta v(t)\vert}{\Delta k \cdot \Delta m}\bigg) \label{eq:new-tap-skips}
\end{align}
This approach should be more accurate for different pairs of preset values, meaning the size of the deadband, the added voltage per tap, the amplifying factor of the \acs{FSM}, etc.
It is expected, that the current logic results in a plausible way, as the ratio addition per tap of the \acs{OLTC} is near the size of the deadband.
This means, that the proposed function is very similar for the present and in this thesis mainly used parameterization of the \acs{FSM} control loop.
However, it is expected that the proposal is thus more robust and better working for the applied cases and more variatons in the configuration.

One comment on this proposal and the original function idea has to be made anyway.
Either logic solely considers the influence of the tap changer, but no load or grid dynamics at all.
Even the time constants do not have a influence on the switching behavior.
As this could also be beneficial for fast responses, the large impact of the novel \acs{FSM} equipped tap changers in a very short time can also irritate other control units or counteract to processes and destabilize a grid area unnecessarily.
Thus a true voltage dfference or voltage difference gradient based control scheme, under consideration of the individual minimum possible time constants would appear to be logically the best solution.
Such an appraoch is decribed in the following.

\subsubsection{Operational Oriented FSM Control}
% \mycomment[MK]{Noch ergänzen: Dämpfung als Dynamische Komponente bei Operation Control}
\label{sec:modeling-op-control}

\sidenote{Comment on the representation of time constants}
The time constants of both parts, the \acs{OLTC} and the \acs{FSM}, are relevant as they model the minimal needed duration of the switching.
This minimal time is based on mechanical or electric limitations, such as the mechanical movement of the \acs{OLTC} tap changer.
In the current schemes they are not represented as a limitation, but more just as a delay.
For example, if the voltage difference falls within the deadband, the integrators or time delays are resetted to their initial state. 
If the falling within was just an error or a short swing, so the deadband is suddenly exceeded again after a very short time.
As the switch of the \acs{OLTC} would mechanically move, this sudden exceed would then let the time delay start from the beginning.
While in reality the physical switch has not been given enough time to reach its starting position, the next switch could be achieved faster. 

\sidenote{Corrective supervision}
As before described, the longer time constants of the \acs{OLTC} come from the mechanical switch movement, which are not the case for the \acs{FSM}.
Therefore the maximum dynamic ability to react on voltage deviations is a lot higher for the \acs{FSM}.
Keeping this dynamic capability means according to the findings of \autoref{sec:case-2}, that damping reserves for e.g. short-circuit events is held in reserve.  
With the move from the preffered \acs{FSM} switching to the voltage dependent activation, a significant step was made towards dynamic influences instead of just a \glqq range extender\grqq.
One could think of even improving this behavior, as keeping the dynamic capabilites through re-arranging the positions of $k$ and $m$.
In more static cases a preset of one of the tap changers postion, e.g. the more dynamic $m$, can be restored with keeping the overall ratio constant.
This could be possible through coordinated counter switching of the \acs{OLTC} and the \acs{FSM}.
When considering, that the range of an \acs{OLTC} is typically aroung $k \in [-10,10]$, the FSM seems very limited with $m \in [-4,4]$.
One impoortant influence at this point, is the amplification of the \acs{FSM}, meaning the factor multiplied with the \acs{OLTC} tap skip change as \acs{FSM} voltage deviation per skip. 
Therefore not all overall tap ratios are representable dependent on the restricted part of the logic and the factor relationships.
This would utilize the \acs{OLTC} better on a long-term perspective, as not only the \acs{FSM} would be used for small dynamic deviations.
The described behavior could be named as corrective supervision or monitoring. 

\sidenote{Using voltage gradients}
In order to select or deselect the \acs{FSM} or \acs{OLTC}, the current approach through the tap skipping function seems hands-on and sufficient.
As before described, it does not account for different time constants and thus durations until the voltage deviation can be corrected.
This calculation is a retrospective procedure, as only the current value is referenced to the voltage setpoint.
If this deviation became too big, the \acs{FSM} switching is initiated.

In contrast to that, if one would account for the current voltage gradient in addition to the current deviation, a prediction over the time constant modeled switching limitation can be given.
This would bring the controller in a mode, where the switching activation would be anticipated.
Further a gradient could easily help to determine which part, the \acs{FSM} for more dynamic action or the \acs{OLTC} for more static actions, should be used.
With this idea one could even imagine neglecting a voltage deadband, as a time deadband would be more applicable towards swing characteristics. 
As swings, or then damping of swings in a system, would symptomatically end in the same characteristics as tap hunting, this in between mode could be realized.
The proposed changes can be realized with a split control path, devided into a preset calculation and a physical switching representation transacting this on to the transformer. 
These two compartments are then forked with a corrective supervision to realize off-nominal transformer ratios with optimal dynamic capabilities as presetted by the operator.

\subsubsection{Dynamic Measurement and Reference Voltage Setpoints}

The last, least expected approach to be profitable, is the area of measurement and reference setting.
On the one hand, it could be imagined, tracking the voltages at both busbars and thus getting insights on the load flow direction.
The load flow direction, in combination with the relative positioning of slack busbars to the transformer, are expected to define the switching direction of the tap changer transformers.
Additionally dynamic setpoints are imagined to be calculated as references.
This means, that a new load flow calculation is done every time the load in the network or at least network area changes.
With considering a direct supply of only a load, or at least a construct summarizeable as one load, an additonal block in the control scheme representing Nose Curves can be imagined as suitable approach.
This representation static possible solutions could allow for an automated reference voltage and initial tap changer position.
Especially when considering the before described operational oriented control.

%%%%%%%%%%%%%%%%%%%%%%%%%%%%%%%
%%%%%%%%%%%%%%%%%%%%%%%%%%%%%%%
\chapter{Summary and Outlook}
\label{chap:summary}

\begin{textblock*}{.7\textwidth}(70mm-\offset,25mm-\offset)
    \begin{fquote}[Robert Frost]
        In three words I can sum up everything I've learned about life: it goes on.
    \end{fquote}
\end{textblock*}

Concluding this master thesis, a variable ratio transformer model is implemented in a power system simulation framework, based on Python.
On top of that, the transformer is equipped with different tap changer control circuits, also satisfying the logic of a novel tap changing technology, the \acf{FSM}, with an increased dynamic capability.
Methods and tools for the evaluation of voltage stability are implemented, the calculation of different nose curves, and an index accounting for the violation integral of a voltage band.

These implementations are compared and validated against the commercial software \textit{DIgSILENT PowerFactory}.
An application study is looking into the funcitonalities and the back propagation on machine dynamics in a simple \acf{SMIB} model.

\sidenote{Main Research Questions}
\textbf{1. How do different control types and characteristics of transformers with \acfp{OLTC} influence the voltage stability of a given system?}

Considering the first, and main research question, it can be stated, that an increased dynamic capability of a transformer does help stabilizing the bus voltages in a network.
Due to the interaction of the fast tapping control with the connected machines, the power and speed oscillations can be slightly damped.
Faster reaction on voltage runaways is possible.
This increased system stability can be quantified by the \acf{TVI}, considering a voltage bandwith as an envelope.
Additionally, the critical time of leaving this operational voltage bandwith can be postponed, enabling other facilities to react with sufficient reactive power supply.

\sidenote{Secondary Questions}
\textbf{2. Can the already existing tap changer control of the \acf{FSM} be improved towards a more operation oriented control?}

Answering this questions is highly dependent on the interest of the operational use.
As every tap change of the mechanical \acs{OLTC} is wearing the switch mechanism, while the \acs{FSM} does not, a grid node with high dynamics will demand other strategies as an either static one.
The higher dynamic capability allows more dynamic use cases, than for example just managing the load flow in a network.
Especially considering a possible damping moment, the use as a supplementary power oscillation damping is conceivable.
Nevertheless, some improvements targeting also the operational use, are discussed within this thesis.
This takes a different voltage dependent enabling into account, which could allow for more variability in the controller parameterization.
Further, a new control proposal is illustrated, ensuring more flexibilities and generic use for different applications.
An additional integration into the control of synchronous generator is possible, when considering the tap changing transformer as connection to the network.

\sidenote{Further Investigations}
Some further investigation has to be done either way. 
The interaction between power plant controllers and the \acs{FSM} can become a crucial part. 
Additionally, one might consider implementing also induction machines to the Python module, allowing to look at more voltage threatening scenarios.
Additionally, an optimization of the control parameters to different grids or application scenarios would contribute to the understanding and the capabilities as well. 

\sidenote{Outlook on the Fast Switching Module}
Looking into the future of the \acs{FSM}, a big topic is the application on phase shifting transformers.
Connecting virtual power plants or \acf{BESS} can become an application for the \acs{FSM} equipped transformers as well.
As the heavy use of inverters in this topic, the advanced transformer could account for side uses as power oscillation damping, enabling more flexibilities in the use of differing operational points for example.
\cleardoublepage

\pagenumbering{Roman}
\setcounter{page}{13}
\begin{spacing}{1.15}
    
%%%%%%%%%%%%%%%%%%%%%%%%%%%%%%%%%%%%%%%%%%%%%%%%%%%%%%%%%%%%%%%%
% Anmerkungen zur Verwendung:
%%%%%%%%%%%%%%%%%%%%%%%%%%%%%%%%%%%%%%%%%%%%%%%%%%%%%%%%%%%%%%%%
%
% nur verwendete Akronyme werden letztlich im Abkürzungsverzeichnis des Dokuments angezeigt
% Verwendung: 
%		\ac{Abk.}   --> fügt die Abkürzung ein, beim ersten Aufruf wird zusätzlich automatisch die ausgeschriebene Version davor eingefügt bzw. in einer Fußnote (hierfür muss in header.tex \usepackage[printonlyused,footnote]{acronym} stehen) dargestellt
%		\acs{Abk.}   -->  fügt die Abkürzung ein
%		\acf{Abk.}   --> fügt die Abkürzung UND die Erklärung ein
%		\acl{Abk.}   --> fügt nur die Erklärung ein
%		\acp{Abk.}  --> gibt Plural aus (angefügtes 's'); das zusätzliche 'p' funktioniert auch bei obigen Befehlen
%	siehe auch: http://golatex.de/wiki/%5Cacronym
%
%%%%%%%%%%%%%%%%%%%%%%%%%%%%%%%%%%%%%%%%%%%%%%%%%%%%%%%%%%%%%%%%
\cleardoublepage
\addcontentsline{toc}{chapter}{Acronyms}
\chapter*{Acronyms}

% \addchap{\langabkverz}
\begin{acronym}[mmmmmm] %hier längstes Acro
% \begin{doublespacing}
% \setlength{\itemsep}{-\parsep}

\acro{CCT}{Critical Clearing Time}
\acro{EMT}{Electromagnetic Transient}
\acro{FSM}{Fast Switching Module}
\acro{IBB}{Infinite Bus Bar}
\acro{IEEE}{Institute of Electrical and Electronics Engineers}
\acro{IM}{Induction Machine}
\acro{ODE}{Ordinary Differential Equation}
\acro{OLTC}{On-Load Tap Changer}
\acro{PCC}{Point of Common Coupling}
\acro{PSS}{Power System Simulation}
% \acro{PSS}{Power System Stabilizer} % Das ist die eigentliche korrekte Bedeutung/Abkürzung!!!
\acro{RMS}{Root Mean Square}
\acro{SG}{Synchronous Generator}
\acro{SMIB}{Single Machine Infinite Bus}
\acro{TDS}{Time Domain Solution}
\acro{TVS}{Tangent Vector Index}

\end{acronym}

%%%%%%%%%%%%%%%%%%%%%%%%%%%%%%%%%%%%%%%%%%%%%%%%%%%%%%%%%%%%%%%%
\addcontentsline{toc}{chapter}{Symbols}	
\chapter*{Symbols}

% \addchap{Symbols}
\begin{tabbing}
    XXXXXXXXX \= XXXXXXXX \= XXXXXXXXXXXXXXXXXXXXXXXXXXXXXXXXXXXXXXXXXXXXXXXXX \kill
    $\delta$            \> $^\circ$ / deg                   \> power angle (or power angle difference) \\
    $\Delta\omega$      \> $\mathrm{\frac{1}{s}}$           \> change of rotor angular speed \\
    $\underline{\theta}$\> -                                \> transformer ratio; complex if phase shifting \\
    $A$                 \> -                                \> acceleration or deceleration area \\
    $\underline{E}$     \> V                                \> voltage of \acs{SG} or \acs{IBB} \\
    $H_\mathrm{gen}$    \> s                                \> inertia constant of a \acf{SG} \\
    $\underline{I}$     \> A                                \> current \\
    $P$                 \> W                                \> effective power; electrical or mechanical \\
    $Q$                 \> var                              \> reactive power \\
    $R$                 \> $\mathrm{\Omega}$                \> ohmic resistance \\
    $\underline{S}$     \> VA                               \> apparent power \\
    $\underline{V}$     \> V                                \> voltage \\
    $\underline{X}$     \> $\mathrm{\Omega}$                \> reactance \\
    $\underline{Y}$     \> $\mathrm{\frac{1}{\Omega}}$ / S  \> admittance \\
    $\underline{Z}$     \> $\mathrm{\Omega}$                \> impedance \\
\end{tabbing}

The different symbols are used with different indices, these are semantic and explained in the surrounding context. Following notation is commonly used for mathematical and physical symbols:
\begin{itemize}[noitemsep]
    \item Phasors or complex quantities are underlined (e.g. $\underline{I}$)
    \item Arrows on top mark a spatial vector (e.g. $\overrightarrow{F}$)
    \item Boldface denotes matrices or vectors (e.g. $\mab{F}$)
    \item Roman typed symbols are units (e.g. $\mathrm{s}$)
    \item Lower case symbols denote instantaneous values (e.g. $\underline{i}$)
    \item Upper case symbols denote \acs{RMS} or peak values (e.g. $\underline{I}$)
    % \item References to objects are written capitalized Roman (e.g. $\underline{Z}_\mathrm{TRAFO}$)
    \item Subscripts relating to physical quantities or numerical variables are written italic (e.g. $\underline{I}_1$) 
\end{itemize}

In the simulations and calculations the per unit system ($\mathrm{p.u.}$) is preferred, thus normalizing all values with a base value. Where necessary, absolute units are added to indicate the explicit use of the normal unit system. For more information about this per-unit system please refer to \textcite{machowskiPowerSystemDynamics2020}, specifically Appendix A.1 provides a detailed description and explanation.									% Symbol- und Abk�rzungsverzeichnis
    \listoffigures   											% Abbildungsverzeichnis
    \listoftables   											% Tabellenverzeichnis
    \lstlistoflistings
    \printbibliography
\end{spacing}

\cleardoublepage
\pagenumbering{alph}
\setcounter{page}{1}
\phantomsection\addcontentsline{toc}{chapter}{Appendix}
\appendix
% !TeX root = ../main.tex

% \appendixtoc
% \appendix
% \ohead[]{\textsc{Appendix}}
\label{app:appendix}

% \renewcommand\thechapter{\roman{chapter}}
% \setcounter{chapter}{0}

% \pagebreak
% \includepdf[pages=-,scale=.9,pagecommand={}]{Aufgabenstellung.pdf} 
% PDF um 10% verkleinert einbinden --> Kopf- und Fußzeile  werden so korrekt dargestellt. Die Option `pages' ermöglicht es, eine bestimmte Sequenz von Seiten (z.B. 2-10 oder `-' für alle Seiten) auszuwählen.
% \pagebreak
%\includepdf[pages=-,scale=.8,pagecommand=\section*{A. eventGenerator.py}]{../appendix/eventGenerator.py.pdf}
%\includepdf[pages=-,scale=.8,pagecommand=\section*{B. sendEvents.py}]{../appendix/sendEvents.py.pdf}


% \RedeclareSectionCommand[beforeskip=\kapitelabstand         ]{chapter}


%%%%%%%%%%%%%%%%%%%%%%%%%%%%%%%%%%%%%%%%%%

%%%%%%%%%%%%%%%%%%%%%%%%%%%%%%%%%%%%%%%%%%
%%%%%%%%%%%%%%%%%%%%%%%%%%%%%%%%%%%%%%%%%%
\chapter{Fundamentals}

\section{Description of the Power System Simulation process}
\label{app:power-system-modeling}

In this appendix section, the general process of power system simulation is described. As this thesis is aiming to understand voltage stability and processes in longer periods of time, these explanations apply to pointer-based simulations, called RMS simulations. Meaning that the considered effects are slower electromechanical nature instead of faster electromagnetic ones. The in this thesis used Python framework \glqq \textit{diffpssi}\grqq~is based on this type of simulation, and due to its open-source based nature traceable.

\begin{figure}[htbp]
    \centering
    % \includegraphics[width=\textwidth]{fundamentals/power-system-simulation-process.pdf}
    \missingfigure{Power system simulation process}
    \caption{Power system simulation process; own illustration}
    \label{fig:power-system-simulation-process}
\end{figure}

\commenting{
    Really basic: (?)
    \begin{itemize}
        \item RMS vs EMT simulation (-> meaning one cannot simulate other faults than 3ph w/o ground)
        \item Phasor description
        \item Basic formulation: Static (algebraic) and dynamic (differential) equations
        \item Using of solvers (Integrators) for time domain simulation
        \item Using of different optimizatinon algorithms for steady state (load flow) simulation -> initial values
    \end{itemize}
    Less basic and more advanced:
    \begin{itemize}
        \item rountines in the framework
        \item two types: Algebraic and Differential equations have to be solved at each time step -> What is which? Which operational equipment is typically described with which type of equation?
        \item per unit system applying for easier simulation (different voltage levels)
        \item ...
    \end{itemize}
}

%%%%%%%%%%%%%%%%%%%%%%%%%%%%%%%%%%%%%%%%%%
\section{Jacobian based voltage stability criterions}
\label{app:jacobian-voltage-indices}

\textcite{danish_2015} is showing, describing, and referencing some voltage stability indices based on the Jacobian matrix. The following table is a collection of these indices.

\begin{sidewaystable}[htbp!]
    \centering
    \small
    \caption{Jacobian based voltage stability criterions; after \textcite{danish_2015}}
    \vspace*{12pt}
    \renewcommand{\arraystretch}{2}
    \begin{tabularx}{23cm}{llXXl}
        % \toprule
        \textbf{Index} & \textbf{Abbreviation} & \textbf{Calculation} & \textbf{Stability Threshold} & \textbf{Reference} \\
        \toprule
        Tangent Vector Index & \acs{TVI} & $\mathrm{TVI}_i=\left\lvert \frac{\dd{V_i}}{\dd{\lambda}}\right\rvert^{-1}$ & depending on load increase & \\ \midrule
        Test Function & & $t_{cc}=\left\lvert e^T_c \cdot \mab{J} \times \mab{J}_{cc}^{-1} \cdot e_c\right\rvert$ & details are given in reference & \\ \midrule
        Second Order Index & $i$ & $i=\frac{1}{i_0} \cdot \sigma_\mathrm{max} \cdot \big( \dv{\sigma_\mathrm{max}}{\lambda_\mathrm{total}} \big)^{-1}$ & $i > 0$ & \\ \midrule
        Minimum Eigenvalue & & $\Delta V=\sum_{i} \frac{\xi_i\eta_i}{\lambda_i} \Delta Q$ & all eigenvalues should be positive & \\ \midrule
        Minimum Singular Value & & $\begin{bmatrix} \Delta \vartheta \\ \Delta V \end{bmatrix}=\mab{V} \sum^-1 \mab{U}^T \begin{bmatrix} \Delta F \\ \Delta G \end{bmatrix}$ & details are given in reference & \\ \midrule
        Predicting Voltage Collapse & & $\frac{V}{V_0}$ & the smallest index value & \\ \midrule
        Impedance Ratio & & $\frac{Z_ii}{Z_i}$ & $\frac{Z_ii}{Z_i} \leq 1$ & \\
        \bottomrule
    \end{tabularx}
\end{sidewaystable}



%%%%%%%%%%%%%%%%%%%%%%%%%%%%%%%%%%%%%%%%%%
\section{Comparison of System based and Jacobian based indices}
\label{app:jacobian-vs-system-indices}

% %%%%%%%%%%%%%%%%%%%%%%%%%%%%%%%%%%%%%%%%%%
% %%%%%%%%%%%%%%%%%%%%%%%%%%%%%%%%%%%%%%%%%%
\chapter{Modeling}

%%%%%%%%%%%%%%%%%%%%%%%%%%%%%%%%%%%%%%%%%%
\section{Admittance Calculation of a Two-Port}
\label{app:admittance-deduction}

Follwing part shall just give a short, but complete and clear overview, how the admittance matrix of a two-port system is calculated.
Therefore the main focus of this thesis, a two-port with variable translation ratio, is kept.

%%%%%%%%%%%%%%%%%%%%%%%%%%%%%%%%%%%%%%%%%%
\section{Class Diagram of the Class Nose Curves}
\label{app:nose-curve}

\begin{figure}[htbp!]
    \centering
    \includegraphics[width=12cm]{tikz_graphics/images/class_diagram_nosecurve_complete.pdf}
    \caption{Complete class diagram of the class Nose Curves; including all attributes and methods with data types, returns, and inputs}
    \label{fig:class-diagram-nose-curves}
\end{figure}

%%%%%%%%%%%%%%%%%%%%%%%%%%%%%%%%%%%%%%%%%%
\section{Analytical Calculation of Simple Nose Curves}
\label{app:analytical-nose-curve}

Some blibla and equations about the analytical calculation of simple nose curves.


%%%%%%%%%%%%%%%%%%%%%%%%%%%%%%%%%%%%%%%%%%
\section{Alternative Current Injection Model}
\label{app:current-injection-model}

\textcite{machowski_2020} describes another way of modeling a \acs{OLTC} transformer with variable ratio.
This model is looking at the shunt brnaches as current injections, which are added to the individual busses.
Beneficial, the system admittance matrix is staying symmetrical, while the different transformer state(s) are represented by the different current injections.
This can be mathematically expressed by following set of equations:
\begin{align}
    \begin{bmatrix}
        \underline{I}_1 \\
        -\underline{I}_2
    \end{bmatrix} &=
    \begin{bmatrix}
        \underline{Y}_\mathrm{T} & -\underline{Y}_\mathrm{T} \\
        -\underline{Y}_\mathrm{T} & \underline{Y}_\mathrm{T}
    \end{bmatrix}
    \begin{bmatrix}
        \underline{U}_1 \\
        \underline{U}_2
    \end{bmatrix} -
    \begin{bmatrix}
        \Delta \underline{I}_1 \\
        \Delta \underline{I}_2
    \end{bmatrix}\text{, where } \notag \\[12pt]
    \begin{bmatrix}
        \Delta \underline{I}_1 \\
        \Delta \underline{I}_2
    \end{bmatrix} &=
    \begin{bmatrix}
        \underline{0} & (\underline{\vartheta}-1)\underline{Y}_\mathrm{T} \\
        -(\underline{\vartheta}^*+1)\underline{Y}_\mathrm{T} & (\underline{\vartheta}^*\underline{\vartheta}+1)\underline{Y}_\mathrm{T}
    \end{bmatrix}
    \begin{bmatrix}
        \underline{U}_1 \\
        \underline{U}_2
    \end{bmatrix} \text{ leading to } \notag \\[12pt]
    \underline{\mab{Y}}_\mathrm{\Pi,T,Current~Injection}&= 
    \begin{bmatrix}
        \underline{Y}_\mathrm{T} & -\underline{Y}_\mathrm{T} \\
        -\underline{Y}_\mathrm{T} & \underline{Y}_\mathrm{T}
    \end{bmatrix} -
    \begin{bmatrix}
        \underline{0} & (\underline{\vartheta}-1)\underline{Y}_\mathrm{T} \\
        -(\underline{\vartheta}^*+1)\underline{Y}_\mathrm{T} & (\underline{\vartheta}^*\underline{\vartheta}+1)\underline{Y}_\mathrm{T}
    \end{bmatrix} \notag % \label{eq:admittance-oltc-2}
\end{align}

% %%%%%%%%%%%%%%%%%%%%%%%%%%%%%%%%%%%%%%%%%%
% \section{OLTC control}


% %%%%%%%%%%%%%%%%%%%%%%%%%%%%%%%%%%%%%%%%%%
% %%%%%%%%%%%%%%%%%%%%%%%%%%%%%%%%%%%%%%%%%%
% \chapter{Verification}

% %%%%%%%%%%%%%%%%%%%%%%%%%%%%%%%%%%%%%%%%%%
% \section{Single-machine infinite bus-bar model}
% \label{app:smib-model}


% %%%%%%%%%%%%%%%%%%%%%%%%%%%%%%%%%%%%%%%%%%
% %%%%%%%%%%%%%%%%%%%%%%%%%%%%%%%%%%%%%%%%%%
% \chapter{Case study}

\end{document}
%------------------------- Ende des Dokuments -----------------------