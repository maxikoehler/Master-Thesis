%%%%%%%%%%%%%%%%%%%%%%%%%%%%%%%%%%%%%%%%%%%%%%%%%%%%%
% Some more used packages
%%%%%%%%%%%%%%%%%%%%%%%%%%%%%%%%%%%%%%%%%%%%%%%%%%%%%
% \usepackage{charter}
% \setkomafont{disposition}{%
% 	\normalfont\bfseries
% }
% \setkomafont{dictum}{\normalfont}                   % Kapitelüberschriften nicht in KOMA Font, sondern in normaler Schriftart
% \renewcommand{\headfont}{\normalfont\sffamily}    		% Kolumnentitel serifenlos
% \renewcommand{\pnumfont}{\normalfont\sffamily}    		% Seitennummern serifenlos

\usepackage{tikz}
\usetikzlibrary{positioning}
\usepackage{tabularx}
\usepackage{amsmath}
\usepackage{amssymb}
\usepackage{float}
% \usepackage{marginnote}
% \usepackage{snotez}
% \setsidenotes{
%   note-mark-format={\hspace*{-2pt}},
%   text-mark-format={\hspace*{-4pt}},
%   text-format+={\RaggedRight\color{ees_blue}\itshape},
%   sidefloat-format={\RaggedRight\color{ees_blue}\itshape\footnotesize},
%   perpage=true
% }
\usepackage{booktabs}
\usepackage{mathtools}
\usepackage{nicematrix}
\usepackage{physics}
\usepackage{enumitem}
\usepackage{csquotes}
\usepackage{ragged2e}
\usepackage{circuitikz}
\usepackage[printonlyused]{acronym}
\usepackage[obeyFinal,backgroundcolor=ees_yellow,linecolor=black, figwidth=.9\linewidth,figcolor=white,textwidth=3cm]{todonotes}
% \usepackage{sidenotes}
\usepackage{lipsum}
\usepackage{tcolorbox}
\usepackage{wrapfig}
\usepackage{xcolor}
\usepackage{rotating}
\usepackage{longtable}
\usepackage{enumitem}	% mehr Optionen bei Aufzählungen
\setlist{itemsep=0pt, parsep=0pt}
\usepackage{graphicx}
\usepackage[right]{eurosym}
\DeclareUnicodeCharacter{20AC}{\euro}
\usepackage{lscape}
\usepackage{bookmark} %nur ein latex-Durchlauf für die Aktualisierung von Verzeichnissen nötig
\usepackage{listings}
\usepackage{comment}
\usepackage{subcaption}


%%%%%%%%%%%%%%%%%%%%%%%%%%%%%%%%%%%%%%%%%%%%%%%%%%%%%
% Citation and bibliobraphy
%%%%%%%%%%%%%%%%%%%%%%%%%%%%%%%%%%%%%%%%%%%%%%%%%%%%%
\usepackage[
	backend=biber,		% empfohlen. Falls biber Probleme macht: bibtex
	bibwarn=true,
	bibencoding=utf8,	% wenn .bib in utf8, sonst ascii
	% sortlocale=en_US,
	sorting=none,
	style=ieee,
]{biblatex}
\setcounter{biburlnumpenalty}{100}
\setcounter{biburlucpenalty}{100}
\setcounter{biburllcpenalty}{100}

\addbibresource{literatur.bib}

%%%%%%%%%%%%%%%%%%%%%%%%%%%%%%%%%%%%%%%%%%%%%%%%%%%%%
% Colors of the Chair
%%%%%%%%%%%%%%%%%%%%%%%%%%%%%%%%%%%%%%%%%%%%%%%%%%%%%
\definecolor{ees_blue}{RGB}{0, 112, 192}
\definecolor{ees_yellow}{RGB}{213, 223, 0}
\definecolor{ees_green}{RGB}{0, 166, 74}
\definecolor{ees_red}{RGB}{192, 0, 0}
\definecolor{ees_lightblue}{RGB}{0, 176, 240}
\definecolor{ees_black}{RGB}{0, 0, 0}

%% Farben (Angabe in HTML-Notation mit großen Buchstaben)
\newcommand{\ladefarben}{%
	\definecolor{LinkColor}{HTML}{00007A}
	% \definecolor{ListingBackground}{HTML}{FCF7DE}
}

%%%%%%%%%%%%%%%%%%%%%%%%%%%%%%%%%%%%%%%%%%%%%%%%%%%%%
% Own commands
%%%%%%%%%%%%%%%%%%%%%%%%%%%%%%%%%%%%%%%%%%%%%%%%%%%%%
\newcommand{\mycomment}{}
\newcommand{\commenting}[1]{{\color{ees_red} #1}} % Kommentar anzeigen
\newcommand{\ai}[1]{{\color{ees_green} #1}} % Kommentar anzeigen
% easy access for comments of missing sources
\newcommand{\quelle}[0]{\commenting{\textbf{[Quelle]}}~}
% Math - vectors/matrices: new command for easy typesetting
\newcommand{\mab}[1]{\mathrm{\textbf{#1}}}

% catching all beauties
\newcommand{\sidenote}[1]{}
\newcommand{\sidecaption}{}
\newcommand{\marginnote}[1]{}
\newcommand{\prefacelogo}{}
% \newenvironment{sidefigure}{}{}%{\begin{sidefigure}}{\end{sidefigure}}

%%%%%%%%%%%%%%%%%%%%%%%%%%%%%%%%%%%%%%%%%%%%%%%%%%%%%
% Preliminary comment on each page
%%%%%%%%%%%%%%%%%%%%%%%%%%%%%%%%%%%%%%%%%%%%%%%%%%%%%
\DeclareNewTOC[%
  owner=\jobname, 
  listname={Appendix},
]{atoc}

\makeatletter
\AfterTOCHead[atoc]{\let\if@dynlist\if@tocleft}% <- gleiches Verhalten (gratuated oder flat) wie toc 
\newcommand*{\useappendixtocs}{%
  \renewcommand*{\ext@toc}{atoc}%
  \scr@ifundefinedorrelax{hypersetup}{}{%
    \hypersetup{bookmarkstype=atoc}%
  }%
  \renewcommand*{\ext@figure}{alof}%
  \renewcommand*{\ext@table}{alot}%
}
\newcommand*{\usestandardtocs}{%
  \renewcommand*{\ext@toc}{toc}%
  \scr@ifundefinedorrelax{hypersetup}{}{%
    \hypersetup{bookmarkstype=toc}%
  }%
  \renewcommand*{\ext@figure}{lof}%
  \renewcommand*{\ext@table}{lot}%
}
\scr@ifundefinedorrelax{ext@toc}{%
  \newcommand*{\ext@toc}{toc}
  \renewcommand{\addtocentrydefault}[3]{%
    \expandafter\tocbasic@addxcontentsline\expandafter{\ext@toc}{#1}{#2}{#3}%
  }
}{}
\makeatother
 
\usepackage{xpatch}
\xapptocmd\appendix{%
%   \addpart{\appendixname}
  \useappendixtocs
  \listofatocs
%   \listofalofs
%   \listofalots
}{}{}


%%%%%%%%%%%%%%%%%%%%%%%%%%%%%%%%%%%%%%%%%%%%%%%%%%%%%
% Preliminary comment on each page
%%%%%%%%%%%%%%%%%%%%%%%%%%%%%%%%%%%%%%%%%%%%%%%%%%%%%
\usepackage{scrtime}
\usepackage{prelim2e}
\renewcommand{\PrelimText}{\textcolor{red}{\textbf{\footnotesize[\today\ at \thistime\ -- preliminary version 0.1]}}}

%%%%%%%%%%%%%%%%%%%%%%%%%%%%%%%%%%%%%%%%%%%%%%%%%%%%%
% Additional stuff
%%%%%%%%%%%%%%%%%%%%%%%%%%%%%%%%%%%%%%%%%%%%%%%%%%%%%
% Hurenkinder und Schusterjungen verhindern
% http://projekte.dante.de/DanteFAQ/Silbentrennung
\clubpenalty = 10000 % schließt Schusterjungen aus (Seitenumbruch nach der ersten Zeile eines neuen Absatzes)
\widowpenalty = 10000 % schließt Hurenkinder aus (die letzte Zeile eines Absatzes steht auf einer neuen Seite)
\displaywidowpenalty=10000

% Bildpfad
\graphicspath{{images/}{../diffpssi-ma-kohler/}}

% Abstände in Tabellen
\setlength{\tabcolsep}{10pt}
\renewcommand{\arraystretch}{1.3}

% Einige häufig verwendete Sprachen
\lstloadlanguages{PHP,Python,Java,C,C++,bash}
%% Programmiersprachen Highlighting (Listings)
\newcommand{\listingsettings}{%
	\lstset{%
		language=Java,			% Standardsprache des Quellcodes
		numbers=left,			% Zeilennummern links
		stepnumber=1,			% Jede Zeile nummerieren.
		numbersep=5pt,			% 5pt Abstand zum Quellcode
		numberstyle=\tiny,		% Zeichengrösse 'tiny' für die Nummern.
		breaklines=true,		% Zeilen umbrechen wenn notwendig.
		breakautoindent=true,	% Nach dem Zeilenumbruch Zeile einrücken.
		postbreak=\space,		% Bei Leerzeichen umbrechen.
		tabsize=2,				% Tabulatorgrösse 2
		basicstyle=\ttfamily\footnotesize, % Nichtproportionale Schrift, klein für den Quellcode
		showspaces=false,		% Leerzeichen nicht anzeigen.
		showstringspaces=false,	% Leerzeichen auch in Strings ('') nicht anzeigen.
		extendedchars=true,		% Alle Zeichen vom Latin1 Zeichensatz anzeigen.
		captionpos=b,			% sets the caption-position to bottom
		% backgroundcolor=\color{ListingBackground}, % Hintergrundfarbe des Quellcodes setzen.
    backgroundcolor=\color{white!50},
		xleftmargin=0pt,		% Rand links
		xrightmargin=0pt,		% Rand rechts
		frame=single,			% Rahmen an
		frameround=ffff,
		rulecolor=\color{black},	% Rahmenfarbe
		% fillcolor=\color{ListingBackground},
    fillcolor=\color{ees_blue!50},
		keywordstyle=\color[rgb]{0.133,0.133,0.6},
		commentstyle=\color[rgb]{0.133,0.545,0.133},
		%stringstyle=\color[rgb]{0.627,0.126,0.941}
		stringstyle=\color{red}
	}
}
\listingsettings{}
% Umbennung des Listings
\renewcommand\lstlistingname{Listing}
\renewcommand\lstlistlistingname{List of Listings}
\def\lstlistingautorefname{Listing}

\lstdefinestyle{style-python}
{
  language=python,	% Programmiersorache einstellen, Latex erkennt dann automatisch code, kommentare, funktionen,...
  basicstyle=\scriptsize\ttfamily,	% Schriftformatierung, ttfamily: text im Schreibmaschinen-Style
  backgroundcolor=\color{white},		% Hintergrundfarbe
  breaklines=true,	% automatischer Zeilenumbruch bei langen Zeilen, funktioniert nur bei bestimmen Zeichen, z.B. Umbruch nach Leerzeichen
  keywordstyle=\bfseries\ttfamily\color{blue},	% Schlüsselwörter einstellen
  stringstyle=\ttfamily\color{ees_yellow},			% Textsrtrings
  showstringspaces=false,	% Leerzeichen in Strings richtig darstellen
  commentstyle=\color{ees_green}\ttfamily,	% Kommentare
  flexiblecolumns=false,	% Spaltenbreite dynamisch/fest
  numbers=left,		% Position der Zeilennummern
  numberstyle=\tiny,	% Größe der Zeilennummern
  numberblanklines=false,		% leere Zeilen werde mit ‚false‘ nicht durchnummeriert
  stepnumber=1,		% Beginn der Nummerierung
  numbersep=10pt,		% Abstand zwischen Zeilennummern und Quellcode
  xleftmargin=20pt,	% Abstand zum linken Rand
  xrightmargin=10pt,	% Abstand zum rechten Rand
  extendedchars=true,	% Sonderzeichen korrekt darstellen
  frame=trbl,			% Rahmen um gesamten Code: Top, right, bottom, left (Großbuchstaben ergeben Doppellinien)
  frameround=ffff,	% Ecken des Rahmens anpassen, t: runde Ecken, f: default (eckig), es müssen 4 Buchstaben da stehen!
  literate=		% ersetzen von Zeichen 1 durch Zeichen 2, hier: korrekte Einbindung der Sonderzeichen
   {Ö}{{\"O}}1 
   {Ä}{{\"A}}1 
   {Ü}{{\"U}}1 
   {ß}{{\ss}}1 
   {ü}{{\"u}}1 
   {ä}{{\"a}}1 
   {ö}{{\"o}}1,
  % mit ‚emph‘ und ‚emphstyle‘ können eigene Styles für Wörter angelegt werden
  emph = [1]{clc, color},
  emphstyle = [1]{\color{blue}},
  emph = [2]{function, endfunction},
  emphstyle = [2]{\color{ees_red}},
  emph = [3]{gcf},
  emphstyle = [3]{\color{black}},
}

%%%%%%%%%%%%%%%%%%%%%%%%%%%%%%%%%%%%%%%%%%%%%%%%%%%%%
% Testing siome stuff
%%%%%%%%%%%%%%%%%%%%%%%%%%%%%%%%%%%%%%%%%%%%%%%%%%%%%
\newcommand{\einstellung}[1]{%
  \expandafter\newcommand\csname #1\endcsname{}
  \expandafter\newcommand\csname setze#1\endcsname[1]{\expandafter\renewcommand\csname#1\endcsname{##1}}
}
\newcommand{\langstr}[1]{\einstellung{lang#1}}
\einstellung{offset}
\setzeoffset{100mm}

\definecolor{quotemark}{gray}{0.7}
\makeatletter
\newlength\origparskip

% \newcommand{\fquote}{}

\newcommand{\fquote}{%
  \@ifnextchar[{\fquote@i}{\fquote@i[]}%]
}

\def\fquote@i[#1]{%
  \@ifnextchar[{\fquote@ii{#1}}{\fquote@ii{#1}[]}%]
}%

\def\fquote@ii#1[#2]{%
  \def\pqm@tempa{#1}%
  \def\pqm@tempb{#2}%
  \noindent
  \list
    {}
    {\setlength{\leftmargin}{0.3\textwidth}%
     \setlength{\rightmargin}{0.1\textwidth}%
     \setlength{\origparskip}{\parskip}}%
    \item[]%
      \begin{picture}(0,0)%
        \put(-15,-8){\makebox(0,0){\scalebox{4}{%
          \textcolor{ees_blue}{\textquotedblright}}}}%
      \end{picture}%
      \begingroup
      \itshape
      \ignorespaces}%

\def\endfquote{%
  \endgroup
  \par
  \raggedleft
  \ifx\pqm@tempa\empty
  \else
    {\bfseries --- \pqm@tempa\par}%
    \setlength{\parskip}{\origparskip}%
    \ifx\pqm@tempb\empty
    \else
      (\pqm@tempb)%
    \fi
  \fi
  \par
  \endlist}
\makeatother

% circles as enumeration
\newcommand*\circledblue[1]{\tikz[baseline=(char.base)]{%
            \node[shape=circle,fill=ees_blue!40,draw,inner sep=2pt] (char) {#1};}}
\newcommand*\circled[1]{\tikz[baseline=(char.base)]{%
            \node[shape=circle,fill=white,draw,inner sep=2pt] (char) {#1};}}