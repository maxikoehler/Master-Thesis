%!TEX root = ../main.tex

% \begingroup
% \newgeometry{left=2.5cm,right=2.5cm,top=2.5cm,bottom=2.5cm}
% \part{Practical Application: Simulation}

% \endgroup

%%%%%%%%%%%%%%%%%%%%%%%%%%%%%%%
%%%%%%%%%%%%%%%%%%%%%%%%%%%%%%%
\chapter{Application Study}
\label{chap:case-study}

\begin{textblock*}{.7\textwidth}(70mm-\offset,25mm-\offset)
    \begin{fquote}[Narcotics Anonymous]
        Insanity is doing the same thing, over and over again, but expecting different results.
    \end{fquote}
\end{textblock*}

This part of the thesis aims to apply the developed and implemented model to different scenarios in possible interest.
Ideas, hat have come up during the implementation and validation of the models, shall be picked up here and used as a demonstration for \textit{diffpssi}
On the other hand they can give more hints and discussion potential for the research questions targeting the \acs{FSM}.
The constituton of the following sections is differing compared to the rest of the thesis.
Each subsection will look at a specific idea or area, where a hypothesis or expectation is presented, the simulation set-up described, and the results drawn.
This deviation targets not to mix the cases or ideas up and to strive for a more consistent argumentation.

%%%%%%%%%%%%%%%%%%%%%%%%%%%%%%%
\section{Voltage Stability after a Short-Cicuit}
\label{sec:case-1}

\sidenote{Used setup}
Exploring the basic behavior and differences between the tap changer control scheme is best possible with simplifyied set-ups.
As the validation in \autoref{sec:validation-oltc-schemes} is already considering the most simple networks with only a load and a generator as source, this part is taking another machine as external grid into consideration.
This means, that the standard \acs{SMIB} model from \autoref{sec:networks} is used, all differing presets are shown in \autoref{tab:case1-parameters}.
The applied transformer in particular, is set with the standard parameters for each control from \autoref{tab:control-params-oltc} and \autoref{tab:control-params-fsm}.
One important deviation from the transformer installation before, is the bus of measuremens.
As the active and voltage dependent unit is located on bus 2, the \acs{LV} bus, this is set at the controlled one.
The measurement at this bus is used as input for the controller.
Following control schemes are part of the comparisons:
\newpage
\begin{enumerate}
    \item Standard Transformer without a control scheme
    \item Discrete \acs{OLTC} control
    \item Voltage dependent \acs{FSM} control
    \item \acs{FSM} preferring control
\end{enumerate} 

\begin{table}[htbp!]
    \caption[Parameter set for case study two]{Parameter set for case study two; all other, not described parameters stay constant to the reference network in \autoref{sec:networks}.}
    \label{tab:case1-parameters}
    \vspace*{12pt}
    \centering
    \small
    \begin{tabularx}{\textwidth}{Xrl}
        % \toprule
        \textbf{Parameter} & \textbf{Value} & \textbf{Unit} \\ \hline
        \toprule
        Generator inertia $H$ & 3.5 & s \\
        Generator damping $D$ & 0.1 & p.u. \\
        Generator resistance $R$ & 0.01 & p.u. \\
        Generator reactance $X$ & 0.1 & p.u. \\
        Transformer resistance $R$ & 0.01 & p.u. \\
        Transformer reactance $X$ & 0.1 & p.u. \\
        Transmission line resistance $R$ & 0.01 & p.u. \\
        Transmission line reactance $X$ & 0.1 & p.u. \\
        \bottomrule
    \end{tabularx}
\end{table}

\sidenote{Added event}
The considered event for this setup and scenario is a three phase short-circuit, placed at bus one for the time steps between $1$ s and $1.07$ s.
As the stability of the bus voltages can also be harmed after a longer period of time, a simulation time for $120$ s is applied.
Also, this time period should indicate tap hunting, if it is occuring.

\sidenote{Postulation}
First, looking at the bus voltages alone and neglecting other factors is central.
The hypothesis, that the \acs{FSM} can support the voltage stability of the system for longer is tested.
As the \acs{OLTC} control can vary and thus support, an improved \acs{TDS} should be visible. 
The \acsp{FSM} have a bigger ratio range, and can switch faster.
Dependent on the used scheme also with a smaller deadband as the \acs{OLTC}, why these should increase the time voltage can be held in the relevant band even more.  

\begin{figure}[htbp!]
    \centering
    \includegraphics[width=14cm]{development_files/case_studies/plots/case1_tds_bus_voltages.pdf}
    \caption[\acs{TDS} for case study one split into busses for comparing the control schemes]{\acs{TDS} for case study one split into busses for comparing the control schemes; The black dashed lines account for an \acs{TVI} envelope with the parameters $\beta=0.1$,  $v_\mathrm{st}=0.9\text{ p.u.}$, and $t_\mathrm{f}=1.06\text{ s}$.}
    \label{fig:case1-voltages}
\end{figure}

% \newpage
\sidenote{Results}
Looking into the first plot of results, \autoref{fig:case1-trans-ratio} is showcasing the completely different reaction of the control schemes on the same scenario.
The discrete \acs{OLTC} controller is slowly reacting, after the machine is not swinging any more.
From the initial state, the transformer ratio is reduced to its maximum end position with a ratio of $u_\mathrm{l}=0.9\text{ p.u.}$.
Both \acs{FSM} modules jump to the end position just a few milliseconds in the fault, while letting the respective \acs{OLTC} parts in the inital states, accounting for a ratio of $u_\mathrm{l}=0.84\text{ p.u.}$.
After that, depending on the enabling functions and tap skipping, the voltage dependent control is switching solely with the \acs{OLTC} part back to higher ratios.
The \acs{FSM} preferring control uses the \acs{FSM} for two upwards switches before enabling and using the \acs{OLTC} part.
Interestingly, both result in a stable state at least until the end of the simulation time, but with different ratios.
While the voltage dependent control stays static after aronud $50\text{ s}$ with a ratio of $u_\mathrm{l}=1.04\text{ p.u.}$, the \acs{FSM} preferring control enters a mode called tap hunting.
This means the control is switching back and forth, and is kind of oscillating around a middle transformer ratio.

\begin{figure}[htbp!]
    \centering
    \includegraphics[width=\linewidth]{development_files/case_studies/plots/case1_transformer_ratio.pdf}
    \caption[\acs{TDS} for case study one of the transformer ratios for comparing the control schemes]{\acs{TDS} for case study one of the transformer ratios for comparing the control schemes.}
    \label{fig:case1-trans-ratio}
\end{figure}

\begin{table}[htbp!]
    \caption[Time stamps for the first envelope cuts of each bus and controller in case study one]{Time stamps for the first envelope cuts of each bus and controller in case study one; values in the unit s.}
    \label{tab:case1-critical-times}
    \vspace*{12pt}
    \centering
    \small
    \begin{tabularx}{\textwidth}{Xrrr}
        % \toprule
        \textbf{Control Scheme} & \textbf{Bus 0} & \textbf{Bus 1} & \textbf{Bus 2} \\ \hline
        \toprule
        Simple Transformer                  & 29.520 & 29.030 & 28.430 \\
        Discrete \acs{OLTC} Control         & 45.905 & 44.810 & 52.525 \\
        Voltage Dependent \acs{FSM} Control & 115.795 & 112.535 & - \\
        \acs{FSM} preferring Control        & - & - & -  \\
        \bottomrule
    \end{tabularx}
\end{table}

% For triggering voltage instabilities, short circuits have to be near the loads \autocite{cutsem_1998}.
% They are always set to the individual load busses at the given systems.

% The scenarios then can be quantified and compared, especially considering voltage stability, with the \acs{TVI} and calculation of critical times.

\sidenote{Transformer ratios}
Looking into the first plot of results, \autoref{fig:case1-trans-ratio} is showcasing the completely different reaction of the control schemes on the same scenario.
The discrete \acs{OLTC} controller is slowly reacting, after the machine is not swinging any more.
From the initial state, the transformer ratio is reduced to its maximum end position with a ratio of $u_\mathrm{l}=0.9\text{ p.u.}$.
Both \acs{FSM} modules jump to the end position just a few milliseconds in the fault, while letting the respective \acs{OLTC} parts in the inital states, accounting for a ratio of $u_\mathrm{l}=0.84\text{ p.u.}$.
After that, depending on the enabling functions and tap skipping, the voltage dependent control is switching solely with the \acs{OLTC} part back to higher ratios.
The \acs{FSM} preferring control uses the \acs{FSM} for two upwards switches before enabling and using the \acs{OLTC} part.
Interestingly, both result in a stable state at least until the end of the simulation time, but with different ratios.
While the voltage dependent control stays static after aronud $50\text{ s}$ with a ratio of $u_\mathrm{l}=1.04\text{ p.u.}$, the \acs{FSM} preferring control enters a mode called tap hunting.
This means the control is switching back and forth, and is kind of oscillating around a middle transformer ratio.

\sidenote{Bus voltages}
The afore obtained switches of the tap changers are visible in the \acs{TDS} plots of the bus voltages as well.
\autoref{fig:case1-voltages} is showing the control schemes compared through the bus voltages in the subplots.
A voltage envelope is inserted with a dashed black line, considered for the parameters of $\beta=0.1$,  $v_\mathrm{st}=0.9\text{ p.u.}$, and $t_\mathrm{f}=1.06\text{ s}$.
Here, the afore described characteristics of tap hunting of the \acs{FSM} preferring control is observable.
Further, all tap changer controls indicate an improvement in stability.
While the base scenario without any tap changer control is exceeding the envelope at around $29\text{ s}$, the discrete \acs{OLTC} al already increasing the time to around $45\text{ s}$. 
The \acs{FSM} controlled tap changers are resulting in stable bus voltages over the complete simulation, while the voltage dependent control is violating the envelope only at around $112\text{ s}$.
The exact results are included in \autoref{tab:case1-critical-times}.

\begin{table}[htbp!]
    \caption[Results for the \acs{TVI} and the \acs{CSI} for case study one]{Results for the \acs{TVI} and the \acs{CSI} for case study one; Values in the unit $\text{p.u.} \cdot \text{s}$, or short just s.}
    \label{tab:case1-tvi}
    \vspace*{12pt}
    \centering
    \small
    \begin{tabularx}{\textwidth}{Xrrrr}
        % \toprule
        \textbf{Control Scheme} & \textbf{Bus 0} & \textbf{Bus 1} & \textbf{Bus 2} & \textbf{\acs{CSI}} \\ \hline
        \toprule
        Simple Transformer                  & 36.19 & 42.50 & 135.36 & 71.35 \\
        Discrete \acs{OLTC} Control         & 31.51 & 31.39 & 98.03 & 53.64 \\
        Voltage Dependent \acs{FSM} Control & 0 & 0 & 92.7 & 30.9 \\
        \acs{FSM} preferring Control        & 0 & 0 & 92.7 & 30.9 \\
        \bottomrule
    \end{tabularx}
\end{table}

\sidenote{TVI and CSI index}
Accounting for the \acs{TVI} values in \autoref{tab:case1-tvi}, this visual based statement is supported as well.
The most critical scenario is represented with us of the simple transformer, with a \acs{CSI} of over $70\text{ s}$.
Both \acs{FSM} controls result in the most stable with a \acs{CSI} of around $30\text{ s}$, while the discrete \acs{OLTC} is in the middle with around $50\text{ s}$.
Since this is the evaluation of the whole system, the values for the single busses are also calculated before.
Here, in every of the four scenarios or control types, bus three is the most critical, with always being the only one above the \acs{CSI}.

%%%%%%%%%%%%%%%%%%%%%%%%%%%%%%%
\section{Interaction with Machines without Control}
\label{case-2}

\sidenote{Set-up, scenario and general}
For the second investigation, looking into the interaction with synchronous machines, the same simulation setup as in case study one is chosen.
Including two machines, and all four control possiblilities, this scenario including the short-circuit event shall indicate a power and therefore machine rotor swinging.

The small differences in the presented \acs{FSM} controls, and the \acs{OLTC} control are expected to have different impacts on this topic.
Short circuit events demand a quick response, thus small changes or shifts in timing of switches or different enabling functions can have a significant influence.

\commenting{  
    Bus voltages shortly after the short circuit.
    
    Machine Speeds, electrical Power, shortyl after the short circuit.
}

\sidenote{Postulation}
The second expectation of interest is the system reaction, specifically the behavior of machines, after a short cicuit event.
Considering the different time constants and enabling functionalities of the \acs{FSM} controllers, or the standard \acs{OLTC} controller, applied to the grid, are expected to trigger differing reponses.
Especially because these shorter time constants protrude into the typical voltage swings of machines after a short circuit event.
As the active power transfer from a bus to another can be expressed as denoted in \autoref{eq:func-p-transfer}, the variable ratio transformer is expected to manipulate this power transfer, and resp. have an influence on the machine swings.

\sidenote{Results}

\begin{figure}[htbp!]
    \centering
    \includegraphics[width=\linewidth]{development_files/case_studies/plots/case2_tds_voltages_zoomed.pdf}
    \caption[\acs{TDS} for case study two split into busses for comparing the control schemes]{\acs{TDS} for case study two split into busses for comparing the control schemes; The result has been zoomed to the limits $x \in [0,8]\text{ s}$ and $y \in [0.7,1.2]\text{ p.u.}$.}
    \label{fig:case2-voltages}
\end{figure}

\begin{figure}[htbp!]
    \centering
    \includegraphics[width=\linewidth]{development_files/case_studies/plots/case2_power_electrical.pdf}
    \caption[\acs{TDS} for case study two considering the electrical power]{\acs{TDS} for case study two considering the electrical power $P_\mathrm{e}$ for comparing the control schemes; considered machine is at bus two; the result has been zoomed to the limits $x \in [0,8]\text{ s}$ and $y \in [-0.1,1.5]\text{ p.u.}$.}
    \label{fig:case2-power}
\end{figure}

\begin{figure}[htbp!]
    \centering
    \includegraphics[width=\linewidth]{development_files/case_studies/plots/case2_machine_speed.pdf}
    \caption[\acs{TDS} for case study two considering the machine speed]{\acs{TDS} for case study two considering the machine speed $\omega$ for comparing the control schemes; considered machine is at bus two; the result has been zoomed to the limits $x \in [0,8]\text{ s}$ and $y \in [-0.01,0.01]\text{ p.u.}$.}
    \label{fig:case2-speed}
\end{figure}

%%%%%%%%%%%%%%%%%%%%%%%%%%%%%%%
\section{Investigation on Controller Interaction}
% \label{sec:case-2}

But this is only possible to happen, if the controllers response time is short enough.

\commenting{
    Look into interactions between machine controllers and OLTC controllers.
}

\commenting{
    Thinking of Rotor Angle Stability -> also looking into omega and delta\\
    What does the fast switching with large magnitudes do to the machines?\\
    Or what does the switching to the swinging of machines?\\
    Thinking of the damping effect of the FSM, especially in the case of power swings, caused by shortages and following swinging of Generators.\\
    Comparison between standard Transformer / OLTC and FSM control.
}

For the simulation set-up, the in \autoref{sec:networks} described \acs{SMIB} model was used.
The network parameters from \autoref{tab:case2-parameters} apply.
The short circuit is placed on bus one, from the time step $1$ s to $1.06$ s.
For the used \acs{OLTC} and \acs{FSM} control used parameter sets are described in \autoref{tab:case2-control-params}.

\ai{
    \begin{enumerate}[nosep]
        \item Control with voltage difference dependent activation not sufficient, but the FSM preferred one as well -> One has a too big dead band due to the tap skips, the other one is taking too long until the OLTC is activated
        \item Increase of stability possible
        \item Somehow damping, although the machine swings are not damped
    \end{enumerate}
}

% %%%%%%%%%%%%%%%%%%%%%%%%%%%%%%%
% \section{Investigation on Control Parameter Variations}
% \label{sec:case-2-1}

% \commenting{
%     \end{itemize}
%     Falling under parameter variation: 
%     \begin{itemize}[nosep]
%         \item Influence of max. ratio change per switching event, and
%         \item Influences of faster response, bigger voltage range, etc.
%         \item How sturdy is the control scheme to such variations?
%     \end{itemize}
% }

%%%%%%%%%%%%%%%%%%%%%%%%%%%%%%%
\section{Novel Control Strategy Aspects to a FSM Control}
% \label{sec:case-3}

\commenting{
    Thinking of fast and slow voltage gradients: fast gradients are compensated by the FSM, slow gradients are compensated by the OLTC. 
    Therefore optimal utilisation of injected damping moment of the FSM. 
    
    Also thinking of different presets of the OLTC and FSM, which are tried to keep constant. 
    Different grid operators can utilize for typical grid conditions of over- or undervoltage at \acs{PCC}.
    
    Following contains:
    \begin{itemize}
        \item Implemetation of different logic
        \item Testing of presets and switchin logic
        \item Damping moment beneficial?
    \end{itemize}
}

\subsection{Alternative Tap Skipping Logics}

\subsection{Voltage Difference Based Time Constants}

\subsection{Voltage Gradient Based Time Constants}



\subsubsection{Objectives and Expectations}

\subsubsection{Simulation Set Up}

\subsubsection{Results}

\subsubsection{Deductions and Futher Improvements}

% %%%%%%%%%%%%%%%%%%%%%%%%%%%%%%%



%%%%%%%%%%%%%%%%%%%%%%%%%%%%%%%
\section{Summary in Short and Simple Terms}

Some blibla.