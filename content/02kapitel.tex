%!TEX root = ../main.tex

\begingroup
\newgeometry{left=2.5cm,right=2.5cm,top=2.5cm,bottom=2.5cm}
\part{Methodical}

\endgroup
% A room without books is like a body without a soul.



%%%%%%%%%%%%%%%%%%%%%%%%%%%%%%%
%%%%%%%%%%%%%%%%%%%%%%%%%%%%%%%
\chapter{Transformer Equipment Modeling}

\begin{textblock*}{.7\textwidth}(70mm,25mm)
        \begin{fquote}[Mae West]
                You only live once, but if you do it right, once is enough.
        \end{fquote}
\end{textblock*}

Some literature and fundamentals about transformers, control, stability assessment, fast-switching modules, and analysis in Python. \todo{Hier steht ein Beispielkommentar.}

% \begin{figure}[H]
%         \centering
%         \vspace{1cm}
%         \begin{circuitikz}[european, scale=.9, smallR/.style={resistor,resistors/scale=.7}]
%                 \draw (0,0) node[oscillator, anchor=east, name=gen]{} --(.5,0)
%                 to[L, name=X_g] ++(2,0) coordinate(f1)
%                 % \bushere{1}{$\underline{E}_\mathrm{T}'$}{}
%                 % to[oosourcetrans,prim=delta,sec=wye] ++(2,0)
%                 \bushere{2}{$\underline{V}_\mathrm{bb}$}{\acs{SG} bus bar} coordinate(b1) ++(0,-1) -- ++(.25,0) -- coordinate(f2) ++(1.75,0)
%                 to[L, name=X_l3] ++(1,0) -- ++(1.5,0) ++(0,1)
%                 \bushere{2}{$E_\mathrm{ibb}~\angle~0^{\circ}$}{\acs{IBB}} ++(0,1) coordinate(b2) ++(0,-1) coordinate(b3) -- ++(.75,0) coordinate(f3) -- ++(.75,0) to[L, name=X_ibb] ++(1,0) -- ++(1,0)
%                 node[gridnode, anchor=left, name=ib]{};

%                 % draw other resistances
%                 \draw (b1) ++(0,1) -- ++(.5,0) to[L, name=X_l1] (b2);
%                 \draw (b1) -- ++(.5,0) to[L, name=X_l2] (b3);
%                 % \draw[line width=2pt] (2.25,1) -- (2.25,-1);
%                 % \draw[line width=2pt] (4.75,1) -- (4.75,-1);
%                 % \draw[line width=2pt] (8.25,1) -- (8.25,-1);

%                 % labels for the resistors
%                 \node[above=6pt] at (X_g) {$X_\mathrm{g}'$};
%                 \node[above=6pt] at (X_ibb) {$X_\mathrm{ibb}$};
%                 \node[above=6pt] at (X_l1) {$3~X_\mathrm{l}$};
%                 % \node[below=6pt] at (X_l2) {$X_\mathrm{l}$};
%                 % \node[below=6pt] at (X_l3) {$X_\mathrm{l}$};

%                 % pole voltages and angles
%                 \path[->] (-1.2,.5) edge [bend right] node[left=6pt]{$E_\mathrm{p}~\angle~\delta$} (-1.2,-.5);
%                 % \path[->] (ib) ++(.8,.5) edge [bend left] node[right=6pt]{$E_\mathrm{ibb}~\angle~0^{\circ}$} ++(0,-1);

%                 % faults
%                 % \draw[-Stealth, very thick, red] (f1) ++(0,-.5) -- ++(-.15,-.45) -- ++(.3,.2) -- ++(-.2,-.6) coordinate(f1_text);
%                 % \node[below, red] at (f1_text) {\scriptsize fault 1};
%                 \draw[-Stealth, very thick, red] (f2) ++(0,.3) -- ++(-.15,-.45) -- ++(.3,.2) -- ++(-.2,-.6) coordinate(f2_text);
%                 \node[below, red, align=center] at (f2_text) {\scriptsize fault 2/3};
%                 \draw[-Stealth, very thick, red] (f3) ++(0,.3) -- ++(-.15,-.45) -- ++(.3,.2) -- ++(-.2,-.6) coordinate(f3_text);
%                 \node[below, red] at (f3_text) {\scriptsize fault 1};
%         \end{circuitikz}
%         \vspace{.5cm}
%         \caption[Representative circuit of a \acf{SMIB} model]{Representative circuit of a \acf{SMIB} model with pole wheel voltage $E_\mathrm{p}~\angle~\delta$ and \acf{IBB} voltage $E_\mathrm{ibb}~\angle~0^{\circ}$; positions of considered faults 1 to 3 are marked with red lightning arrows}
%         \label{fig:smib-model}
% \end{figure}

\begin{figure}[h]
        \centering
        \begin{tikzpicture}[node distance = 1.5cm, auto]
                % Place nodes
                \node [papStart] (start) {start};
                \node [papProcess, below of = start, yshift= -2mm] (pro1){initialize start and simulation parameters};
                \node [papPredProc, below of = pro1, yshift= -5mm] (pro2){\nodepart{two} \shortstack{do \acs{TDS} with\\constant fault}};
                \node [papProcess, below of = pro2, yshift= -5mm] (pro3){determine \acs{CCT} and critical angle};
                \node [papPredProc, below of = pro3, yshift= -7mm](pro4){\nodepart{two}\shortstack{solving \acs{TDS}\\stable and unstable}};
                \node [papProcess, below of = pro4, yshift= -5mm](pro5){plot results};
                \node [papEnd, below of = pro5, yshift= -2mm] (end) {end};

                \node [right of = pro4, xshift=25mm] (iter) {\footnotesize\itshape{iterate 1x}};
                \node [right of = pro1, xshift=25mm] () {\footnotesize\itshape{init()}};
                \node [right of = pro2, xshift=25mm] () {\footnotesize\itshape{odeint()}};
                \node [right of = pro3, xshift=25mm] () {\footnotesize\itshape{determine\_cct()}};

                % Place joins
                \coordinate [above = of pro4, yshift= -9mm] (join1);
                \coordinate [below of = pro4, yshift= 5mm] (join2);

                % Draw edges
                \path [papLine] (start) -- (pro1);
                \path [papLine] (pro1) -- (pro2);
                \path [papLine] (pro2) -- (pro3);
                \path [papLine] (pro3) -- (pro4);
                \path [papLine] (pro4) -- (pro5);
                \path [papLine] (pro5) -- (end);
                \path [papLine] (join2) -| (iter);
                \path [papLine] (iter) |- (join1);
                % \path [papLine] (dec1) -- node [above] {\papNo} (predproc1);
                % \path [papLine] (predproc1) -- (pro3);
                % \path [papLine] (pro3) |- (join1);
        \end{tikzpicture}
        \caption[Program plan proposal for determining the \acf{CCT}]{Program plan proposal for determining the \acf{CCT} $t_\mathrm{cc}$, critical power angle $\delta_\mathrm{cc}$ and the \acf{TDS} of the \acf{SMIB}-model; including the associated main function name}
        \label{fig:program-plan}
\end{figure}

%%%%%%%%%%%%%%%%%%%%%%%%%%%%%%%
\section{Current implementation of transformers}

\commenting{Describe the current implementation of transformers in the Python framework.}

%%%%%%%%%%%%%%%%%%%%%%%%%%%%%%%
\section{Dynamic behavior of transformers}

\commenting{This is the description of the \glqq new\grqq implementation.}

\subsection{Model Demands and Changes in the Framework}

\subsection{Additional Modifications through a Fast Switching module}

%%%%%%%%%%%%%%%%%%%%%%%%%%%%%%%
\section{Tap Changer Control Modeling}

\commenting{This is the description of the ideas, development, and implementation of a OLTC control scheme.}

\subsection{Discrete Control Loop}
This control method represents the currently most used and thus representative control scheme for \acsp{OLTC}. With the mechanic nature of the switching mechanism, the control look can only access discrete ratios within time frames of around a few seconds. Such a discrete control loop is described by \textcite{milanoHybridControlModel2011}. A scheme of this control loop is shown in \autoref{fig:discrete-control-loop}.

\begin{figure}[htb!]
        \centering
        \missingfigure{Discrete control loop}
        \caption{Discrete control loop of an \acs{OLTC}; scheme based on \textcite{milanoHybridControlModel2011}}
        \label{fig:discrete-control-loop}
\end{figure}

This control loop type is beneficial due to its accurate representability of current \acs{OLTC} abilities. It gains access to assess stability within simulation environments, as analytical methods are not suited.

A negative aspect of a discrete control loop is the missing opportunity of generating a transfer function. This blocks the stability assessment with standard control engineering methods. Further, popular analysis methods like eigenvalue analysis is not possible, due to the lack of possibility to form derivatives.

\commenting{
        \begin{itemize}
                \item Describe implementation
                \item Describe benefits / drawbacks
                \item Control scheme
                \item Switching logic and behavior (voltage tracking)
        \end{itemize}
}

\subsection{Continous Control Loop}

\subsection{Control Schemes for the Fast Switching module}

\subsubsection{Discrete Control Loop as most Representative}
A continous control loop for a \acs{FSM} is presented within \textcite{burlakinEnhancedVoltageControl2024,burlakinEnhancingVariableShunt2024}. Similar to the solely \acs{OLTC} loop, it represents the real behavior best, but is obstructive for stability assessments. The scheme of the logic is shown in \autoref{fig:fsm-continuous-control-loop}.

\begin{figure}[htb!]
        \centering
        \missingfigure{Continuous control loop of a FSM}
        \caption{Continuous control loop of a \acs{FSM}; scheme based on \textcite{burlakinEnhancedVoltageControl2024}}
        \label{fig:fsm-continuous-control-loop}
\end{figure}

\subsubsection{Continuous Control Loop for best Stability Assessment}
