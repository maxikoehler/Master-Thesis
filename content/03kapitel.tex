%!TEX root = ../main.tex

% \part{Enhancing the System Stability Assessment}

%%%%%%%%%%%%%%%%%%%%%%%%%%%%%%%
%%%%%%%%%%%%%%%%%%%%%%%%%%%%%%%
\chapter{Application of Voltage Stability}

% \begin{textblock*}{.7\textwidth}(70mm,25mm)
%         \begin{fquote}[Albert Einstein]
%             All models are wrong, but some are useful.
%         \end{fquote}
%     \end{textblock*}

%%%%%%%%%%%%%%%%%%%%%%%%%%%%%%%
\section{Influences of other device characteristics}

\commenting{
        Just look on other mutual influences in the power system (simulation), such as:
        \begin{itemize}
                \item Load characteristics and types of modeling
                \item Maximum thermal currents of cables and operating components
                \item Asynchronous machines (or called \glqq induction motors\grqq?)
        \end{itemize}
}

%%%%%%%%%%%%%%%%%%%%%%%%%%%%%%%
\section{Observing the current state of the system}

\subsection{Static and Dynamic Indices}

\commenting{
        \begin{itemize}
                \item Which indices can be implemented?
                \item Which make sense?
                \item Implementation and calculation of them?
        \end{itemize}
}

\subsection{Stability Monitoring}

\commenting{
        \begin{itemize}
                \item Index combination and \glqq traffic light\grqq\~monitoring
                \item Restauration options and opportunities
                \item Local mapping
                \item Weak point identification
        \end{itemize}
}

%%%%%%%%%%%%%%%%%%%%%%%%%%%%%%%
\section{Wide-area control mechanisms}

\commenting{
        \begin{itemize}
                \item What influences could an interconnected information system have on curretn \glqq dumb transformer control\grqq?
                \item Reference voltages usually come from load flow analysis out of the back office (day-ahead); How can this be changed? How can transformers get more \glqq smart\grqq?
        \end{itemize}
}