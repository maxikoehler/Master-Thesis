%!TEX root = ../main.tex

% \addtocontents{toc}{\protect\addvspace{2.25em}}
% \bookmarksetup{startatroot}

%%%%%%%%%%%%%%%%%%%%%%%%%%%%%%%
%%%%%%%%%%%%%%%%%%%%%%%%%%%%%%%
\chapter{Discussion of the Results}
\label{chap:discussion}

\begin{textblock*}{.7\textwidth}(70mm-\offset,25mm-\offset)
    \begin{fquote}[Joseph Joubert]
        The aim of argument, or discussion, should not be victory, but progress.
    \end{fquote}
\end{textblock*}

This chapter is discussing all chapters combined, with found aspects, expected or unexpected behaviors, comparisons and so on.
Singular discussions in each chapter are avoided, to maintain a high level view on the \acf{FSM}.
Nevertheless, details do matter here and are included in the summarized evaluation.
The structure of this chapter is not holding up with the afore used structure.
This shall make use of the combined discussion, more orientating on the found aspects, than the done work. 

%%%%%%%%%%%%%%%%%%%%%%%%%%%%%%%
\section{Evaluation Current FSM Control}

\commenting{
    -> Basically discussion of the Application Study:
    \begin{itemize}[nosep]
        \item Achieved stabilization in the applied case with OLTC and FSM
        \item Do I have longer time to act on this based on the enhanced functionality or just because of the bigger band?
        \item Back propagation of fast switches to machines.
        \item Interactions with other controllers? E.g. Machine controllers.
        \item Deaf band?
    \end{itemize}
}

% \subsubsection{Comparison FSM vs. OLTC}

% \subsubsection{Conspicuities of the FSM Control}

    
%%%%%%%%%%%%%%%%%%%%%%%%%%%%%%%
\section{Diffpssi Integration}

\commenting{
    Possible uses of diffpssi:
    \begin{itemize}[nosep]
        \item Main point of diffpssi: \textbf{DIFFERENTIABLE}: Then making use of it? Why not optimize the FSM control aspects with it?
    \end{itemize}
    Restrictions of the models:
    \begin{itemize}[nosep]
        \item 
        \item One key missing model / aspect, especially for short term voltage stability in combination with stability support of the FSM: \textbf{INDUCTION MOTORS}.\\
        \item This thesis is only presenting a control circuit for standard OLTCs, and their extensional FSM. 
        \item Line Drop Compensations are not included, WAMPAC methods are excluded, parallel transformer realizations (N-1 criteria; redundancies), are not considered.
        \item For recent developments, \textcite{sarimuthu_2016} is presenting a review paper considering a few of these topics.
    \end{itemize}
    Missing assessments:
    \begin{itemize}
        \item Parameter variation
        \item Different types: Machine / Power Plant connecting transformer; Grid coupler; RONT; \dots: Applicability, what to consider, possible restricctions, \dots
        \item Implementation and Testing of new control proposal
    \end{itemize}
}

%%%%%%%%%%%%%%%%%%%%%%%%%%%%%%%
\section{Development Potential of the FSM and its Control}

% %%%%%%%%%%%%%%%%%%%%%%%%%%%%%%%
% \subsection{Experimental: Extended Ideas and Improvements}
\label{sec:experimental-modeling}

This subsection introduces a few ideas for improvements of the \acs{FSM} voltage controller.
Based on the conducted application studies from \autoref{chap:case-study}, these ideas are not implemented and tested.
% They fall under the aspect modeling, although just approaches are realized and tested.
% As this is based on observations during the development, validation or analyzing case studies, one might consider going through this after understanding the thoughts of the other remaining parts.
% Doing a re-read at the end would be beneficial either way.   

\subsubsection{Alternative Tap Skipping Logic}
\label{sec:modeling-alt-tap-skip}

The first idea is concerning the function \textit{tap$\_$skip()} in the voltage controller of the \acsp{FSM}.
Especially as the in \autoref{sec:validation-fsm-schemes} illustrated results show either a kind \glqq deaf\grqq~band for the \acs{FSM} preferred control loop, this logic is to be questioned for the voltage deviation dependent switching.
In the latter logic, this function \textit{tap$\_$skip()} is making a big influence on the dynamic behavior.
If one would try to formulate the current tap skipping function from \autoref{eq:tap-skip} in words, something in the following form could describe it: 
\begin{quote} \itshape
        How many times does the deadband fit into the voltage deviation? The tap skips are then considered under the amplifying factor of the \acs{FSM} applied on the tap addition of the \acs{OLTC} $\Delta m$.
\end{quote}
As the deadband has less to do with the impact of a \acs{FSM} switch, and it is already respected within the controller activation of both \acs{OLTC} and \acs{FSM} contribution, this relation to the dynamics seems obsolete.
An alternative and seemingly more targeted approach would be sound like: 
\begin{quote} \itshape
        How many switches of the FSM would the current offset voltage bring back to the reference value?
        In translated terms meaning: How many times does one FSM switch fit into the voltage deviation?
\end{quote}
This being translated in mathematical terms, according to \autoref{eq:tap-skip} and the respect of $\eta(t) \in \mathbb{Z}$, the new function for calculating the ideal tap skips by the \acs{FSM} is formulated in \autoref{eq:new-tap-skips}.
\begin{align}
        \eta(t)=\text{floor}\bigg(\frac{\vert \Delta v(t)\vert}{\Delta k \cdot \Delta m}\bigg) \label{eq:new-tap-skips}
\end{align}
This approach should be more accurate for different pairs of preset values, meaning the size of the deadband, the added voltage per tap, the amplifying factor of the \acs{FSM}, etc.
It is expected, that the current logic is resulting in a plausible way, as the ratio addition per tap of the \acs{OLTC} is near the size of the deadband.
This means, that the proposed function is very similar for the present and in this thesis mainly used parameterization of the \acs{FSM} control loop.
However, it is expected that the proposal is thus more robust and better working for the applied cases and more variatons in the configuration.

% \ai{
%         \textit{Or this following green part in the discussion?}

        One comment on this proposal and the original function idea has to made anyway.
        Either logic is solely considering the influence of the tap changer, but no load or grid dynamics at all.
        Even the time constants do not have a influence on the switching behavior.
        As this could also be beneficial for fast responses, the large impact of the novel \acs{FSM} equipped tap changers in a very short time can also irritate other control units or counteract to processes and destabilize a grid area unnecessarily.
        Thus a true voltage dfference or voltage difference gradient based control scheme, under consideration of the individual minimum possible time constants would appear to be logically the best solution.
% }

\subsubsection{Operational Oriented FSM Control}
\mycomment[MK]{Noch ergänzen: Dämpfung als Dynamische Komponente bei Operation Control}
\label{sec:modeling-op-control}


\sidenote{Comment on the representation of time constants}
The time constants of both parts, the \acs{OLTC} and the \acs{FSM}, are relevant as they model the minimal needed duration of the switching.
This minimal time is based on mechanical or electric limitations, such as the mechanical movement of the \acs{OLTC} tap changer.
In the current schemes they are not represented as a limitation, but more just as a delay.
For example, if the voltage difference falls below the deadband, the integrators or time delays are resetted to their initial state. 
The falling below was just an errot or a short swing, so the deadband is suddenly exceeded again after a very short time.
As the switch of the \acs{OLTC} would mechanically move, this sudden exceed would then let the time delay start from the beginning.
While in reality the physical switch has not been given enough time to reach its starting position, the next switch could be achieved faster. 

\sidenote{Corrective supervision}
As afore described, the longer time constants of the \acs{OLTC} come from the mechanical switch movement, which are not the case for the \acs{FSM}.
Therefore the maximum dynamic ability to react on voltage deviations is a lot higher for the \acs{FSM}.
With the move from the preffered \acs{FSM} switching to the voltage dependent activation, a significant step was made towards dynamic influences instead of just a \glqq range extender\grqq.
One could think of even improving this behavior, as keeping the dynamic capabilites through rearranging the positions of $k$ and $m$.
In more static cases a preset of one of the tap changers postion, e.g. the more dynamik $m$, can be restored with keeping the overall ratio constant.
This could be possible through coordinated counter switching of the \acs{OLTC} and the \acs{FSM}.
When considering, that the range of an \acs{OLTC} is typically aroung $k \in [-10,10]$, the FSM seems very limited with $m \in [-4,4]$.
This would utilize the \acs{OLTC} better on a long-term perspective, as not only the FSM would be used for small dynamic deviations.
The described behavior could be named as corrective supervision or monitoring. 

\sidenote{Using voltage gradients}
In order to select or deselect the \acs{FSM} or \acs{OLTC}, the current approach through the tap skipping function seems hand-on and sufficient.
As afore described, it does not account for different time constants and thus durations until the voltage deviation can be corrected.
This calculation is a retrospective procedure, as only the current value is referenced to the voltage setpoint.
If this deviation became too big, the \acs{FSM} switching is initiated.

In contrast to that, if one would account for the current voltage gradient in addition to the current deviation, a prediction over the time constant modeled switching limitation can be estimated.
This would bring the controller in a mode, where the switching activation would be anticipated.
Further a gradient could easily help to determine which part, the \acs{FSM} for more dynamic action or the \acs{OLTC} for more static actions, should be used.
With this idea one could even imagine neglecting a voltage deadband, as a time deadband would be more applicable towards swing characteristics. 
As swings, or then damping of swings in a system, would symptomatically end in the same characteristics as tap hunting, this in between mode could be realized.

The proposed changes can be realized with a split control path, devided into a preset calculation and a physical switching representation.
A provisional scheme or sketch is illustrated in \autoref{fig:sketch-control-improvement}. 
These two compartments are then forked with a corrective supervision to realize off-nominal transformer ratios with optimal dynamic capabilities as pre setted by the operator.

\begin{figure}
        \centering
        \missingfigure{Sketch control improvement}
        \caption[Sketch of an dynamics improved operational oriented control scheme for the \acs{FSM}]{Sketch of an dynamics improved operational oriented control scheme for the \acs{FSM}.}
        \label{fig:sketch-control-improvement}
\end{figure}

\subsubsection{Dynamic Measurement and Reference Voltage Setpoints}

The last, least expected approach to be profitable, is the area of measurement and reference setting.
On the one hand, it could be imagined, tracking the voltages at both busses and thus getting insights on the load flow direction.
The load flow in combination of positioning slack busses are expected to define the switching direction of the tap changer transformers.
Additionally dynamic setpoints are imagined to be calculated as references.
This means, that a new load flow calculation is done every time the load in the network or at least network area changes.
With considering a direct supply of only a load, or at least a construct summarizeable as one load, an additonal block in the control scheme representing Nose Curves can be imagined as suitable approach.
This representation static possible solutions could allow for an automated reference voltage and initial tap changer position.
Especially when considering the afore described operational oriented control.

%%%%%%%%%%%%%%%%%%%%%%%%%%%%%%%
%%%%%%%%%%%%%%%%%%%%%%%%%%%%%%%
\chapter{Summary and Outlook}
\label{chap:summary}

\begin{textblock*}{.7\textwidth}(70mm-\offset,25mm-\offset)
    \begin{fquote}[Robert Frost]
        In three words I can sum up everything I've learned about life: it goes on.
    \end{fquote}
\end{textblock*}

Some conclusion.

\sidenote{Research Questions}
\textbf{1. How do different control types and characteristics of Tap Changing transformers influence the voltage stability of the given system?}

Some summary about Research Question 1. \lipsum[1]

\sidenote{Additional Questions}
\textbf{2. Can the already existing Tap Changer Control of the \acf{FSM} be improved towards a more operation oriented control?}

Some summary about Research Question 2. \lipsum[2]

\sidenote{Further Investigations}
Some outlook and nice blibla. \lipsum[3]

\sidenote{Future outlook on the Fast Switching Module}
Some outlook and nice blibla. \lipsum[3]

\commenting{
    Damping element?\\
    Why not apply it on a phase shifter?\\
    With the so small time constants, nearly no limitations in switching speed: The control is the limiting factor.
    This means for so many Use Cases this technology can be used, as long as there is a sufficient control: Improved Machine integration; Damping effects on power oscillations; Fast Load Flow optimizations; Virtual power plants (Imporved operating points, assistance of inverter controllers via harmonics damping, power oscillation damping, more flexibility, \dots)\\
    Why not substitute the complete mechanical OLTC with multiple FSM modules, maybe even with more range than $m \in [-4,4]$, if the application could use it / coud be improved with that: Improved operational range (faster, bigger) of the grid connection of a rotating phase shifter? 
}
\ai{
    Maybe interesting assessment: Can one prevent the forming of one weak bus? 
    -> If the grid is weak, all busses are getting weaker / problematic during a fault. 
    -> But does ONE get weaker than the others? 
    Or can the FSM prevent that and utilize all busses similar?
}