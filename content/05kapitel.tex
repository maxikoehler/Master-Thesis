%!TEX root = ../main.tex

% \addtocontents{toc}{\protect\addvspace{2.25em}}
% \bookmarksetup{startatroot}

%%%%%%%%%%%%%%%%%%%%%%%%%%%%%%%
%%%%%%%%%%%%%%%%%%%%%%%%%%%%%%%
\chapter{Discussion of the Results}
\label{chap:discussion}

\begin{textblock*}{.7\textwidth}(70mm-\offset,25mm-\offset)
    \begin{fquote}[Joseph Joubert]
        The aim of argument, or discussion, should not be victory, but progress.
    \end{fquote}
\end{textblock*}

This chapter is discussing all chapters combined, with found aspects, expected or unexpected behaviors, comparisons and so on.
Singular discussions in each chapter are avoided, to maintain a high level view on the \acf{FSM}.
Nevertheless, details do matter here and are included in the summarized evaluation.
The structure of this chapter is not holding up with the afore used structure.
This shall make use of the combined discussion, more orientating on the found aspects, than the done work. 

% \section{Transformer Model and Validation}

\ai{Better to structure with the desired statements, instead of afore used structure -> Make use of the combined discussion chapter.}

\commenting{
    One key missing model / aspect, especially for short term voltage stability in combination with stability support of the FSM: \textbf{INDUCTION MOTORS}.

    This thesis i sonly presenting a control circuit for standard OLTCs, and their extensional FSM. 
    Line Drop Compensations are not included, WAMPAC methods are excluded, parallel transformer realizations (N-1 criteria; redundancies), are not considered.
    For recent developments, \textcite{sarimuthu_2016} is presenting a review paper considering a few of these topics.
}

\commenting{
    Different types: Machine / Power Plant connecting transformer; Grid coupler; RONT; \dots: Applicability, what to consider, possible restricctions, \dots

    Back propagation of fast switches to machines.

    Do I have longer time to act on this based on the enhanced functionality or just because of the bigger band?

    Illustration: Just holding the Voltage band longer in the area, where units stay connected -> no reactive power support and therefore stabilization from the \acs{OLTC} alone possible.

    Damping element?

    Deaf band?

    Interactions with other controllers? E.g. Machine controllers.

    Standard: What happens with different parameterization of the FSM, e.g. when deadband conventional (big) used, \dots
}

% \section{Voltage Stability Rating and Assessment}

% \section{Case Studies and Practical Applications}

\section{Comparison FSM vs. OLTC}

\section{Diffpssi Integration}

\commenting{Main point of diffpssi: \textbf{DIFFERENTIABLE}: Then making use of it? Why not optimize the FSM control aspects with it?}

\section{Development Potential of the FSM and its Control}

\commenting{
    Why not apply it on a phase shifter?

    With the so small time constants, nearly no limitations in switching speed: The control is the limiting factor.
    This means for so many Use Cases this technology can be used, as long as there is a sufficient control: Improved Machine integration; Damping effects on power oscillations; Fast Load Flow optimizations; Virtual power plants (Imporved operating points, assistance of inverter controllers via harmonics damping, power oscillation damping, more flexibility, \dots)

    Why not substitute the complete mechanical OLTC with multiple FSM modules, maybe even with more range than $m \in [-4,4]$, if the application could use it / coud be improved with that: Improved operational range (faster, bigger) of the grid connection of a rotating phase shifter? 
}

%%%%%%%%%%%%%%%%%%%%%%%%%%%%%%%
%%%%%%%%%%%%%%%%%%%%%%%%%%%%%%%
\chapter{Summary and Outlook}
\label{chap:summary}

\begin{textblock*}{.7\textwidth}(70mm-\offset,25mm-\offset)
    \begin{fquote}[Robert Frost]
        In three words I can sum up everything I've learned about life: it goes on.
    \end{fquote}
\end{textblock*}

Some conclusion.

\sidenote{Research Questions}
\textbf{1. How do different control types and characteristics of Tap Changing transformers influence the voltage stability of the given system?}

Some summary about Research Question 1. \lipsum[1]

\sidenote{Additional Questions}
\textbf{2. Can the already existing Tap Changer Control of the \acf{FSM} be improved towards a more operation oriented control?}

Some summary about Research Question 2. \lipsum[2]

\sidenote{Further Investigations}
Some outlook and nice blibla. \lipsum[3]

\sidenote{Future outlook on the Fast Switching Module}
Some outlook and nice blibla. \lipsum[3]