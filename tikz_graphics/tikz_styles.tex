%!TEX root = ../main.tex

\usetikzlibrary{positioning,calc,shapes,arrows,shapes.multipart}

% circuitikz: creating a bus
\tikzset{
   bus/.style={fullgeneric, %
        bipoles/fullgeneric/width=0.02, bipoles/fullgeneric/height=#1
   },
   bus/.default=3
}
% Some predefined styles for the blocks
\tikzset{
    block/.style = {draw, fill=white, rectangle, minimum height=3em, minimum width=3em},
    tmp/.style  = {coordinate}, 
    sum/.style= {draw, fill=white, circle, node distance=1cm},
    input/.style = {coordinate},
    output/.style= {coordinate},
    pinstyle/.style = {pin edge={to-,thin,black}}
    limblock/.style = {
            draw, 
            fill=white, 
            rectangle, 
            minimum height=3em, 
            minimum width=3em
        },
}
\newcommand{\bushere}[3]{% length, text above, text below
% optional arguments do not work in paths
    %
    % starting point; draw an edge and then two nodes
    % save the position
    coordinate(tmp)
    % go up and do an edge down
    ++(0,#1) node[anchor=base]{#2} edge[ultra thick] ++(0,{-2*#1})
    % edges do not move the current point, go down to position the node
    ++(0,{-2*#1}) node[below]{#3}
    % go back to where we started
    (tmp)
}
\newcommand{\qbushere}[3]{% length, text above, text below
% optional arguments do not work in paths
    %
    % starting point; draw an edge and then two nodes
    % save the position
    coordinate(tmp)
    % go up and do an edge down
    ++(#1,0) node[anchor=base]{#2} edge[ultra thick] ++({-2*#1},0)
    % edges do not move the current point, go down to position the node
    ++({-2*#1},0) node[below]{#3}
    % go back to where we started
    (tmp)
}
\newcommand{\loadside}[3]{
   coordinate(tmp)
   -| ++(0.5,-1) node[fill=white,shape=regular polygon, rotate=180, regular polygon sides=3,minimum size=0.8,draw](){} ++(#2,#3) node[below]{#1}
   (tmp)
}
\newcommand{\loadhanging}[3]{
   coordinate(tmp)
   -- ++(0,-1) node[fill=white,shape=regular polygon, rotate=180, regular polygon sides=3,minimum size=0.8,draw](){} ++(#2,#3) node[below]{#1}
   (tmp)
}

% program plan
\tikzset{
   papDecision/.style = {
         diamond,
         draw, 
         text width = 20 mm, 
         align = center, 
         text badly centered,
         inner sep = 1 pt,
         font=\ttfamily\footnotesize,
         %line width = 1,
         minimum width = 30mm,
         minimum height = 7mm,
      },
   papStart/.style = {
         rectangle,
         draw, 
         align = center, 
         text width = 3cm, 
         text badly centered,
         inner sep = 4 pt,
         rounded corners=10pt,
         font=\ttfamily\footnotesize,
         %line width = 1,
         minimum width = 30mm,
         minimum height = 7mm,
      },
   papEnd/.style = {
         rectangle,
         draw, 
         align = center, 
         text width = 3cm, 
         text badly centered,
         inner sep = 4 pt,
         rounded corners=10pt,
         font=\ttfamily\footnotesize,
         %line width = 1,
         minimum width = 30mm,
         minimum height = 7mm,
      },
   papData/.style = {
         trapezium,
         draw, 
         align = center, 
         text width = 20 mm, 
         text badly centered,
         inner sep = 4 pt,
         trapezium left angle=70,
         trapezium right angle=110,
         font=\ttfamily\footnotesize,
         %line width = 1,
         minimum width = 30mm,
         minimum height = 7mm,
      },
   papPredProc/.style = {
         draw,
         rectangle split,
         rectangle split horizontal,
         rectangle split parts = 3,
         rectangle split empty part width=-8pt,
         align = center, 
 %       text width = 4.5 em, 
         text badly centered,
 %        inner sep = 4 pt,
         font=\ttfamily\footnotesize,
         %line width = 1,
         minimum width = 30mm,
         minimum height = 7mm,
      },
   papProcess/.style = {
         rectangle,
         draw,
         align = center, 
         text width = 3cm, 
         text badly centered,
         %inner sep = 2 pt,
         font=\ttfamily\footnotesize,
         %line width = 1,
         minimum width = 30mm,
         minimum height = 7mm,
      },
   papLine/.style = {
         draw,
         -stealth,
         font=\ttfamily\footnotesize,
         %line width = 1,
      },
}
\newcommand{\papYes}{ja}
\newcommand{\papNo}{nein}