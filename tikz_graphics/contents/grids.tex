\chapter{Grids}

\lipsum[2]

\vspace{1cm}

% \begin{tikzpicture}
%     % \vspace{10cm}
%     \begin{circuitikz}[european, scale=.9, smallR/.style={resistor,resistors/scale=.7}]
%         \draw (0,0) node[oscillator, anchor=east, name=gen]{} --(.5,0)
%         to[L, name=X_g] ++(2,0) coordinate(f1)
%         % \bushere{1}{$\underline{E}_\mathrm{T}'$}{}
%         % to[oosourcetrans,prim=delta,sec=wye] ++(2,0)
%         \bushere{2}{}{} coordinate(b1) ++(0,-1) -- ++(.25,0) -- coordinate(f2) ++(1.75,0)
%         to[L, name=X_l3] ++(1,0) -- ++(1.5,0) ++(0,1)
%         \bushere{2}{$E_\mathrm{ibb}~\angle~0^{\circ}$}{} ++(0,1) coordinate(b2) ++(0,-1) coordinate(b3) -- ++(.75,0) coordinate(f3) -- ++(.75,0) to[L, name=X_ibb] ++(1,0) -- ++(1,0)
%         node[gridnode, anchor=left, name=ib]{};
    
%         % draw other resistances
%         \draw (b1) ++(0,1) -- ++(.5,0) to[L, name=X_l1] (b2);
%         \draw (b1) -- ++(.5,0) to[L, name=X_l2] (b3);
%         % \draw[line width=2pt] (2.25,1) -- (2.25,-1);
%         % \draw[line width=2pt] (4.75,1) -- (4.75,-1);
%         % \draw[line width=2pt] (8.25,1) -- (8.25,-1);
    
%         % labels for the resistors
%         \node[above=6pt] at (X_g) {$X_\mathrm{g}'$};
%         \node[above=6pt] at (X_ibb) {$X_\mathrm{ibb}$};
%         \node[above=6pt] at (X_l1) {$3~X_\mathrm{l}$};
%         % \node[below=6pt] at (X_l2) {$X_\mathrm{l}$};
%         % \node[below=6pt] at (X_l3) {$X_\mathrm{l}$};
    
%         % pole voltages and angles
%         \path[->] (-1.2,.5) edge [bend right] node[left=6pt]{$E_\mathrm{p}~\angle~\delta$} (-1.2,-.5);
%         % \path[->] (ib) ++(.8,.5) edge [bend left] node[right=6pt]{$E_\mathrm{ibb}~\angle~0^{\circ}$} ++(0,-1);
    
%         % faults
%         % \draw[-Stealth, very thick, red] (f1) ++(0,-.5) -- ++(-.15,-.45) -- ++(.3,.2) -- ++(-.2,-.6) coordinate(f1_text);
%         % \node[below, red] at (f1_text) {\scriptsize fault 1};
%         % \draw[-Stealth, very thick, red] (f2) ++(0,.3) -- ++(-.15,-.45) -- ++(.3,.2) -- ++(-.2,-.6) coordinate(f2_text);
%         % \node[below, red, align=center] at (f2_text) {\scriptsize Fehler 2/3};
%         % \draw[-Stealth, very thick, red] (f3) ++(0,.3) -- ++(-.15,-.45) -- ++(.3,.2) -- ++(-.2,-.6) coordinate(f3_text);
%         % \node[below, red] at (f3_text) {\scriptsize Fehler 1};
%     \end{circuitikz}
% \end{tikzpicture}


\vspace{2cm}
\tikzsetnextfilename{sm_load_model}
\begin{figure}[htb]
    \begin{tikzpicture}[european, scale=.9, smallR/.style={resistor,resistors/scale=.7}]
        % \draw [help lines] (-1,-5) grid (15,5);
        \small
        \draw (0,0) node[oscillator, anchor=east, name=gen]{} --(.5,0)
        to ++(.5,0) \bushere{1}{}{Bus 2}
        to[oosourcetrans] ++(2,0) 
        \bushere{1}{}{Bus 1} -- ++(1.5,0) coordinate(line) -- ++(1.5,0)
        \bushere{2}{}{Bus 0} coordinate(b0);
        \draw (b0) ++(0,1) -- ++(1,0) \loadside{Load 1}{1}{0};
        \draw (b0) ++(0,-1) -- ++(1,0) \loadside{Load 2}{1}{0} ++(1,0);
    \end{tikzpicture}
    \caption{Single line network with two loads}
\end{figure}

\tikzsetnextfilename{smib_model}
\begin{figure}[htb]
    \begin{tikzpicture}[european, scale=.9, smallR/.style={resistor,resistors/scale=.7}]
        % \draw [help lines] (-1,-5) grid (15,5);
        \draw (0,0) node[oscillator, anchor=east, name=gen]{} --(.5,0)
        to ++(.5,0) \bushere{1}{}{Bus 2}
        to[oosourcetrans] ++(2,0) 
        \bushere{1}{}{Bus 1} -- ++(3,0) coordinate(line) -- ++(3,0)
        \bushere{1}{}{Bus 0}
        -- ++(1,0) node[gridnode, anchor=left, name=ib]{};
    \end{tikzpicture}
    \caption{SMIB model}
\end{figure}

\tikzsetnextfilename{smib_model_with_load}
\begin{figure}[htb]
    \begin{tikzpicture}[european, scale=.9, smallR/.style={resistor,resistors/scale=.7}]
        % \draw [help lines] (-1,-5) grid (15,5);
        \draw (0,0) node[oscillator, anchor=east, name=gen]{} --(.5,0)
        to ++(.5,0) \bushere{1}{}{Bus 2}
        to[oosourcetrans] ++(2,0) coordinate(trafo_b1) ++(0,-0.5)
        \bushere{1.5}{}{Bus 1} ++(0,-0.5) -- ++(1,0) \loadside{Load 1}{1}{0};
        \draw (trafo_b1) -- ++(3,0) coordinate(line) -- ++(3,0)
        \bushere{1}{}{Bus 0}
        -- ++(1,0) node[gridnode, anchor=left, name=ib]{};
    \end{tikzpicture}
    \caption{SMIB model with additional load}
\end{figure}

\tikzsetnextfilename{4bus3load_model_random}
\begin{figure}[htb]
    \begin{tikzpicture}[european, scale=.9, smallR/.style={resistor,resistors/scale=.7}]
        % \draw [help lines] (-5,-10) grid (5,0);
        \draw (0,0) node[oscillator, anchor=south, name=gen]{} --(0,-0.5)
        to ++(0,-0.5) \qbushere{2}{}{} ++(-1,0) coordinate(trafo)
        to[oosourcetrans] ++(0,-3)
        \qbushere{1}{}{} coordinate(b2) ++(.25,0) |- ++(2,-1) \bushere{1}{}{} -- ++(1,0) \loadhanging{Load 3}{0}{-.5};
        \draw (b2) ++(-.25,0) |- ++(-2,-1) \bushere{1}{}{} -- ++(-1,0) \loadhanging{Load 2}{0}{-.5};
        \draw (trafo) ++(2,0) \loadhanging{Load 1}{0}{-.5};
    \end{tikzpicture}
    \caption{Random network with three loads on multiple voltage levels}
\end{figure}

% \begin{figure}[htb]
%     \begin{circuitikz}[european, scale=.9, smallR/.style={resistor,resistors/scale=.7}]
%         \small
%         \node [oscillator, anchor=east](gen1){};
%         \node [gridnode, anchor=left](ibb){};
    
%         \draw (gen1) -- ++(1,0) \bushere{1}{}{} ++( 1,0) to[oosourcetrans] -- (ibb);
%         \draw (0,0) node[oscillator, anchor=east, name=gen]{} --(.5,0)
%         to ++(.5,0) \bushere{1}{}{Bus 2}
%         to[oosourcetrans] ++(2,0) 
%         \bushere{1}{}{Bus 1} -- ++(3,0) coordinate(line) -- ++(3,0)
%         \bushere{1}{}{Bus 0}
%         -- ++(1,0) node[gridnode, anchor=left, name=ib]{};
%     \end{circuitikz}
%     \caption{}
% \end{figure}

\tikzsetnextfilename{transformer_complete}
\begin{figure}[htb]
    \centering
    \begin{tikzpicture}[european, scale=.9, smallR/.style={resistor,resistors/scale=.7}]
        \footnotesize
        % \ctikzset{quadpoles/transformer/height=2.5}
        % \draw [help lines] (-1,-5) grid (15,5);

        \draw node[transformer, american] (T) at (8, 0) {} node[anchor=north] at ([yshift=-0.38cm]T.south){$\underline{\vartheta}:1$};
        \draw[-{Latex[length=2mm]}, thick] (T.outer dot A2) -- (T.outer dot B1);
        
        \draw (T.A1) to [L, l_=$\underline{Z}_1$, label distance=4pt] ++(-3,0) coordinate(cross) to [L, l_=$\underline{Z}_2$, label distance=4pt] ++(-3,0) node[ocirc](v1){};
        \draw (T.A2) -- ++(-3,0) coordinate(cross_2) -- ++(-3,0) node[ocirc]{};
        \draw (cross) to [L, l_=$\underline{Z}_\mathrm{Fe, \mu}$, label distance=4pt] (cross_2);
        
        \draw (T.B1) -- ++(1,0) coordinate(v2_1) node[ocirc]{};
        \draw (T.B2) -- ++(1,0) coordinate(v2_2) node[ocirc]{};
        \draw[-{Latex}, color=ees_black] (v2_1) ++(0,-.2) -- node[anchor=west]{$\underline{U}_1$} ++(0,-1.9);
        \node[below of=v2_2, anchor=south east](lv){LV side};
        \draw[-{Latex}, color=ees_black] (v1) ++(0,-.2) -- node[anchor=east]{$\underline{U}_2$} ++(0,-1.9);
        
        \node[below of=cross_2, anchor=south](hv){HV side};   
        \node[ocirc] at (T.A1){};
        \node[ocirc] at (T.A2){};
        \node[ocirc] at (T.B1){};
        \node[ocirc] at (T.B2){};
    \end{tikzpicture}
    \caption{Complete transformer circuit}
\end{figure}


\tikzsetnextfilename{transformer_reduced}
\begin{figure}[htb]
    \centering
    \begin{tikzpicture}[european, scale=.9, smallR/.style={resistor,resistors/scale=.7}]
        \footnotesize
        % \ctikzset{quadpoles/transformer/height=2.5}
        % \draw [help lines] (-1,-5) grid (15,5);

        \draw node[transformer, american] (T) at (8, 0) {} node[anchor=north] at ([yshift=-0.38cm]T.south){$\underline{\vartheta}:1$};
        \draw[-{Latex[length=2mm]}, thick] (T.outer dot A2) -- (T.outer dot B1);
        
        \draw (T.A1) to[short, i_>=$\color{ees_red}\underline{\vartheta}^*\underline{I}_1$, color=ees_red] ++(-1,0) to [L, l_=$\underline{Y}_\mathrm{T}$, label distance=4pt] ++(-2,0) to[short, i_<=$\color{ees_red}\underline{I}_2$, color=ees_red] ++(-1,0) node[ocirc](v1){};
        \draw (T.A2) -- ++(-2,0) coordinate(cross_2) -- ++(-2,0) node[ocirc]{};
        \node[below of=cross_2, anchor=south](hv){HV side};
        
        \draw[-{Latex}, color=ees_red] (T.A1) ++(0,-.2) -- node[anchor=east]{$\frac{1}{\underline{\vartheta}} \underline{U}_1$} ++(0,-1.9);
        \draw[-{Latex}, color=ees_black] (v1) ++(0,-.2) -- node[anchor=east]{$\underline{U}_2$} ++(0,-1.9);
        
        \draw (T.B1) -- ++(.5,0) to[short, i<=$\underline{I}_1$] ++(1,0) node[ocirc](v2_1){};
        \draw (T.B2) -- ++(1.5,0) node[ocirc](v2_2){};
        \draw[-{Latex}, color=ees_black] (v2_1) ++(0,-.2) -- node[anchor=west]{$\underline{U}_1$} ++(0,-1.9);
        \node[below of=v2_2, anchor=south east](lv){LV side};

        \node[ocirc] at (T.A1){};
        \node[ocirc] at (T.A2){};
        \node[ocirc] at (T.B1){};
        \node[ocirc] at (T.B2){};
    \end{tikzpicture}
    \caption{Reduced transformer circuit; based on Ilyas calculation}
\end{figure}


\tikzsetnextfilename{transformer_pi}
\begin{figure}[htb]
    \centering
    \begin{tikzpicture}[european, scale=.9, smallR/.style={resistor,resistors/scale=.7}]
        \footnotesize
        \draw (0,0) node[ocirc](v1_1){} to[short, i>=$\color{ees_red}\underline{I}_1$, color=ees_red] ++(1,0) -- ++(1,0) coordinate(cross1) to [L, name=12] ++(5,0) coordinate(cross3) -- ++(1,0) to[short, i<=$\color{ees_red}\underline{I}_2$, color=ees_red] ++(1,0) node[ocirc](v2_1){};
        \draw (0,-3) node[ocirc](v1_2){} -- ++(2,0) coordinate(cross2) -- ++(5,0) coordinate(cross4) -- ++(2,0) node[ocirc](v2_2){};

        \draw (cross1) to [L=$\underline{Z}_{10}$, name=10, label distance=4pt] (cross2);
        \draw (cross3) to [L=$\underline{Z}_{20}$, name=20, label distance=4pt] (cross4);

        \draw[-{Latex}, color=ees_red] (v1_1) ++(0,-.2) -- node[anchor=east]{$\color{ees_red}\underline{U}_1$} ++(0,-2.6);
        
        \draw[-{Latex}, color=ees_red] (v2_1) ++(0,-.2) -- node[anchor=west]{$\color{ees_red}\underline{U}_2$} ++(0,-2.6);

        \node[above right of=10, anchor=west]{$\underline{Z}_{10}=\frac{1}{\underline{\vartheta}} \cdot \underline{Y}_\mathrm{T}$};
        \node[below left of=20, anchor=east]{$\underline{Z}_{20}=\frac{1}{\underline{\vartheta}} \cdot \underline{Y}_\mathrm{T}$};
        \node[above of=12, anchor=center]{$\underline{Z}_{12}=\frac{1}{\underline{\vartheta}} \cdot \underline{Y}_\mathrm{T}$; $\underline{Z}_{21}=\frac{1}{\underline{\vartheta}^*} \cdot \underline{Y}_\mathrm{T}$};
    \end{tikzpicture}
    \caption{Transformer Pi circuit}
\end{figure}