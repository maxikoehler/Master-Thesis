\chapter{Grids}

\lipsum[2]

\vspace{1cm}

% \begin{tikzpicture}
%     % \vspace{10cm}
%     \begin{circuitikz}[european, scale=.9, smallR/.style={resistor,resistors/scale=.7}]
%         \draw (0,0) node[oscillator, anchor=east, name=gen]{} --(.5,0)
%         to[L, name=X_g] ++(2,0) coordinate(f1)
%         % \bushere{1}{$\underline{E}_\mathrm{T}'$}{}
%         % to[oosourcetrans,prim=delta,sec=wye] ++(2,0)
%         \bushere{2}{}{} coordinate(b1) ++(0,-1) -- ++(.25,0) -- coordinate(f2) ++(1.75,0)
%         to[L, name=X_l3] ++(1,0) -- ++(1.5,0) ++(0,1)
%         \bushere{2}{$E_\mathrm{ibb}~\angle~0^{\circ}$}{} ++(0,1) coordinate(b2) ++(0,-1) coordinate(b3) -- ++(.75,0) coordinate(f3) -- ++(.75,0) to[L, name=X_ibb] ++(1,0) -- ++(1,0)
%         node[gridnode, anchor=left, name=ib]{};
    
%         % draw other resistances
%         \draw (b1) ++(0,1) -- ++(.5,0) to[L, name=X_l1] (b2);
%         \draw (b1) -- ++(.5,0) to[L, name=X_l2] (b3);
%         % \draw[line width=2pt] (2.25,1) -- (2.25,-1);
%         % \draw[line width=2pt] (4.75,1) -- (4.75,-1);
%         % \draw[line width=2pt] (8.25,1) -- (8.25,-1);
    
%         % labels for the resistors
%         \node[above=6pt] at (X_g) {$X_\mathrm{g}'$};
%         \node[above=6pt] at (X_ibb) {$X_\mathrm{ibb}$};
%         \node[above=6pt] at (X_l1) {$3~X_\mathrm{l}$};
%         % \node[below=6pt] at (X_l2) {$X_\mathrm{l}$};
%         % \node[below=6pt] at (X_l3) {$X_\mathrm{l}$};
    
%         % pole voltages and angles
%         \path[->] (-1.2,.5) edge [bend right] node[left=6pt]{$E_\mathrm{p}~\angle~\delta$} (-1.2,-.5);
%         % \path[->] (ib) ++(.8,.5) edge [bend left] node[right=6pt]{$E_\mathrm{ibb}~\angle~0^{\circ}$} ++(0,-1);
    
%         % faults
%         % \draw[-Stealth, very thick, red] (f1) ++(0,-.5) -- ++(-.15,-.45) -- ++(.3,.2) -- ++(-.2,-.6) coordinate(f1_text);
%         % \node[below, red] at (f1_text) {\scriptsize fault 1};
%         % \draw[-Stealth, very thick, red] (f2) ++(0,.3) -- ++(-.15,-.45) -- ++(.3,.2) -- ++(-.2,-.6) coordinate(f2_text);
%         % \node[below, red, align=center] at (f2_text) {\scriptsize Fehler 2/3};
%         % \draw[-Stealth, very thick, red] (f3) ++(0,.3) -- ++(-.15,-.45) -- ++(.3,.2) -- ++(-.2,-.6) coordinate(f3_text);
%         % \node[below, red] at (f3_text) {\scriptsize Fehler 1};
%     \end{circuitikz}
% \end{tikzpicture}


\vspace{2cm}
\tikzsetnextfilename{sm_load_model.pdf}
\begin{figure}[htb]
    \begin{tikzpicture}[european, scale=.9, smallR/.style={resistor,resistors/scale=.7}]
        % \draw [help lines] (-1,-5) grid (15,5);
        \small
        \draw (0,0) node[oscillator, anchor=east, name=gen]{} --(.5,0)
        to ++(.5,0) \bushere{1}{}{Bus 2}
        to[oosourcetrans] ++(2,0) 
        \bushere{1}{}{Bus 1} -- ++(1.5,0) coordinate(line) -- ++(1.5,0)
        \bushere{2}{}{Bus 0} coordinate(b0);
        \draw (b0) ++(0,1) -- ++(1,0) \loadside{Load 1}{1}{0};
        \draw (b0) ++(0,-1) -- ++(1,0) \loadside{Load 2}{1}{0} ++(1,0);
    \end{tikzpicture}
    \caption{Single line network with two loads}
\end{figure}

\tikzsetnextfilename{smib_model.pdf}
\begin{figure}[htb]
    \begin{tikzpicture}[european, scale=.9, smallR/.style={resistor,resistors/scale=.7}]
        % \draw [help lines] (-1,-5) grid (15,5);
        \draw (0,0) node[oscillator, anchor=east, name=gen]{} --(.5,0)
        to ++(.5,0) \bushere{1}{}{Bus 2}
        to[oosourcetrans] ++(2,0) 
        \bushere{1}{}{Bus 1} -- ++(3,0) coordinate(line) -- ++(3,0)
        \bushere{1}{}{Bus 0}
        -- ++(1,0) node[gridnode, anchor=left, name=ib]{};
    \end{tikzpicture}
    \caption{SMIB model}
\end{figure}

\tikzsetnextfilename{smib_model_with_load.pdf}
\begin{figure}[htb]
    \begin{tikzpicture}[european, scale=.9, smallR/.style={resistor,resistors/scale=.7}]
        % \draw [help lines] (-1,-5) grid (15,5);
        \draw (0,0) node[oscillator, anchor=east, name=gen]{} --(.5,0)
        to ++(.5,0) \bushere{1}{}{Bus 2}
        to[oosourcetrans] ++(2,0) coordinate(trafo_b1) ++(0,-0.5)
        \bushere{1.5}{}{Bus 1} ++(0,-0.5) -- ++(1,0) \loadside{Load 1}{1}{0};
        \draw (trafo_b1) -- ++(3,0) coordinate(line) -- ++(3,0)
        \bushere{1}{}{Bus 0}
        -- ++(1,0) node[gridnode, anchor=left, name=ib]{};
    \end{tikzpicture}
    \caption{SMIB model with additional load}
\end{figure}

\tikzsetnextfilename{4bus3load_model_random.pdf}
\begin{figure}[htb]
    \begin{tikzpicture}[european, scale=.9, smallR/.style={resistor,resistors/scale=.7}]
        % \draw [help lines] (-5,-10) grid (5,0);
        \draw (0,0) node[oscillator, anchor=south, name=gen]{} --(0,-0.5)
        to ++(0,-0.5) \qbushere{2}{}{} ++(-1,0) coordinate(trafo)
        to[oosourcetrans] ++(0,-3)
        \qbushere{1}{}{} coordinate(b2) ++(.25,0) |- ++(2,-1) \bushere{1}{}{} -- ++(1,0) \loadhanging{Load 3}{0}{-.5};
        \draw (b2) ++(-.25,0) |- ++(-2,-1) \bushere{1}{}{} -- ++(-1,0) \loadhanging{Load 2}{0}{-.5};
        \draw (trafo) ++(2,0) \loadhanging{Load 1}{0}{-.5};
    \end{tikzpicture}
    \caption{Random network with three loads on multiple voltage levels}
\end{figure}

% \begin{figure}[htb]
%     \begin{circuitikz}[european, scale=.9, smallR/.style={resistor,resistors/scale=.7}]
%         \small
%         \node [oscillator, anchor=east](gen1){};
%         \node [gridnode, anchor=left](ibb){};
    
%         \draw (gen1) -- ++(1,0) \bushere{1}{}{} ++( 1,0) to[oosourcetrans] -- (ibb);
%         \draw (0,0) node[oscillator, anchor=east, name=gen]{} --(.5,0)
%         to ++(.5,0) \bushere{1}{}{Bus 2}
%         to[oosourcetrans] ++(2,0) 
%         \bushere{1}{}{Bus 1} -- ++(3,0) coordinate(line) -- ++(3,0)
%         \bushere{1}{}{Bus 0}
%         -- ++(1,0) node[gridnode, anchor=left, name=ib]{};
%     \end{circuitikz}
%     \caption{}
% \end{figure}

